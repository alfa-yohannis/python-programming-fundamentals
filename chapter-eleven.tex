\chapter{Tkinter in Python}

\section{Pengenalan Tkinter}

Tkinter adalah pustaka standar Python yang digunakan untuk membuat antarmuka grafis (Graphical User Interface/GUI). Dengan Tkinter, program tidak lagi hanya berjalan di terminal, tetapi dapat menampilkan jendela, tombol, teks, kotak input, dan berbagai elemen visual lainnya. Karena sudah termasuk dalam instalasi Python, Tkinter menjadi pilihan ideal bagi mahasiswa pemula yang ingin mengenal konsep GUI tanpa perlu menginstal modul tambahan.

Berbeda dengan program berbasis teks yang berjalan secara berurutan, aplikasi Tkinter bekerja menggunakan pendekatan \textit{event-driven}, yaitu program merespons tindakan pengguna seperti klik tombol atau mengetik. Konsep ini membantu mahasiswa memahami bagaimana aplikasi modern bekerja. Tkinter juga menyediakan berbagai \textit{widget} dasar seperti \textit{label}, \textit{button}, \textit{entry}, dan \textit{menu}, yang akan dibahas lebih detail pada bagian-bagian berikutnya.

\section{Membuat Window Utama}

Selain menggunakan pendekatan prosedural, Tkinter juga dapat dibangun dengan gaya pemrograman berorientasi objek. Cara ini umum digunakan dalam aplikasi berskala lebih besar karena membuat struktur program lebih rapi, modular, dan mudah dikembangkan. Dengan pendekatan ini, jendela utama biasanya dibuat sebagai sebuah kelas yang mewarisi \texttt{tk.Tk}, sehingga seluruh konfigurasi jendela dapat dikelola langsung di dalam konstruktor kelas.

Ketika aplikasi dijalankan, sebuah objek dari kelas tersebut dibuat dan kemudian menjalankan \texttt{mainloop()} seperti biasa. Pendekatan ini memudahkan penambahan komponen baru di masa depan dan memungkinkan mahasiswa memahami bagaimana antarmuka Tkinter dapat diorganisasikan dengan lebih baik dalam struktur kelas.

\begin{lstlisting}[style=PythonStyle, caption={Membuat Window Utama Tkinter dengan Pendekatan Class}]
import tkinter as tk

# Membuat kelas untuk window utama
class MainWindow(tk.Tk):
    def __init__(self):
        super().__init__()

        # Mengatur judul jendela
        self.title("Aplikasi Tkinter Pertama")

        # Mengatur ukuran jendela
        self.geometry("400x300")

# Menjalankan aplikasi
if __name__ == "__main__":
    app = MainWindow()
    app.mainloop()
\end{lstlisting}

Pada kode di atas, kelas \texttt{MainWindow} didefinisikan dengan mewarisi \texttt{tk.Tk}. Di dalam metode \texttt{\_\_init\_\_()}, jendela utama dikonfigurasi dengan judul serta ukuran awal. Ketika objek \texttt{MainWindow} dibuat, seluruh pengaturan ini langsung diterapkan. Terakhir, \texttt{mainloop()} dipanggil untuk menjaga jendela tetap aktif dan merespons interaksi pengguna.



\section{Label}

Label adalah salah satu widget paling dasar di Tkinter yang digunakan untuk menampilkan teks atau gambar di dalam jendela aplikasi. Widget ini sering dipakai untuk memberikan judul, penjelasan, atau informasi statis kepada pengguna. Untuk membuat sebuah label, Tkinter menyediakan kelas \texttt{Label} yang dapat dihubungkan dengan jendela utama atau \textit{frame}. Pembuatan label biasanya dimulai dengan menentukan teks yang ingin ditampilkan, kemudian menyimpannya dalam sebuah variabel untuk digunakan di dalam program.

Setelah label dibuat, pengembang dapat mengatur berbagai properti seperti ukuran font, warna teks, warna latar belakang, serta posisi perataan teks. Properti-properti ini membantu menyesuaikan tampilan label agar sesuai dengan desain aplikasi. Selain itu, label juga dapat menampilkan gambar menggunakan objek \texttt{PhotoImage}, sehingga memungkinkan penggunaan ikon atau ilustrasi kecil dalam antarmuka.

Agar label terlihat oleh pengguna, widget ini harus ditempatkan pada jendela menggunakan salah satu metode tata letak Tkinter seperti \texttt{pack()}, \texttt{grid()}, atau \texttt{place()}. Tanpa pemanggilan metode tata letak, label tidak akan muncul meskipun sudah dibuat. Dengan menguasai penggunaan label dan propertinya, mahasiswa dapat mulai membangun antarmuka yang lebih informatif dan terstruktur.

\begin{lstlisting}[style=PythonStyle, caption={Contoh Penggunaan Label di Tkinter}]
import tkinter as tk

root = tk.Tk()
root.title("Contoh Label")

# Membuat label dengan teks sederhana
label1 = tk.Label(root, text="Halo, selamat datang di Tkinter!")

# Mengatur properti label: font, warna teks, dan latar
label2 = tk.Label(root,
                  text="Ini adalah label dengan properti.",
                  font=("Arial", 14),
                  fg="blue",
                  bg="lightgray")

# Menampilkan label pada window
label1.pack(pady=10)
label2.pack(pady=10)

root.mainloop()
\end{lstlisting}

Penjelasan kode di atas dimulai dengan mengimpor modul Tkinter untuk mengaktifkan seluruh fungsionalitas GUI. Baris \texttt{root = tk.Tk()} membuat jendela utama aplikasi, kemudian \texttt{root.title("Contoh Label")} memberikan judul pada jendela tersebut. Selanjutnya, \texttt{label1} dibuat sebagai label sederhana dengan teks biasa, sementara \texttt{label2} menunjukkan bagaimana sebuah label dapat diberi properti tambahan seperti ukuran font, warna teks, dan warna latar belakang. Setelah label dibuat, keduanya ditampilkan pada jendela menggunakan metode \texttt{pack()} dengan jarak vertikal tambahan melalui parameter \texttt{pady}. Terakhir, \texttt{root.mainloop()} memastikan jendela tetap tampil dan merespons interaksi pengguna hingga ditutup secara manual.



\section{Button}

Button adalah widget Tkinter yang berfungsi sebagai tombol interaktif yang dapat ditekan oleh pengguna untuk menjalankan suatu aksi tertentu. Tombol ini merupakan komponen penting dalam antarmuka grafis karena digunakan untuk memicu berbagai fungsi, seperti menampilkan pesan, mengubah nilai komponen lain, menyimpan data, atau membuka jendela tambahan. Dengan menggunakan kelas \texttt{Button}, pengembang dapat membuat tombol dan memberikan teks yang menjelaskan fungsi tombol tersebut.

Agar tombol dapat melakukan sesuatu saat ditekan, diperlukan sebuah \textit{event handler} yang dihubungkan melalui parameter \texttt{command}. Event handler ini biasanya berupa metode di dalam sebuah kelas apabila pendekatan berbasis objek digunakan. Dalam pendekatan ini, tombol dapat berinteraksi dengan widget lain, misalnya mengubah teks pada label. Interaksi sederhana seperti ini membantu mahasiswa memahami bagaimana komponen GUI saling berhubungan dalam sebuah aplikasi yang dibangun secara terstruktur dengan OOP.

\begin{lstlisting}[style=PythonStyle, caption={Contoh Button dan Interaksi dengan Label Menggunakan Class}]
import tkinter as tk

class ButtonDemo(tk.Tk):
    def __init__(self):
        super().__init__()
        self.title("Contoh Button")
        self.geometry("300x200")

        # Label awal
        self.label = tk.Label(self, text="Teks awal")
        self.label.pack(pady=10)

        # Button yang memanggil method saat ditekan
        self.button = tk.Button(self, text="Klik Saya", command=self.ubah_teks)
        self.button.pack(pady=10)

    # Event handler dalam bentuk method
    def ubah_teks(self):
        self.label.config(text="Tombol ditekan!")

if __name__ == "__main__":
    app = ButtonDemo()
    app.mainloop()
\end{lstlisting}

Pada contoh di atas, kelas \texttt{ButtonDemo} dibuat dengan mewarisi \texttt{tk.Tk}. Di dalam konstruktor \texttt{\_\_init\_\_()}, sebuah label ditampilkan sebagai teks awal, kemudian sebuah tombol dibuat dan dihubungkan dengan method \texttt{ubah\_teks()} melalui parameter \texttt{command}. Ketika tombol ditekan, method tersebut dipanggil dan label diperbarui dengan teks baru. Dengan pendekatan ini, mahasiswa dapat melihat bagaimana penggunaan class membuat struktur aplikasi lebih rapi serta memudahkan pengelolaan interaksi antar komponen.


\section{EditText (Entry)}

Entry adalah widget Tkinter yang digunakan untuk menerima input teks dari pengguna. Widget ini sering digunakan untuk memasukkan data sederhana seperti nama atau angka. Dalam aplikasi dasar, Entry biasanya dikombinasikan dengan Label sebagai tampilan hasil dan Button untuk memicu proses penyalinan atau pemrosesan data. Salah satu contoh interaksi paling sederhana adalah menyalin isi Entry ke Label ketika tombol ditekan.

Untuk mengambil nilai dari Entry, Tkinter menyediakan metode \texttt{get()}, sementara Label dapat diperbarui menggunakan metode \texttt{config()}. Dengan menggabungkan komponen-komponen ini di dalam sebuah kelas, mahasiswa dapat memahami pola dasar interaksi antara widget dalam antarmuka grafis.

\begin{lstlisting}[style=PythonStyle, caption={Menyalin Teks dari Entry ke Label Menggunakan Class}]
import tkinter as tk

class EntryDemo(tk.Tk):
    def __init__(self):
        super().__init__()
        self.title("Contoh Entry dan Label")
        self.geometry("300x150")

        # Label hasil
        self.label = tk.Label(self, text="Hasil akan tampil di sini")
        self.label.pack(pady=5)

        # Entry untuk input
        self.entry = tk.Entry(self)
        self.entry.pack(pady=5)

        # Tombol untuk menyalin teks
        self.button = tk.Button(self, text="Tampilkan", command=self.copy_text)
        self.button.pack(pady=5)

    # Method untuk menyalin teks dari Entry ke Label
    def copy_text(self):
        teks = self.entry.get()
        self.label.config(text=teks)

if __name__ == "__main__":
    app = EntryDemo()
    app.mainloop()
\end{lstlisting}

Pada contoh ini, pengguna memasukkan teks ke dalam Entry, lalu ketika tombol ditekan, method \texttt{copy\_text()} mengambil teks tersebut melalui \texttt{get()} dan menampilkannya pada Label. Contoh ini menunjukkan interaksi dasar antar widget dan merupakan fondasi untuk membangun form atau aplikasi input yang lebih kompleks.


\section{Layout pada Tkinter}

\subsection{Pengenalan Sistem Layout}

Tkinter menyediakan tiga metode utama untuk mengatur tata letak widget, yaitu \texttt{pack()}, \texttt{grid()}, dan \texttt{place()}. Ketiga metode ini berfungsi untuk menentukan bagaimana widget ditampilkan di dalam jendela atau frame. Metode \texttt{pack()} menyusun widget secara berurutan secara vertikal atau horizontal, \texttt{grid()} menyusun widget dalam bentuk baris dan kolom seperti tabel, sementara \texttt{place()} memberikan kontrol penuh untuk menempatkan widget berdasarkan koordinat tertentu. Memahami karakteristik dari setiap metode akan membantu pengembang memilih cara yang paling sesuai untuk merancang tampilan aplikasi.

\subsection{Menggunakan \texttt{pack()}}

Metode \texttt{pack()} adalah metode layout yang paling sederhana. Widget akan ditempatkan secara berurutan di area jendela sesuai posisi yang ditentukan melalui parameter seperti \texttt{side}, \texttt{fill}, dan \texttt{expand}. Parameter \texttt{fill} digunakan agar widget dapat melebar secara horizontal atau vertikal, sedangkan \texttt{expand} memungkinkan widget memanfaatkan ruang kosong tambahan saat jendela diperbesar. Dengan pendekatan berbasis kelas, pengaturan layout ditempatkan di dalam konstruktor sehingga struktur aplikasi lebih terorganisasi.

\begin{lstlisting}[style=PythonStyle, caption={Contoh Layout pack() Menggunakan Class (1 Baris per pack)}]
import tkinter as tk

class PackDemo(tk.Tk):
    def __init__(self):
        super().__init__()
        self.title("Contoh pack() Lengkap")
        self.geometry("300x250")

        tk.Label(self, text="Atas", bg="lightblue").pack(side="top", fill="x", pady=5)
        tk.Button(self, text="Kiri", bg="yellow").pack(side="left", fill="y", padx=10, pady=10)
        tk.Button(self, text="Kanan", bg="cyan").pack(side="right", expand=True, padx=10, pady=10)
        tk.Label(self, text="Bawah", bg="lightgreen").pack(side="bottom", pady=5)

if __name__ == "__main__":
    app = PackDemo()
    app.mainloop()
\end{lstlisting}



\subsection{Menggunakan \texttt{grid()}}

Metode \texttt{grid()} memungkinkan widget ditempatkan dalam baris dan kolom. Pendekatan ini sangat ideal untuk membuat form karena tata letaknya teratur dan mudah dibaca. Setiap widget ditempatkan menggunakan parameter \texttt{row} dan \texttt{column}. Untuk memperlebar widget ke beberapa kolom, dapat digunakan \texttt{columnspan}, sedangkan \texttt{sticky} digunakan agar widget “menempel” ke arah tertentu seperti utara (N), selatan (S), timur (E), dan barat (W). Dengan menggunakan pendekatan kelas, form yang dibangun dengan \texttt{grid()} menjadi lebih mudah dikembangkan dan dipelihara.

\begin{lstlisting}[style=PythonStyle, caption={Contoh Layout dengan grid() Menggunakan Class}]
import tkinter as tk

class GridDemo(tk.Tk):
    def __init__(self):
        super().__init__()
        self.title("Contoh grid()")
        self.geometry("300x150")

        tk.Label(self, text="Nama:").grid(row=0, column=0, padx=5, pady=5)
        self.entry_nama = tk.Entry(self)
        self.entry_nama.grid(row=0, column=1, padx=5, pady=5)

        tk.Label(self, text="Umur:").grid(row=1, column=0, padx=5, pady=5)
        self.entry_umur = tk.Entry(self)
        self.entry_umur.grid(row=1, column=1, padx=5, pady=5)

        tk.Button(self, text="Simpan").grid(row=2, column=0, columnspan=2, pady=10)

if __name__ == "__main__":
    app = GridDemo()
    app.mainloop()
\end{lstlisting}


\subsection{Menggunakan \texttt{place()}}

Metode \texttt{place()} memungkinkan widget ditempatkan menggunakan koordinat absolut (\texttt{x}, \texttt{y}) maupun relatif (\texttt{relx}, \texttt{rely}). Sistem layout ini memberikan kontrol penuh terhadap posisi setiap widget, tetapi kurang cocok untuk aplikasi yang harus bersifat responsif atau menyesuaikan ukuran jendela. Pada pendekatan berbasis kelas, semua posisi absolut didefinisikan di dalam konstruktor sehingga lebih mudah dibaca dan dikelola.

\begin{lstlisting}[style=PythonStyle, caption={Contoh Layout dengan place() Menggunakan Class}]
import tkinter as tk

class PlaceDemo(tk.Tk):
    def __init__(self):
        super().__init__()
        self.title("Contoh place()")
        self.geometry("300x200")

        tk.Label(self, text="Label di x=20, y=30").place(x=20, y=30)
        tk.Button(self, text="Tombol").place(x=100, y=100)
        tk.Entry(self).place(x=150, y=150, width=120)

if __name__ == "__main__":
    app = PlaceDemo()
    app.mainloop()
\end{lstlisting}

\subsection{Layout dalam Frame}

Frame adalah wadah (container) di Tkinter yang digunakan untuk mengelompokkan widget dalam area tertentu. Dengan menggunakan frame, pengembang dapat membangun struktur layout yang lebih rapi dan terorganisasi. Setiap frame dapat menggunakan metode layout yang berbeda, misalnya \texttt{pack()} di frame luar dan \texttt{grid()} di frame dalam. Teknik ini disebut \textit{nested layout}, dan sangat berguna ketika aplikasi memiliki beberapa bagian seperti header, form input, dan area tombol.

Pada pendekatan berbasis kelas, frame didefinisikan di dalam konstruktor sehingga struktur antarmuka menjadi lebih modular dan mudah diperluas.

\begin{lstlisting}[style=PythonStyle, caption={Contoh Layout dalam Frame Menggunakan Class}]
import tkinter as tk

class FrameDemo(tk.Tk):
    def __init__(self):
        super().__init__()
        self.title("Layout dalam Frame")
        self.geometry("300x200")

        # Frame atas
        self.frame_atas = tk.Frame(self, bg="lightgray")
        self.frame_atas.pack(fill="x", pady=10)

        tk.Label(self.frame_atas, text="Ini bagian atas").pack(pady=5)

        # Frame bawah
        self.frame_bawah = tk.Frame(self)
        self.frame_bawah.pack(pady=10)

        tk.Button(self.frame_bawah, text="Tombol 1").pack(side="left", padx=5)
        tk.Button(self.frame_bawah, text="Tombol 2").pack(side="left", padx=5)

if __name__ == "__main__":
    app = FrameDemo()
    app.mainloop()
\end{lstlisting}

Pada contoh di atas, frame atas digunakan untuk menampilkan label, sementara frame bawah digunakan untuk menempatkan dua tombol secara berdampingan. Dengan memisahkan widget berdasarkan fungsinya ke dalam frame terpisah, tampilan aplikasi menjadi lebih terorganisasi dan kode lebih mudah dipahami.

\subsection{Contoh Kombinasi Layout Sederhana}

Dalam banyak aplikasi, kombinasi layout sering digunakan untuk menghasilkan tampilan yang lebih fleksibel. Misalnya, \texttt{pack()} dapat digunakan untuk menempatkan beberapa frame utama secara vertikal, sementara \texttt{grid()} digunakan di dalam masing-masing frame untuk mengatur label dan entry secara terstruktur. Pendekatan kombinasi ini memberikan keseimbangan antara fleksibilitas dan kemudahan dalam mengatur komponen.

Dengan memanfaatkan kombinasi layout, aplikasi Tkinter dapat dibuat lebih modular, di mana setiap bagian memiliki gaya layout yang sesuai dengan kebutuhannya. Contoh berikut menunjukkan penggunaan \texttt{pack()} untuk frame utama dan \texttt{grid()} untuk form input di dalam frame tersebut menggunakan pendekatan berbasis class.

\begin{lstlisting}[style=PythonStyle, caption={Kombinasi pack() dan grid() Menggunakan Class}]
import tkinter as tk

class KombinasiLayoutDemo(tk.Tk):
    def __init__(self):
        super().__init__()
        self.title("Kombinasi Layout")
        self.geometry("300x250")

        # Frame form (diletakkan dengan pack)
        self.frame_form = tk.Frame(self, padx=10, pady=10, relief="groove", borderwidth=2)
        self.frame_form.pack(pady=10)

        # Menggunakan grid di dalam frame form
        tk.Label(self.frame_form, text="Nama:").grid(row=0, column=0, padx=5, pady=5)
        tk.Entry(self.frame_form).grid(row=0, column=1, padx=5, pady=5)

        tk.Label(self.frame_form, text="Umur:").grid(row=1, column=0, padx=5, pady=5)
        tk.Entry(self.frame_form).grid(row=1, column=1, padx=5, pady=5)

        # Frame tombol bawah (pakai pack)
        self.frame_tombol = tk.Frame(self)
        self.frame_tombol.pack(pady=10)

        tk.Button(self.frame_tombol, text="Simpan").pack(side="left", padx=5)
        tk.Button(self.frame_tombol, text="Batal").pack(side="left", padx=5)

if __name__ == "__main__":
    app = KombinasiLayoutDemo()
    app.mainloop()
\end{lstlisting}

Pada contoh di atas, frame form ditempatkan menggunakan \texttt{pack()}, sementara widget di dalamnya diatur menggunakan \texttt{grid()} untuk menghasilkan tampilan form yang rapi. Frame tombol berada di bawah form dan menggunakan \texttt{pack()} untuk menempatkan tombol secara berdampingan. Pendekatan class membuat struktur antarmuka lebih terorganisasi, mudah diperluas, dan lebih cocok untuk aplikasi yang berkembang.




\section{Menampilkan Gambar Menggunakan File Dialog}

Tkinter menyediakan objek \texttt{PhotoImage} untuk menampilkan gambar pada aplikasi. Gambar dapat dimuat dari file dan dipasang pada widget \texttt{Label}. Untuk memungkinkan pengguna memilih gambar sendiri, Tkinter menyediakan fungsi \texttt{askopenfilename()} dari modul \texttt{filedialog}. Dengan menggabungkan satu tombol, satu label, dan file dialog, aplikasi sederhana dapat dibuat untuk menampilkan gambar yang dipilih pengguna.

Dalam penerapannya, tombol digunakan untuk membuka dialog pemilihan file. Jika pengguna memilih gambar, file tersebut dimuat sebagai objek \texttt{PhotoImage} lalu ditampilkan pada Label. Agar gambar muncul dengan benar, referensinya harus disimpan sehingga tidak terhapus oleh sistem memori Python. Contoh berikut menunjukkan implementasi dasar dari proses tersebut.

\begin{lstlisting}[style=PythonStyle, caption={Contoh Menampilkan Gambar dengan Class}]
import tkinter as tk
from tkinter import filedialog

class GambarDemo(tk.Tk):
    def __init__(self):
        super().__init__()
        self.title("Contoh Menampilkan Gambar")
        self.geometry("400x300")

        # Label tempat menampilkan gambar
        self.label_gambar = tk.Label(self)
        self.label_gambar.pack(pady=10)

        # Tombol untuk memilih gambar
        self.button = tk.Button(self, text="Pilih Gambar", command=self.pilih_gambar)
        self.button.pack(pady=10)

    def pilih_gambar(self):
        filepath = filedialog.askopenfilename(
            filetypes=[("Image Files", "*.png;*.gif")]
        )

        if filepath:
            img = tk.PhotoImage(file=filepath)
            self.label_gambar.img = img   # simpan referensi
            self.label_gambar.config(image=img)

if __name__ == "__main__":
    app = GambarDemo()
    app.mainloop()
\end{lstlisting}

Pada contoh di atas, tombol digunakan untuk membuka file dialog sehingga pengguna dapat memilih gambar. Jika gambar berhasil dipilih, objek \texttt{PhotoImage} dibuat dan ditampilkan pada Label. Penyimpanan referensi gambar melalui \texttt{self.label\_gambar.img = img} diperlukan agar gambar tidak hilang dari memori. Dengan cara ini, mahasiswa dapat memahami dasar menampilkan gambar menggunakan Tkinter.

\section{Menu}

Menu merupakan komponen antarmuka yang digunakan untuk memberikan daftar perintah dalam bentuk \textit{menu bar} di bagian atas jendela. Melalui menu, pengguna dapat memilih perintah tertentu seperti membuka form lain, menyimpan data, atau mengubah pengaturan aplikasi. Tkinter menyediakan kelas \texttt{Menu} yang memungkinkan pengembang membuat menu bar dan menambahkan item menu secara sederhana. Dengan menghubungkan item menu ke sebuah method, menu dapat berinteraksi langsung dengan komponen dalam aplikasi.

Untuk memahami konsep dasar menu, contoh berikut memperlihatkan sebuah item menu yang berfungsi mengubah judul jendela. Ketika pengguna memilih menu tersebut, sebuah method dipanggil dan judul jendela diperbarui. Contoh ini menunjukkan bagaimana menu dapat dihubungkan dengan logika program dengan cara yang mudah dan terorganisasi menggunakan pendekatan berbasis kelas.

\begin{lstlisting}[style=PythonStyle, caption={Contoh Menu untuk Mengubah Judul Window Menggunakan Class}]
import tkinter as tk

class MenuDemo(tk.Tk):
    def __init__(self):
        super().__init__()
        self.title("Judul Awal")
        self.geometry("300x150")

        # Membuat menu bar
        menu_bar = tk.Menu(self)

        # Membuat menu "File"
        menu_file = tk.Menu(menu_bar, tearoff=0)
        menu_file.add_command(label="Ubah Judul", command=self.ubah_judul)

        # Menambahkan menu ke menu bar
        menu_bar.add_cascade(label="File", menu=menu_file)

        # Menampilkan menu bar
        self.config(menu=menu_bar)

    # Method yang dipanggil dari menu
    def ubah_judul(self):
        self.title("Judul Berubah!")

if __name__ == "__main__":
    app = MenuDemo()
    app.mainloop()
\end{lstlisting}

Pada contoh di atas, menu bar dibuat di dalam konstruktor kelas \texttt{MenuDemo}. Menu bernama “File” ditambahkan ke menu bar, kemudian item menu “Ubah Judul” dihubungkan dengan method \texttt{ubah\_judul()}. Ketika item menu dipilih, judul jendela berubah dari “Judul Awal” menjadi “Judul Berubah!”. Contoh ini memperlihatkan bagaimana menu dapat berfungsi sebagai pemicu perintah dalam aplikasi Tkinter dengan cara yang sederhana dan terstruktur.


\section{Navigasi Antar Form}

Dalam aplikasi Tkinter, pengembang dapat membuat lebih dari satu form atau jendela agar fitur aplikasi dapat dipisahkan dengan rapi. Salah satu cara untuk membuka form lain adalah melalui menu, sehingga pengguna dapat melakukan navigasi dengan memilih item menu tertentu. Tkinter menyediakan kelas \texttt{Toplevel} untuk membuat jendela tambahan yang berdiri sendiri tetapi tetap terhubung dengan aplikasi utama.

Pada contoh ini, form utama akan memiliki menu bernama “Navigasi.” Ketika pengguna memilih “Buka Form Kedua,” sebuah form baru akan muncul. Form kedua dibuat pada file terpisah dan hanya merupakan jendela kosong tanpa komponen tambahan. Contoh ini menunjukkan struktur navigasi antar file secara sederhana.

\begin{lstlisting}[style=PythonStyle, caption={main_window.py: Form Utama dengan Menu Navigasi}]
import tkinter as tk
from second_form import SecondForm

class MainWindow(tk.Tk):
    def __init__(self):
        super().__init__()
        self.title("Form Utama")
        self.geometry("400x200")

        # Label sederhana
        self.label = tk.Label(self, text="Ini form utama.")
        self.label.pack(pady=20)

        # Menu bar
        menu_bar = tk.Menu(self)

        # Menu Navigasi
        menu_navigasi = tk.Menu(menu_bar, tearoff=0)
        menu_navigasi.add_command(label="Buka Form Kedua",
                                  command=self.buka_form_kedua)

        menu_bar.add_cascade(label="Navigasi", menu=menu_navigasi)

        # Tampilkan menu bar
        self.config(menu=menu_bar)

    def buka_form_kedua(self):
        # Membuka form kedua
        SecondForm(self)

if __name__ == "__main__":
    app = MainWindow()
    app.mainloop()
\end{lstlisting}

\begin{lstlisting}[style=PythonStyle, caption={second_form.py: Form Kedua Kosong}]
import tkinter as tk

class SecondForm(tk.Toplevel):
    def __init__(self, master):
        super().__init__(master)
        self.title("Form Kedua")
        self.geometry("300x150")
        # Tidak ada komponen tambahan (form kosong)
\end{lstlisting}

Dalam contoh ini, form kedua dibuat sebagai kelas \texttt{SecondForm} yang mewarisi \texttt{tk.Toplevel}. Ketika pengguna memilih item menu “Buka Form Kedua,” jendela baru akan muncul sebagai form kosong. Pendekatan ini membantu mahasiswa memahami struktur dasar navigasi antar form tanpa menambahkan kompleksitas komponen tambahan.

\section{Tabel Data (Table)}

Pada Tkinter, tampilan data dalam bentuk tabel dapat dibuat menggunakan widget \texttt{ttk.Treeview} dari modul \texttt{tkinter.ttk}. Meskipun Tkinter tidak memiliki widget tabel khusus seperti di spreadsheet, \texttt{Treeview} dapat berfungsi sebagai tabel sederhana dengan beberapa kolom dan banyak baris. Widget ini sering digunakan untuk menampilkan data yang terstruktur, misalnya daftar mahasiswa, data barang, atau hasil perhitungan. Setiap baris pada \texttt{Treeview} disebut \textit{item}, sedangkan kolom-kolomnya dapat diberi nama dan lebar tertentu sesuai kebutuhan.

Untuk membuat tabel, pengembang perlu mendefinisikan nama kolom, judul header, serta menambahkan baris data satu per satu. Selain itu, biasanya ditambahkan scrollbar agar pengguna dapat menggulir tabel ketika jumlah datanya banyak. Pengguna juga dapat memilih salah satu baris, kemudian data dari baris terpilih tersebut dapat diambil dan ditampilkan di Label atau diproses lebih lanjut. Dengan kombinasi sederhana antara \texttt{Treeview}, \texttt{Scrollbar}, dan \texttt{Button}, mahasiswa sudah dapat membangun tampilan tabel dasar yang cukup untuk banyak kasus aplikasi pemula.

\begin{lstlisting}[style=PythonStyle, caption={Contoh Tabel Data dengan ttk.Treeview}]
import tkinter as tk
from tkinter import ttk

class TableDemo(tk.Tk):
    def __init__(self):
        super().__init__()
        self.title("Contoh Tabel Data")
        self.geometry("400x250")

        # Frame untuk menampung tabel dan scrollbar
        frame_table = tk.Frame(self)
        frame_table.pack(fill="both", expand=True, pady=10, padx=10)

        # Membuat Treeview dengan dua kolom
        self.tree = ttk.Treeview(
            frame_table,
            columns=("nama", "umur"),
            show="headings"  # hanya menampilkan header, tanpa kolom tree
        )

        # Mengatur header kolom
        self.tree.heading("nama", text="Nama")
        self.tree.heading("umur", text="Umur")

        # Mengatur lebar kolom
        self.tree.column("nama", width=200)
        self.tree.column("umur", width=80, anchor="center")

        # Menambahkan beberapa baris data
        self.tree.insert("", tk.END, values=("Andi", 20))
        self.tree.insert("", tk.END, values=("Budi", 21))
        self.tree.insert("", tk.END, values=("Citra", 19))

        # Scrollbar vertikal
        scrollbar = tk.Scrollbar(frame_table, orient="vertical", command=self.tree.yview)
        self.tree.configure(yscrollcommand=scrollbar.set)

        # Menempatkan tabel dan scrollbar berdampingan
        self.tree.pack(side="left", fill="both", expand=True)
        scrollbar.pack(side="right", fill="y")

        # Label untuk menampilkan baris terpilih
        self.label_info = tk.Label(self, text="Belum ada baris yang dipilih")
        self.label_info.pack(pady=5)

        # Tombol untuk mengambil data baris terpilih
        button_pilih = tk.Button(self, text="Tampilkan Baris Terpilih",
                                 command=self.tampilkan_pilihan)
        button_pilih.pack(pady=5)

    def tampilkan_pilihan(self):
        # Mengambil item yang sedang dipilih
        selected = self.tree.selection()
        if selected:
            values = self.tree.item(selected[0], "values")
            self.label_info.config(text=f"Terpilih: Nama = {values[0]}, Umur = {values[1]}")
        else:
            self.label_info.config(text="Belum ada baris yang dipilih")

if __name__ == "__main__":
    app = TableDemo()
    app.mainloop()
\end{lstlisting}

Pada contoh di atas, \texttt{ttk.Treeview} digunakan sebagai tabel dengan dua kolom: \texttt{Nama} dan \texttt{Umur}. Beberapa baris data dimasukkan menggunakan \texttt{insert()}, kemudian sebuah scrollbar vertikal ditambahkan agar tabel tetap nyaman dilihat meskipun data bertambah banyak. Pengguna dapat memilih salah satu baris, lalu menekan tombol “Tampilkan Baris Terpilih” untuk menampilkan data baris tersebut pada Label di bawah tabel. Contoh ini memberikan gambaran dasar bagaimana membuat dan menggunakan tabel data di Tkinter tanpa konfigurasi yang terlalu kompleks.

\section{Struktur Program Tkinter}

\subsection{Struktur Dasar File}
Struktur program Tkinter yang baik biasanya memisahkan logika antarmuka, fungsi-fungsi pendukung, dan form tambahan ke dalam file yang berbeda. Pada program yang sederhana, satu file sudah mencukupi, tetapi ketika aplikasi mulai berkembang, pemisahan file akan membantu menjaga kode tetap teratur dan mudah dibaca. Biasanya, file utama berisi jendela utama aplikasi, sedangkan form tambahan ditempatkan dalam file lain menggunakan kelas \texttt{Toplevel}. Struktur file yang rapi juga memudahkan proses debugging dan pengembangan fitur baru.

\subsection{Pemanggilan Fungsi dan Modul}
Aplikasi Tkinter dapat menggunakan berbagai modul Python untuk mendukung operasinya. Modul standar seperti \texttt{tkinter}, \texttt{tkinter.ttk}, dan \texttt{filedialog} biasanya diimpor pada bagian atas file. Jika aplikasi memiliki beberapa form, file tambahan dapat diimpor menggunakan sintaks \texttt{from nama\_file import NamaClass}. Pola ini memungkinkan form utama memanggil dan membuka form lainnya secara modular. Selain itu, fungsi-fungsi kecil yang mendukung logika aplikasi dapat ditulis sebagai metode di dalam kelas, sehingga keterkaitan antara tombol, menu, dan fungsi menjadi lebih jelas.

\subsection{Best Practice Organisasi Kode}
Untuk menjaga kualitas kode, beberapa praktik umum dapat diterapkan. Pertama, gunakan pendekatan berbasis kelas agar struktur program lebih mudah diatur dan diperluas. Kedua, pisahkan komponen antarmuka dan logika ke dalam metode yang terorganisasi sehingga program tidak menjadi satu blok besar yang sulit dibaca. Ketiga, gunakan nama variabel dan metode yang jelas serta deskriptif untuk memudahkan pemahaman. Terakhir, pertimbangkan untuk memisahkan form atau komponen kompleks ke dalam file terpisah guna mendukung pengembangan aplikasi yang lebih besar. Dengan mengikuti prinsip-prinsip ini, mahasiswa dapat membangun aplikasi Tkinter yang lebih profesional dan mudah dipelihara.


\section{Daftar Widgets}

\begin{table}[h]
\centering
\begin{tabular}{|l|l|}
\hline
\textbf{Widget} & \textbf{Fungsi} \\ \hline
Label & Menampilkan teks atau gambar \\ \hline
Button & Tombol untuk menjalankan aksi \\ \hline
Entry & Input teks satu baris \\ \hline
Text & Input teks multiline \\ \hline
Frame & Kontainer widget \\ \hline
LabelFrame & Frame dengan judul \\ \hline
Checkbutton & Checkbox (true/false) \\ \hline
Radiobutton & Pilihan tunggal dalam grup \\ \hline
Listbox & Daftar item \\ \hline
Scrollbar & Scroll vertikal/horizontal \\ \hline
Canvas & Menggambar bentuk, gambar, animasi \\ \hline
Scale & Slider untuk memilih nilai \\ \hline
Spinbox & Input angka naik/turun \\ \hline
Menu & Menu bar aplikasi \\ \hline
Menubutton & Tombol dengan menu drop-down \\ \hline
Message & Menampilkan teks panjang dengan wrapping \\ \hline
Toplevel & Membuat window baru \\ \hline
PanedWindow & Panel dapat di-resize \\ \hline
OptionMenu & Dropdown sederhana \\ \hline
PhotoImage & Objek gambar PNG/GIF \\ \hline
BitmapImage & Objek bitmap hitam/putih \\ \hline
\end{tabular}
\caption{Widget Dasar Tkinter}
\end{table}

\begin{table}[h]
\centering
\begin{tabular}{|l|l|}
\hline
\textbf{Widget ttk} & \textbf{Fungsi} \\ \hline
ttk.Label & Label bergaya modern \\ \hline
ttk.Button & Tombol modern \\ \hline
ttk.Entry & Input teks modern \\ \hline
ttk.Frame & Kontainer modern \\ \hline
ttk.Checkbutton & Checkbox modern \\ \hline
ttk.Radiobutton & Radio modern \\ \hline
ttk.Combobox & Dropdown yang bisa diketik \\ \hline
ttk.Spinbox & Spinbox modern \\ \hline
ttk.Progressbar & Progress bar \\ \hline
ttk.Separator & Garis pemisah \\ \hline
ttk.Sizegrip & Pegangan resize window \\ \hline
ttk.Treeview & Tabel / daftar hierarchical \\ \hline
ttk.Notebook & Tab (tabbed interface) \\ \hline
ttk.PanedWindow & Panel modern dapat di-resize \\ \hline
ttk.Scrollbar & Scroll modern \\ \hline
ttk.LabelFrame & Frame dengan judul modern \\ \hline
ttk.Menubutton & Menu button modern \\ \hline
\end{tabular}
\caption{Widget Modern Tkinter (ttk)}
\end{table}

\begin{table}[h]
\centering
\begin{tabular}{|l|l|}
\hline
\textbf{Dialog} & \textbf{Fungsi} \\ \hline
askopenfilename() & Memilih file \\ \hline
asksaveasfilename() & Menyimpan file \\ \hline
askdirectory() & Memilih folder \\ \hline
colorchooser.askcolor() & Memilih warna \\ \hline
messagebox.showinfo() & Pesan informasi \\ \hline
messagebox.showwarning() & Peringatan \\ \hline
messagebox.showerror() & Pesan error \\ \hline
messagebox.askyesno() & Dialog ya/tidak \\ \hline
messagebox.askokcancel() & OK/Cancel \\ \hline
\end{tabular}
\caption{Dialog Tkinter}
\end{table}

