\documentclass[aspectratio=169, table]{beamer}
\usepackage[utf8]{inputenc}
\usepackage{listings} 
\usepackage[strings]{underscore}
\usepackage{caption}
\usepackage{float}


\renewcommand{\lstlistingname}{} 

\makeatletter
\def\input@path{{../../themes/Pradita}}
\makeatother

\usetheme{Pradita}

\subtitle{IF120203-Programming Fundamentals}

\title{Chapter-07:\\\LARGE{File I/O pada Python\\}
\vspace{10pt}}
\date[Serial]{\scriptsize {PRU/SPMI/FR-BM-18/0222}}
\author[Pradita]{\small{\textbf{Alfa Yohannis}}}


% Define Python language style for listings
\lstdefinestyle{PythonStyle}{
    language=Python,
    basicstyle=\ttfamily\footnotesize,
    keywordstyle=\color{blue}\bfseries,
    commentstyle=\color{gray}\itshape,
    stringstyle=\color{red},
    showstringspaces=false,
    breaklines=true,
    frame=lines,
    numbers=left,
    numberstyle=\tiny\color{gray},
    backgroundcolor=\color{lightgray!10},
    tabsize=2,
    captionpos=b
}

\lstdefinelanguage{bash} {
	keywords={},
	basicstyle=\ttfamily\small,
	keywordstyle=\color{blue}\bfseries,
	ndkeywords={iex},
	ndkeywordstyle=\color{purple}\bfseries,
	sensitive=true,
	commentstyle=\color{gray},
	stringstyle=\color{red},
	numbers=left,
	numberstyle=\tiny\color{gray},
	breaklines=true,
	frame=lines,
	backgroundcolor=\color{lightgray!10},
	tabsize=2,
	comment=[l]{\#},
	morecomment=[s]{/*}{*/},
	commentstyle=\color{gray}\ttfamily,
	stringstyle=\color{purple}\ttfamily,
	showstringspaces=false,
	captionpos=b
}

\begin{document}

\frame{\titlepage}

% Add table of contents slide
\begin{frame}[fragile]{Contents}
\vspace{15pt}
\begin{columns}[t]
\begin{column}{.4\textwidth}
\tableofcontents[sections={1-4}]
\end{column}
\begin{column}{.6\textwidth}
\tableofcontents[sections={5-7}]
\end{column}
\end{columns}
\end{frame}


\section{Pendahuluan}
\begin{frame}{Pendahuluan}
\vspace{20pt}
\begin{itemize}
  \item File I/O: membaca/menulis data ke file untuk penyimpanan dan pertukaran informasi.
  \item Python memudahkan dengan \texttt{open()} dan \texttt{with} untuk pengelolaan resource yang aman.
  \item Jenis umum: file teks (berbasis karakter) dan CSV (tabel sederhana dipisah pemisah).
  \item Manfaat: pengolahan data, laporan otomatis, log, arsip hasil perhitungan.
  \item Alur dasar: buka \textrightarrow{} baca/proses \textrightarrow{} tulis.
  \item Tujuan: memahami konsep dan siap menerapkan pada contoh kode berikutnya.
\end{itemize}
\end{frame}

\section{Membaca dan Menulis File Teks}

\begin{frame}{Membaca dan Menulis File Teks}
\vspace{20pt}
File teks menyimpan data berbasis karakter seperti catatan atau log.  
Python menggunakan fungsi \texttt{open()} untuk mengakses file dengan mode tertentu:  
\texttt{'r'} (baca), \texttt{'w'} (tulis), \texttt{'a'} (tambah), dan \texttt{'x'} (buat baru).  
Gunakan \texttt{with open(...)} agar file tertutup otomatis.  
Metode umum: \texttt{read()}, \texttt{readline()}, \texttt{readlines()} untuk membaca,  
serta \texttt{write()} dan \texttt{writelines()} untuk menulis.  
Pastikan encoding menggunakan UTF-8 agar karakter terbaca dengan benar.
\end{frame}

\begin{frame}[fragile]{Contoh Kode: File Teks di Python}
\vspace{20pt}
\begin{lstlisting}[style=PythonStyle]
# Menulis file teks
with open("data.txt", "w", encoding="utf-8") as f:
    f.write("Baris pertama\n")
    f.write("Baris kedua\n")

# Membaca kembali file
with open("data.txt", "r", encoding="utf-8") as f:
    for baris in f:
        print(baris.strip())
\end{lstlisting}
\end{frame}

\section{Penyaringan dan Penghitungan Data dari File}

\begin{frame}{Penyaringan dan Penghitungan Data}
\vspace{20pt}
Pemrosesan file teks memungkinkan kita mengekstrak informasi penting dari data mentah.  
Dua teknik umum yang sering digunakan adalah \textbf{penyaringan} (filtering) dan \textbf{penghitungan} (counting).  
Program membaca file baris demi baris dan memeriksa apakah baris tersebut memenuhi kondisi tertentu.  
Python mempermudah hal ini dengan perulangan \texttt{for line in file:} yang efisien dan hemat memori.  
Pendekatan ini cocok untuk memproses file besar tanpa perlu memuat seluruh isi ke dalam memori.
\end{frame}

\begin{frame}[fragile]{Isi File \texttt{log\_aktivitas.txt}}
\vspace{20pt}
\begin{lstlisting}[language=bash, caption={Cuplikan isi file log_aktivitas.txt}]
[08:45] Starting system check...
[09:00] User login successful
[09:15] Launching Python script for data processing
[09:30] ERROR: Missing configuration file
[10:00] Process completed successfully
[10:30] Running Python script for report generation
[11:00] ERROR: Disk read failure
[11:30] User logged out
\end{lstlisting}
\end{frame}

\begin{frame}[fragile]{Contoh 1: Penyaringan Baris}
\vspace{20pt}
\begin{lstlisting}[style=PythonStyle, caption={Menyaring baris yang mengandung kata "Python"}]
with open("log_aktivitas.txt", "r", encoding="utf-8") as file:
    for baris in file:
        if "Python" in baris:
            print(baris.strip())
\end{lstlisting}

\noindent\textbf{Output:}
\begin{lstlisting}[language=bash, caption={Hasil penyaringan baris yang mengandung "Python"}]
[09:15] Launching Python script for data processing
[10:30] Running Python script for report generation
\end{lstlisting}
\end{frame}

\begin{frame}[fragile]{Contoh 2: Penghitungan Kemunculan Kata}
\vspace{20pt}
\begin{lstlisting}[style=PythonStyle, caption={Menghitung kemunculan kata "error"}]
jumlah_error = 0
with open("log_aktivitas.txt", "r", encoding="utf-8") as file:
    for baris in file:
        kata_kata = baris.lower().split()
        for kata in kata_kata:
            if kata.startswith("error"):
                jumlah_error += 1

print(f"Kata 'error' muncul {jumlah_error} kali.")
\end{lstlisting}

\noindent\textbf{Output:}
\begin{lstlisting}[language=bash, caption={Hasil penghitungan kemunculan kata "error"}]
Kata 'error' muncul 2 kali.
\end{lstlisting}
\end{frame}


\section{Membaca dan Menulis File CSV}

\begin{frame}[fragile]{Membaca dan Menulis File CSV}
\vspace{20pt}
\begin{columns}[T, totalwidth=\textwidth]
  %=== Kolom Kiri ===%
  \begin{column}{0.65\textwidth}
  \begin{itemize}
    \item File CSV (\textit{Comma-Separated Values}) menyimpan data dalam bentuk tabel sederhana.
    \item Setiap baris mewakili satu entri, dan setiap kolom dipisahkan oleh koma atau pemisah lain.
    \item Format umum untuk pertukaran data antar aplikasi seperti Excel atau Google Sheets.
    \item Python menyediakan modul bawaan \texttt{csv} untuk membaca dan menulis data secara efisien.
    \item Baris pertama (\textbf{header}) berisi nama kolom, diikuti baris-baris nilai aktual.
    \item Gunakan \texttt{csv.DictReader} dan \texttt{csv.DictWriter} untuk akses kolom berdasarkan nama.
  \end{itemize}
  \end{column}

  %=== Kolom Kanan ===%
  \begin{column}{0.3\textwidth}
  \vspace{5pt}
  \textbf{Contoh isi file \texttt{nilai\_mahasiswa.csv}:}
  \begin{lstlisting}[language=bash, caption={Cuplikan isi file CSV}]
Nama,Nilai
Andi,85.5
Budi,90.0
Citra,78.0
Dewi,88.5
  \end{lstlisting}
  \end{column}
\end{columns}
\end{frame}



\begin{frame}{Struktur dan Pendekatan}
\vspace{20pt}
File CSV umumnya terdiri dari:
\begin{itemize}
  \item Baris \textbf{header}: nama kolom (mis. \texttt{Nama}, \texttt{Nilai})
  \item Baris \textbf{data}: nilai aktual untuk setiap kolom
\end{itemize}
Pendekatan utama:
\begin{itemize}
  \item \texttt{csv.reader} / \texttt{csv.writer} → bekerja berbasis \textit{list}
  \item \texttt{csv.DictReader} / \texttt{csv.DictWriter} → berbasis \textit{dictionary}
\end{itemize}
Pendekatan kedua lebih disarankan karena aman terhadap perubahan urutan kolom.  
Pastikan juga konversi tipe data numerik menggunakan \texttt{int()} atau \texttt{float()} saat membaca file.
\end{frame}

\begin{frame}[fragile]{Contoh 1: Menulis File CSV}
\vspace{20pt}
\begin{columns}[T, totalwidth=\textwidth]
  %=== Kolom Kiri ===%
  \begin{column}{0.65\textwidth}
  \begin{lstlisting}[style=PythonStyle]
import csv

data = [
    {"Nama": "Andi", "Nilai": 85.5},
    {"Nama": "Budi", "Nilai": 90.0},
    {"Nama": "Citra", "Nilai": 78.0},
    {"Nama": "Dewi", "Nilai": 88.5}
]

with open("nilai_mahasiswa.csv", "w", newline="", encoding="utf-8") as f:
    kolom = ["Nama", "Nilai"]
    writer = csv.DictWriter(f, fieldnames=kolom)
    writer.writeheader()
    writer.writerows(data)
  \end{lstlisting}
  \end{column}

  %=== Kolom Kanan ===%
  \begin{column}{0.3\textwidth}
  \textbf{Hasil File:}
  \begin{lstlisting}[language=bash, caption={Isi file nilai_mahasiswa.csv}]
Nama,Nilai
Andi,85.5
Budi,90.0
Citra,78.0
Dewi,88.5
  \end{lstlisting}
  \end{column}
\end{columns}
\end{frame}


\begin{frame}[fragile]{Contoh 2: Membaca File CSV}
\vspace{20pt}
\begin{lstlisting}[style=PythonStyle, caption={Membaca data CSV menggunakan DictReader}]
import csv

with open("nilai_mahasiswa.csv", "r", encoding="utf-8") as f:
    pembaca = csv.DictReader(f)
    for baris in pembaca:
        nama = baris["Nama"]
        nilai = float(baris["Nilai"])
        print(f"{nama} memperoleh nilai {nilai:.2f}")
\end{lstlisting}
\end{frame}

\begin{frame}[fragile]{Output Terminal}
\vspace{20pt}
\begin{lstlisting}[language=bash, caption={Hasil pembacaan file CSV di terminal}]
Andi memperoleh nilai 85.50
Budi memperoleh nilai 90.00
Citra memperoleh nilai 78.00
Dewi memperoleh nilai 88.50
\end{lstlisting}

Kedua contoh memperlihatkan proses penuh pengolahan file CSV —  
menulis data menggunakan \texttt{DictWriter} dan membacanya kembali dengan \texttt{DictReader}.  
Teknik ini penting untuk memproses data tabel sederhana dan menjadi dasar analisis data yang lebih kompleks.
\end{frame}

\section{Agregasi dan Ringkasan Data}

%=== FRAME 1: Konsep Agregasi dan Pengelompokan ===%
\begin{frame}{Agregasi dan Ringkasan Data}
\vspace{20pt}
\begin{itemize}
  \item \textbf{Agregasi} berarti merangkum sejumlah data menjadi informasi yang lebih ringkas dan bermakna.
  \item Contoh: menghitung total, rata-rata, jumlah kemunculan, nilai maksimum, atau minimum.
  \item Python memungkinkan agregasi sederhana menggunakan \texttt{for loop}, \texttt{list}, dan \texttt{dict}.
  \item \textbf{Pengelompokan} dapat dilakukan manual berdasarkan kategori, misalnya mata kuliah.
  \item Hasil agregasi dapat ditampilkan dalam teks terformat atau disimpan ke file laporan otomatis.
\end{itemize}
\end{frame}

%=== FRAME 2: Kode Input (Membaca dan Menghitung) ===%
\begin{frame}[fragile]{Input: Menghitung Rata-Rata Nilai}
\vspace{20pt}
\begin{lstlisting}[style=PythonStyle, caption={Menghitung rata-rata dari file CSV}]
import csv

total_nilai = 0
jumlah_data = 0

with open("nilai_mahasiswa.csv", "r", encoding="utf-8") as file:
    pembaca = csv.DictReader(file)
    for baris in pembaca:
        total_nilai += float(baris["Nilai"])
        jumlah_data += 1

rata_rata = total_nilai / jumlah_data if jumlah_data > 0 else 0
print(f"Rata-rata nilai mahasiswa: {rata_rata:.2f}")
\end{lstlisting}
\end{frame}

%=== FRAME 3: Kode Output (Menulis Laporan) ===%
\begin{frame}[fragile]{Output: Menulis File Laporan}
\vspace{20pt}
\begin{lstlisting}[style=PythonStyle, caption={Menyimpan hasil agregasi ke file laporan}]
with open("laporan.txt", "w", encoding="utf-8") as file:
    file.write("Laporan Ringkasan Nilai Mahasiswa\n")
    file.write("===============================\n")
    file.write(f"Jumlah data  : {jumlah_data}\n")
    file.write(f"Total nilai  : {total_nilai:.2f}\n")
    file.write(f"Rata-rata    : {rata_rata:.2f}\n")

print("Laporan disimpan ke laporan.txt.")
\end{lstlisting}
\end{frame}

%=== FRAME 4: Hasil Akhir (Isi File Laporan) ===%
\begin{frame}[fragile]{Hasil: Isi File \texttt{laporan.txt}}
\vspace{20pt}
\begin{lstlisting}[language=bash, caption={Isi file hasil ringkasan data}]
Laporan Ringkasan Nilai Mahasiswa
=================================
Jumlah data  : 4
Total nilai  : 342.00
Rata-rata    : 85.50
\end{lstlisting}

File laporan menampilkan jumlah data, total nilai, dan rata-rata dalam format teks terstruktur.  
Hasil ini menunjukkan penerapan konsep agregasi sederhana dan pelaporan otomatis dalam Python.
\end{frame}

\section{Penanganan Error dan Praktik Terbaik}

%=== FRAME 1: Konsep dan Prinsip ===%
\begin{frame}{Penanganan Error dan Praktik Terbaik}
\vspace{20pt}
\begin{itemize}
  \item Kesalahan (\textit{error}) saat bekerja dengan file tidak dapat dihindari, misalnya file hilang, rusak, atau tidak memiliki izin akses.
  \item Jenis error umum: 
  \begin{itemize}
    \item \texttt{FileNotFoundError} – file tidak ditemukan.  
    \item \texttt{IOError} – kesalahan input/output.  
    \item \texttt{ValueError} – format atau tipe data tidak sesuai.
  \end{itemize}
  \item Gunakan \texttt{try--except--finally} untuk mencegah program berhenti tiba-tiba.
  \item Gunakan \texttt{with open(...)} untuk penutupan file otomatis dan aman.
  \item Praktik terbaik:
  \begin{itemize}
    \item Gunakan encoding konsisten (mis. UTF-8).  
    \item Baca file besar secara bertahap (\textit{streaming}).  
    \item Gunakan struktur folder dan nama file yang jelas.
  \end{itemize}
\end{itemize}
\end{frame}

%=== FRAME 2: Contoh 1 - try-except-finally ===%
\begin{frame}[fragile]{Contoh 1: Penanganan Error Dasar}
\vspace{20pt}
\begin{lstlisting}[style=PythonStyle]
try:
    file = open("data_tidak_ada.txt", "r", encoding="utf-8")
    isi = file.read()
    print(isi)
except FileNotFoundError:
    print("Error: File tidak ditemukan.")
except IOError:
    print("Error: Terjadi kesalahan I/O saat membaca file.")
finally:
    try:
        file.close()
    except NameError:
        pass  # file belum dibuka, tidak perlu ditutup
\end{lstlisting}

\noindent
Pendekatan ini memastikan file ditutup meskipun terjadi error, 
dan pesan kesalahan ditampilkan dengan ramah bagi pengguna.
\end{frame}

%=== FRAME 3: Contoh 2 - with open() ===%
\begin{frame}[fragile]{Contoh 2: Praktik Terbaik dengan \texttt{with}}
\vspace{20pt}
\begin{lstlisting}[style=PythonStyle]
try:
    with open("data_mahasiswa.txt", "r", encoding="utf-8") as file:
        for baris in file:
            print(baris.strip())
except FileNotFoundError:
    print("File data_mahasiswa.txt tidak ditemukan.")
except UnicodeDecodeError:
    print("Encoding file tidak sesuai. Gunakan UTF-8.")
\end{lstlisting}

\noindent
Penggunaan \texttt{with} menjamin file selalu ditutup otomatis bahkan saat error terjadi, 
sehingga kode lebih aman, ringkas, dan bebas kebocoran resource.
\end{frame}


\section{Rangkuman}

\begin{frame}{Rangkuman}
\vspace{20pt}
\begin{itemize}
  \item Bab ini membahas konsep dasar pengelolaan file di Python, mencakup file teks dan file CSV.
  \item Mahasiswa mempelajari penggunaan fungsi \texttt{open()} dan context manager \texttt{with} untuk operasi file yang aman.
  \item Telah dijelaskan teknik membaca, menulis, menyaring, dan menghitung pola dalam file teks.
  \item Mahasiswa juga diperkenalkan pada modul \texttt{csv} untuk membaca dan menulis data terstruktur dalam format tabel.
  \item Latihan praktikum mencakup perhitungan total dan rata-rata serta pembuatan laporan otomatis.
  \item Setelah bab ini, mahasiswa diharapkan mampu mengelola file eksternal secara efisien dan menerapkan konsep File I/O dalam berbagai aplikasi pemrograman.
\end{itemize}
\end{frame}

\end{document}