\documentclass[aspectratio=169, table]{beamer}
\usepackage[utf8]{inputenc}
\usepackage{listings} 
\usepackage[strings]{underscore}
\usepackage{caption}
\usepackage{float}


\renewcommand{\lstlistingname}{} 

\makeatletter
\def\input@path{{../../themes/Pradita}}
\makeatother

\usetheme{Pradita}

\subtitle{IF120203-Programming Fundamentals}

\title{Chatper-05:\\\LARGE{Perulangan (Looping)\\}
\vspace{10pt}}
\date[Serial]{\scriptsize {PRU/SPMI/FR-BM-18/0222}}
\author[Pradita]{\small{\textbf{Alfa Yohannis}}}


% Define Python language style for listings
\lstdefinestyle{PythonStyle}{
language=Python,
basicstyle=\ttfamily\footnotesize,
keywordstyle=\color{blue},
commentstyle=\color{gray},
stringstyle=\color{red},
breaklines=true,
showstringspaces=false,
tabsize=2,
captionpos=b,
numbers=left,
numberstyle=\tiny\color{gray},
comment=[l]{//},
morecomment=[s]{/*}{*/},
commentstyle=\color{gray}\ttfamily,
string=[s]{'}{'},
morestring=[s]{"}{"},
}

\begin{document}

\frame{\titlepage}

% Add table of contents slide
\begin{frame}[fragile]{Contents}
\vspace{15pt}
\begin{columns}[t]
\begin{column}{.5\textwidth}
\tableofcontents[sections={1-3}]
\end{column}
\begin{column}{.5\textwidth}
\tableofcontents[sections={4-5}]
\end{column}
\end{columns}
\end{frame}

\section{Definisi Perulangan}
\begin{frame}{Definisi Perulangan}
Dalam pemrograman, seringkali kita perlu menjalankan blok kode yang sama berulang kali. 
Misalnya, mencetak angka dari 1 sampai 100, membaca setiap baris dari sebuah file, atau 
memproses setiap elemen dalam sebuah daftar data. 
Proses pengulangan eksekusi blok kode ini dikenal sebagai \textbf{perulangan} atau \textit{looping} (iterasi).
\end{frame}

\section{Perulangan di Python}
\begin{frame}{Perulangan di Python}
Python menyediakan dua mekanisme utama untuk melakukan perulangan:
\begin{enumerate}
    \item \textbf{Perulangan \texttt{for}}: Digunakan untuk melakukan iterasi pada sebuah urutan (seperti \texttt{list}, \texttt{tuple}, \texttt{string}) atau objek \textit{iterable} lainnya. Perulangan ini sering disebut sebagai \textit{definite iteration} karena jumlah pengulangannya sudah ditentukan oleh panjang urutan.
    \item \textbf{Perulangan \texttt{while}}: Digunakan untuk mengulang blok kode selama sebuah kondisi bernilai \texttt{True}. Perulangan ini disebut \textit{indefinite iteration} karena jumlah pengulangannya tidak pasti dan bergantung pada kapan kondisi menjadi \texttt{False}.
\end{enumerate}

Menguasai perulangan adalah langkah fundamental untuk menulis program yang efisien dan otomatis.
\end{frame}

\subsection{For Loop}
\begin{frame}[fragile]{For Loop (Sintaks Dasar)}
Perulangan \texttt{for} di Python bekerja dengan cara mengambil setiap elemen dari sebuah urutan secara bergantian.

Sintaks umum dari perulangan \texttt{for} adalah sebagai berikut:
\begin{lstlisting}[style=PythonStyle, caption={Sintaks Dasar Perulangan for}]
for nama_variabel in urutan:
    # Blok kode yang akan diulang
    # ...
\end{lstlisting}
\end{frame}

\begin{frame}[fragile]{For Loop (Iterasi Menggunakan \texttt{range()})}
Fungsi \texttt{range()} sangat umum digunakan bersama \texttt{for} untuk menghasilkan urutan angka.
\begin{itemize}
    \item \texttt{range(stop)}: Membuat urutan dari 0 hingga \texttt{stop-1}.
    \item \texttt{range(start, stop)}: Membuat urutan dari \texttt{start} hingga \texttt{stop-1}.
    \item \texttt{range(start, stop, step)}: Membuat urutan dari \texttt{start} hingga \texttt{stop-1} dengan lompatan sebesar \texttt{step}.
\end{itemize}
\end{frame}

\begin{frame}[fragile]{For Loop (Contoh Penggunaan)}
\begin{lstlisting}[style=PythonStyle, caption={Kode Python: for_with_range.py}]
# Mencetak angka dari 0 sampai 4
print("Contoh 1: range(5)")
for i in range(5):
    print(f"Perulangan ke-{i}")

# Mencetak angka dari 2 sampai 5
print("\nContoh 2: range(2, 6)")
for j in range(2, 6):
    print(f"Angka: {j}")
\end{lstlisting}
\end{frame}

\begin{frame}[fragile]{For Loop (Iterasi pada List dan String)}
Anda bisa melakukan iterasi secara langsung pada elemen-elemen dari sebuah \texttt{list} atau karakter-karakter dari sebuah \texttt{string}.

\begin{lstlisting}[style=PythonStyle, caption={Kode Python: list_and_string_iteration.py}]
# Iterasi pada sebuah list
daftar_buah = ["apel", "mangga", "jeruk"]
for buah in daftar_buah:
    print(f"Saya suka {buah}")

# Iterasi pada sebuah string
nama = "PYTHON"
for huruf in nama:
    print(huruf, end=' ')
\end{lstlisting}
\end{frame}

\subsection{While Loop}
\begin{frame}[fragile]{While Loop (Sintaks Dasar)}
Perulangan \texttt{while} akan terus mengeksekusi blok kode di dalamnya selama kondisi yang diberikan bernilai \texttt{True}.

\begin{lstlisting}[style=PythonStyle, caption={Sintaks Dasar Perulangan while}]
while kondisi:
    # Blok kode yang akan diulang
    # ...
    # Penting: Harus ada perubahan yang membuat kondisi akhirnya False
\end{lstlisting}
\end{frame}

\begin{frame}[fragile]{While Loop (Contoh Penggunaan 1)}
\begin{lstlisting}[style=PythonStyle, caption={Kode Python: while_loop.py}]
# Menghitung dari 1 sampai 5
angka = 1
while angka <= 5:
    print(f"Hitungan: {angka}")
    angka = angka + 1 # atau angka += 1

print("Selesai")
\end{lstlisting}
\end{frame}

\begin{frame}[fragile]{While Loop (Contoh Penggunaan 2)}
\begin{lstlisting}[style=PythonStyle, caption={Kode Python: cowok_selalu_salah.py}]
def cewek_nanya():
    print('Cewek: Kamu salah ga?')

def respon_cowok():
    return input('Cowok: ')

def cek_jawaban_cowok(jawaban):
    if jawaban.startswith('iy'):
        return True
    else:
        return False
\end{lstlisting}
\end{frame}

\begin{frame}[fragile]{While Loop (Contoh Penggunaan 3)}
\begin{lstlisting}[style=PythonStyle, caption={Kode Python: cowok_selalu_salah.py}]
def main():
    while True:
        cewek_nanya()
        jawaban_cowo = respon_cowok()

        if cek_jawaban_cowok(jawaban_cowo):
            break

main()
\end{lstlisting}
\end{frame}

\section{Kontrol Alur Perulangan}
\begin{frame}[fragile]{Definisi Kontrol Alur Perulangan}
Python menyediakan dua pernyataan untuk mengontrol alur eksekusi di dalam perulangan: \texttt{break} dan \texttt{continue}.
\end{frame}

\begin{frame}[fragile]{\texttt{break} Statement}
Statement \texttt{break} digunakan untuk menghentikan paksa (keluar dari) perulangan saat itu juga, bahkan jika kondisi perulangan masih terpenuhi.

\begin{lstlisting}[style=PythonStyle, caption={Kode Python: break_keyword.py}]
# Mencari angka 5 dalam rentang 1-10
for i in range(1, 11):
    print(i, end=' ')
    if i == 5:
        print("\nAngka 5 ditemukan, perulangan dihentikan!")
        break 
\end{lstlisting}
\end{frame}

\begin{frame}[fragile]{\texttt{continue} Statement}
Statement \texttt{continue} digunakan untuk melewati sisa blok kode pada iterasi saat ini dan langsung melanjutkan ke iterasi berikutnya.

\begin{lstlisting}[style=PythonStyle, caption={Kode Python: continue_keyword.py}]
# Mencetak angka ganjil dari 1 sampai 10
for i in range(1, 11):
    if i % 2 == 0: # Jika angka genap
        continue   # Lewati iterasi ini dan lanjut ke angka berikutnya
    print(f"Angka ganjil: {i}")
\end{lstlisting}
\end{frame}

\section{Perulangan Bersarang (\textit{Nested Loops})}
\begin{frame}{Definisi Perulangan Bersarang}
\textit{Nested loop} adalah sebuah perulangan yang berada di dalam perulangan lainnya. Perulangan di dalam (\textit{inner loop}) akan menyelesaikan seluruh iterasinya untuk setiap satu iterasi dari perulangan di luar (\textit{outer loop}).
\\
Konsep ini sering digunakan untuk memproses data dalam format dua dimensi, seperti matriks atau tabel.
\end{frame}

\begin{frame}[fragile]{Contoh Penggunaan Perulangan Bersarang}
\begin{lstlisting}[style=PythonStyle, caption={Kode Python: nested_loop.py}]
# Outer loop untuk baris
for i in range(1, 4):  # Baris 1 sampai 3
    # Inner loop untuk kolom
    for j in range(1, 4): # Kolom 1 sampai 3
        print(f"{i}x{j} = {i*j}", end='\t')
    print() # Pindah ke baris baru setelah inner loop selesai
\end{lstlisting}
\end{frame}

\section{Penutup}
\begin{frame}{Penutup}
    Berikut adalah poin-poin penting yang telah kita pelajari dalam bab ini:
    \begin{itemize}
        \item \textbf{Perulangan} adalah konsep fundamental untuk mengeksekusi blok kode secara berulang tanpa harus menulis ulang kode tersebut.
        \vspace{10pt}
        \item \textbf{For Loop} ideal digunakan untuk iterasi pada objek yang memiliki jumlah elemen pasti (definite iteration), seperti \texttt{list}, \texttt{string}, atau \texttt{range()}.
        \vspace{10pt}
        \item \textbf{While Loop} cocok digunakan saat jumlah perulangan tidak diketahui dan bergantung pada sebuah kondisi yang harus terpenuhi (indefinite iteration).
    \end{itemize}
\end{frame}

\begin{frame}{Penutup}
    \begin{itemize}
        \item \textbf{\texttt{break}} dan \textbf{\texttt{continue}} adalah pernyataan untuk mengontrol alur perulangan. \texttt{break} menghentikan loop sepenuhnya, sedangkan \texttt{continue} hanya melewati iterasi saat ini.
        \vspace{10pt}
        \item \textbf{Nested Loops} (Perulangan Bersarang) memungkinkan adanya perulangan di dalam perulangan lain, sangat berguna untuk memproses data berstruktur dua dimensi.
    \end{itemize}
\end{frame}
\end{document}