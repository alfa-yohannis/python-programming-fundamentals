\documentclass[aspectratio=169, table]{beamer}
\usepackage[utf8]{inputenc}
\usepackage{listings} 
\usepackage[strings]{underscore}
\usepackage{caption}
\usepackage{float}

\usepackage{tikz}
\usetikzlibrary{shapes.geometric, arrows.meta, trees, positioning}


\renewcommand{\lstlistingname}{} 

\makeatletter
\def\input@path{{../../themes/Pradita}}
\makeatother

\usetheme{Pradita}

\subtitle{IF120203-Programming Fundamentals}

\title{Chapter-12:\\\LARGE{GUI Programming 2\\}
\vspace{10pt}}
\date[Serial]{\scriptsize {PRU/SPMI/FR-BM-18/0222}}
\author[Pradita]{\small{\textbf{Alfa Yohannis}}}


% Define Python language style for listings
\lstdefinestyle{PythonStyle}{
    language=Python,
    basicstyle=\ttfamily\footnotesize,
    keywordstyle=\color{blue}\bfseries,
    commentstyle=\color{gray}\itshape,
    stringstyle=\color{red},
    showstringspaces=false,
    breaklines=true,
    frame=lines,
    numbers=left,
    numberstyle=\tiny\color{gray},
    backgroundcolor=\color{lightgray!10},
    tabsize=2,
    captionpos=b
}

\lstdefinelanguage{bash} {
	keywords={},
	basicstyle=\ttfamily\small,
	keywordstyle=\color{blue}\bfseries,
	ndkeywords={iex},
	ndkeywordstyle=\color{purple}\bfseries,
	sensitive=true,
	commentstyle=\color{gray},
	stringstyle=\color{red},
	numbers=left,
	numberstyle=\tiny\color{gray},
	breaklines=true,
	frame=lines,
	backgroundcolor=\color{lightgray!10},
	tabsize=2,
	comment=[l]{\#},
	morecomment=[s]{/*}{*/},
	commentstyle=\color{gray}\ttfamily,
	stringstyle=\color{purple}\ttfamily,
	showstringspaces=false,
	captionpos=b
}

\begin{document}

\frame{\titlepage}

% Add table of contents slide
\begin{frame}[fragile]{Contents}
\vspace{15pt}
\begin{columns}[t]
\begin{column}{.4\textwidth}
\tableofcontents[sections={1-4}]
\end{column}
\begin{column}{.6\textwidth}
\tableofcontents[sections={5-7}]
\end{column}
\end{columns}
\end{frame}


\section{Menampilkan Gambar Menggunakan File Dialog}

\begin{frame}[fragile]{Menampilkan Gambar via File Dialog}
\vspace{20pt}

\begin{columns}[T, totalwidth=0.95\linewidth]
    \begin{column}{0.4\linewidth}
        Contoh dasar pemakaian \texttt{PhotoImage} untuk
        memuat gambar, \texttt{Label} untuk menampilkan, serta
        \texttt{askopenfilename()} untuk memilih file secara
        interaktif. Referensi gambar perlu disimpan agar tidak
        hilang dari memori.
    \end{column}

    \begin{column}{0.55\linewidth}
\begin{lstlisting}[style=PythonStyle, basicstyle=\ttfamily\scriptsize]
class GambarDemo(tk.Tk):
    def __init__(self):
        super().__init__()
        self.label = tk.Label(self); self.label.pack()
        tk.Button(self,text="Pilih",
                  command=self.pilih).pack()

    def pilih(self):
        f = filedialog.askopenfilename(
            filetypes=[("Image","*.png;*.gif")])
        if f:
            img = tk.PhotoImage(file=f)
            self.label.img = img
            self.label.config(image=img)
\end{lstlisting}
    \end{column}
\end{columns}

\end{frame}


\section{Menu}

\begin{frame}[fragile]{Menu pada Tkinter}
\vspace{20pt}

\begin{columns}[T, totalwidth=\linewidth]
    \begin{column}{0.4\linewidth}
        Menu bar memungkinkan aplikasi menyediakan perintah
        seperti membuka form, menyimpan data, atau mengubah
        pengaturan. Contoh ini menunjukkan cara menghubungkan
        item menu dengan sebuah method sehingga aksi dapat
        dijalankan secara langsung ketika pengguna memilih menu.
    \end{column}

    \begin{column}{0.55\linewidth}
\begin{lstlisting}[style=PythonStyle, basicstyle=\ttfamily\scriptsize]
class MenuDemo(tk.Tk):
    def __init__(self):
        super().__init__()
        self.title("Judul Awal")

        m = tk.Menu(self)
        f = tk.Menu(m, tearoff=0)
        f.add_command(label="Ubah Judul",
                      command=self.ubah)
        m.add_cascade(label="File", menu=f)
        self.config(menu=m)

    def ubah(self):
        self.title("Judul Berubah!")
\end{lstlisting}
    \end{column}
\end{columns}

\end{frame}


\section{Navigasi Antar Form}

\begin{frame}[fragile]{Form Utama dengan Menu Navigasi}
\vspace{20pt}

\begin{columns}
    \begin{column}{0.4\linewidth}
        Form utama memiliki menu \texttt{Navigasi} untuk
        membuka form kedua. Ketika item menu dipilih, method
        \texttt{buka\_form\_kedua()} dijalankan dan membuat
        instance \texttt{SecondForm}. Pendekatan ini memisahkan
        fitur aplikasi ke dalam jendela terpisah.
    \end{column}

    \begin{column}{0.55\linewidth}
\begin{lstlisting}[style=PythonStyle, basicstyle=\ttfamily\scriptsize]
class MainWindow(tk.Tk):
    def __init__(self):
        super().__init__()
        self.title("Form Utama")

        menu = tk.Menu(self)
        nav = tk.Menu(menu, tearoff=0)
        nav.add_command(label="Buka Form Kedua",
            command=self.buka_kedua)
        menu.add_cascade(label="Navigasi", menu=nav)
        self.config(menu=menu)

    def buka_kedua(self):
        SecondForm(self)
\end{lstlisting}
    \end{column}
\end{columns}

\end{frame}

\section{Tabel Data (Table)}

% ========================= FRAME 1 =========================
\begin{frame}[fragile]{Membuat Tabel dengan ttk.Treeview}
\vspace{20pt}

Pada frame ini ditunjukkan proses pembuatan tabel dasar menggunakan \texttt{ttk.Treeview}. Kolom didefinisikan, header ditetapkan, dan beberapa baris data dimasukkan sebagai contoh. Pendekatan ini sudah cukup untuk membuat tampilan tabel sederhana pada aplikasi Tkinter.

\begin{lstlisting}[style=PythonStyle, basicstyle=\ttfamily\scriptsize]
class TableDemo(tk.Tk):
    def __init__(self):
        super().__init__()
        self.title("Contoh Tabel")
        frame = tk.Frame(self)
        frame.pack(fill="both", expand=True)
        self.tree = ttk.Treeview(frame, columns=("nama","umur"), show="headings")
        self.tree.heading("nama", text="Nama")
        self.tree.heading("umur", text="Umur")
        self.tree.column("nama", width=200)
        self.tree.column("umur", width=80, anchor="center")
        self.tree.insert("", tk.END, values=("Andi",20))
        self.tree.insert("", tk.END, values=("Budi",21))
        self.tree.insert("", tk.END, values=("Citra",19))
\end{lstlisting}

\end{frame}

% ========================= FRAME 2 =========================
\begin{frame}[fragile]{Scrollbar dan Baris Terpilih}
\vspace{20pt}

Bagian berikut menunjukkan cara menambahkan scrollbar, menampilkan tabel ke layar, serta mengambil data baris yang dipilih. Tombol digunakan untuk memicu method \texttt{show()} agar nilai terpilih dapat ditampilkan pada Label.

\begin{lstlisting}[style=PythonStyle, basicstyle=\ttfamily\scriptsize]
        sb = tk.Scrollbar(frame, orient="vertical", command=self.tree.yview)
        self.tree.config(yscrollcommand=sb.set)
        self.tree.pack(side="left", fill="both", expand=True)
        sb.pack(side="right", fill="y")
        self.label = tk.Label(self, text="Belum ada pilihan")
        self.label.pack()
        tk.Button(self, text="Tampilkan", command=self.show).pack()

    def show(self):
        s = self.tree.selection()
        if s:
            v = self.tree.item(s[0], "values")
            self.label.config(text=f"Nama={v[0]}, Umur={v[1]}")
\end{lstlisting}

\end{frame}



\section{Menggambar Gambar dan Bentuk Geometris pada Canvas Tkinter}

% ========================= FRAME 1 =========================
\begin{frame}[fragile]{Gambar: File PNG pada Canvas}
\vspace{20pt}

\begin{columns}[T]
    \begin{column}{0.35\linewidth}
        Frame pertama memperlihatkan cara menampilkan gambar
        raster (PNG) di dalam \texttt{Canvas}. Gambar dimuat
        menggunakan \texttt{PhotoImage} dan ditempatkan pada
        koordinat awal \texttt{(0,0)}. Area \texttt{Canvas}
        berfungsi sebagai ruang kerja untuk menggabungkan
        gambar dan bentuk vektor.
    \end{column}

    \begin{column}{0.6\linewidth}
\begin{lstlisting}[language=Python, style=PythonStyle, basicstyle=\ttfamily\scriptsize]
import tkinter as tk

class CanvasShapesDemo(tk.Tk):
    def __init__(self):
        super().__init__()
        self.title("Canvas: Gambar + Bentuk")
        self.geometry("600x450")

        self.canvas = tk.Canvas(self, width=560, height=400, bg="white")
        self.canvas.pack(pady=10)

        self.img = tk.PhotoImage(file="./image.png")
        self.canvas.create_image(0, 0, anchor="nw", image=self.img)

        self.gambar_bentuk()
\end{lstlisting}
    \end{column}
\end{columns}

\end{frame}

% ========================= FRAME 2 =========================
\begin{frame}[fragile]{Gambar: Persegi dan Segitiga pada Canvas}
\vspace{20pt}

\begin{columns}[T]
    \begin{column}{0.40\linewidth}
        Frame kedua menampilkan dua bentuk vektor: persegi dan
        segitiga. Persegi dibuat menggunakan
        \texttt{create\_rectangle()} dengan garis biru, sedangkan
        segitiga dibuat melalui \texttt{create\_polygon()} dengan
        outline merah dan isian merah muda. Kedua bentuk dapat
        digabung dengan gambar raster tanpa masalah.
    \end{column}

    \begin{column}{0.55\linewidth}
\begin{lstlisting}[language=Python, style=PythonStyle, basicstyle=\ttfamily\scriptsize]
    def gambar_bentuk(self):
        self.canvas.create_rectangle(50,50,150,150,outline="blue",width=3)
        self.canvas.create_polygon(250,250,200,350,300,350,outline="red",fill="pink",width=2)

if __name__ == "__main__":
    app = CanvasShapesDemo()
    app.mainloop()
\end{lstlisting}
    \end{column}
\end{columns}

\end{frame}


\section{Animasi Sederhana Menggunakan Canvas di Tkinter}

% ========================= FRAME 1 =========================
\begin{frame}[fragile]{Animasi: Konsep Dasar Animasi Halus}
\vspace{20pt}

Animasi halus di \texttt{Tkinter} dapat dicapai dengan menggabungkan  
\textbf{delta time} (selisih waktu antar frame) dan \textbf{frame rate konstan}.  
Pendekatan ini membuat animasi tetap stabil meskipun komputer sibuk atau mengalami delay.  
Frame pertama menyiapkan jendela dan area \texttt{Canvas} untuk menampilkan bola bergerak.

\begin{lstlisting}[language=Python, style=PythonStyle, basicstyle=\ttfamily\scriptsize]
import tkinter as tk
import time

class AnimasiCanvas(tk.Tk):
    def __init__(self):
        super().__init__()
        self.title("Animasi Halus di Canvas")
        self.geometry("400x300")

        self.canvas = tk.Canvas(self, width=380,
                                height=250, bg="white")
        self.canvas.pack(pady=10)
\end{lstlisting}

\end{frame}

% ========================= FRAME 2 =========================
\begin{frame}[fragile]{Animasi: Inisialisasi Bola dan Parameter Animasi}
\vspace{20pt}

Frame kedua menjelaskan proses pembuatan bola di dalam \texttt{Canvas}.  
Posisi awal, ukuran objek, kecepatan dalam piksel per detik, serta variabel delta time  
disiapkan untuk mendukung perhitungan gerakan yang konsisten pada 60 FPS.

\begin{lstlisting}[language=Python, style=PythonStyle, basicstyle=\ttfamily\scriptsize]
        self.x_pos = 10
        self.y_pos = 100
        self.ball = self.canvas.create_oval(
            self.x_pos, self.y_pos,
            self.x_pos + 30, self.y_pos + 30,
            fill="blue")

        self.dx = 120           # kecepatan px/s
        self.last = time.time() # delta time
        self.frame_delay = int(1000 / 60)
        self.gerak()
\end{lstlisting}

\end{frame}

% ========================= FRAME 3 =========================
\begin{frame}[fragile]{Animasi: Delta Time, Bouncing, \& 60 FPS}
\vspace{20pt}

\begin{columns}[T]
    \begin{column}{0.30\linewidth}
        Bagian ini inti animasi.  
        \textbf{Delta time} untuk menghitung jarak
        perpindahan setiap frame agar gerakan tetap mulus.  
        Ketika bola mencapai batas kiri atau kanan, kecepatannya
        dibalik untuk menghasilkan efek pantulan.  
        Fungsi animasi dipanggil ulang setiap \textbf{16 ms}
        untuk mempertahankan laju sekitar 60 FPS.
    \end{column}

    \begin{column}{0.65\linewidth}
\begin{lstlisting}[language=Python, style=PythonStyle, basicstyle=\ttfamily\scriptsize]
    def gerak(self):
        now = time.time()
        dt = now - self.last
        self.last = now

        move_x = self.dx * dt
        self.x_pos += move_x

        if self.x_pos >= 350:
            self.dx = -abs(self.dx)
        if self.x_pos <= 0:
            self.dx = abs(self.dx)

        self.canvas.move(self.ball, move_x, 0)
        self.after(self.frame_delay, self.gerak)

if __name__ == "__main__":
    app = AnimasiCanvas()
    app.mainloop()
\end{lstlisting}
    \end{column}
\end{columns}

\end{frame}


\section{Kontrol Objek Segitiga Menggunakan Keyboard pada Canvas Tkinter}

% ========================= FRAME 1 =========================
\begin{frame}[fragile]{Kontrol: Konsep Kontrol + Animasi Halus}
\vspace{20pt}

\begin{columns}[T]
    \begin{column}{0.30\linewidth}
        Contoh ini menggabungkan kontrol keyboard dengan teknik \textit{smooth animation}. Tombol panah tidak memindahkan objek secara langsung, tetapi mengubah nilai kecepatan (velocity). Pergerakan dihitung memakai delta time dengan frame rate 60 FPS sehingga animasi tampil mulus.
    \end{column}

    \begin{column}{0.65\linewidth}
\begin{lstlisting}[language=Python, style=PythonStyle, basicstyle=\ttfamily\scriptsize]
import tkinter as tk
import time

class TriangleControlDemo(tk.Tk):
    def __init__(self):
        super().__init__()
        self.title("Kontrol Segitiga Halus")
        self.geometry("400x400")

        self.canvas = tk.Canvas(self, width=380, height=380, bg="white")
        self.canvas.pack(pady=10)
        self.triangle = self.canvas.create_polygon(190,150,160,200,220,200,fill="blue")
\end{lstlisting}
    \end{column}
\end{columns}

\end{frame}

% ========================= FRAME 2 =========================
\begin{frame}[fragile]{Kontrol:  Gambar Segitiga + Kecepatan}
\vspace{20pt}

\begin{columns}[T]
    \begin{column}{0.30\linewidth}
        Segitiga dibuat menggunakan \texttt{create\_polygon()}. Nilai \texttt{vx} dan \texttt{vy} mewakili kecepatan horizontal dan vertikal. Event \texttt{KeyPress} dan \texttt{KeyRelease} mengubah nilai kecepatan ini, sementara loop animasi berjalan terus untuk menghitung gerakan berdasarkan delta time.
    \end{column}

    \begin{column}{0.65\linewidth}
\begin{lstlisting}[language=Python, style=PythonStyle, basicstyle=\ttfamily\scriptsize]
        self.vx = 0
        self.vy = 0
        self.speed = 160
        self.last = time.time()

        self.bind("<KeyPress>", self.on_key_down)
        self.bind("<KeyRelease>", self.on_key_up)
        self.focus_set()

        self.frame_delay = int(1000/60)
        self.update_anim()

    def on_key_down(self, event):
        if event.keysym=="Left": self.vx=-self.speed
        elif event.keysym=="Right": self.vx=self.speed
        elif event.keysym=="Up": self.vy=-self.speed
        elif event.keysym=="Down": self.vy=self.speed

\end{lstlisting}
    \end{column}
\end{columns}

\end{frame}

% ========================= FRAME 3 =========================
\begin{frame}[fragile]{Kontrol: Loop+Delta Time+Pergerakan Halus}
\vspace{20pt}

\begin{columns}[T]
    \begin{column}{0.27\linewidth}
        Fungsi animasi menghitung perpindahan berdasarkan delta time agar objek bergerak stabil. Tombol yang ditekan mengubah kecepatan; saat tombol dilepas, arah itu dihentikan. Penggunaan \texttt{after()} menjaga laju animasi tetap sekitar 60 FPS.
    \end{column}

    \begin{column}{0.68\linewidth}
\begin{lstlisting}[language=Python, style=PythonStyle, basicstyle=\ttfamily\scriptsize]
    
    def on_key_up(self, event):
        if event.keysym in ("Left","Right"): self.vx=0
        if event.keysym in ("Up","Down"): self.vy=0

    def update_anim(self):
        now = time.time()
        dt = now - self.last
        self.last = now

        self.canvas.move(self.triangle, self.vx*dt, self.vy*dt)
        self.after(self.frame_delay, self.update_anim)

if __name__=="__main__":
    app = TriangleControlDemo()
    app.mainloop()
\end{lstlisting}
    \end{column}
\end{columns}

\end{frame}

\begin{frame}{Rangkuman Bab GUI 2}
\vspace{20pt}

\begin{itemize}
    \item Menampilkan gambar dengan \texttt{PhotoImage} dan \texttt{filedialog.askopenfilename()} pada \texttt{Label}.
    \item Membuat \texttt{menu bar} dengan \texttt{Menu} dan menghubungkannya ke method (misalnya mengubah judul window).
    \item Navigasi antar form menggunakan \texttt{Toplevel} dan pemisahan form utama / form kedua.
    \item Menampilkan data terstruktur dengan \texttt{ttk.Treeview}, scrollbar, dan pemilihan baris.
    \item Menggambar gambar dan bentuk vektor (persegi, segitiga) pada \texttt{Canvas}.
    \item Animasi halus dan kontrol objek (lingkaran/segitiga) menggunakan \texttt{Canvas}, \texttt{after()}, dan event keyboard.
\end{itemize}

\end{frame}


\end{document}