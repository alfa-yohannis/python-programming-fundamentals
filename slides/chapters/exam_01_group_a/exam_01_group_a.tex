\documentclass[aspectratio=169, table]{beamer}
\usepackage[utf8]{inputenc}
\usepackage{listings} 
\usepackage[strings]{underscore}
\usepackage{caption}
\usepackage{float}


\renewcommand{\lstlistingname}{} 

\makeatletter
\def\input@path{{../../themes/Pradita}}
\makeatother

\usetheme{Pradita}

\subtitle{IF120203-Programming Fundamentals}

\title{Chapter-07:\\\LARGE{Exam File I/O pada Python (A)\\}
\vspace{10pt}}
\date[Serial]{\scriptsize {PRU/SPMI/FR-BM-18/0222}}
\author[Pradita]{\small{\textbf{Alfa Yohannis}}}


% Define Python language style for listings
\lstdefinestyle{PythonStyle}{
    language=Python,
    basicstyle=\ttfamily\footnotesize,
    keywordstyle=\color{blue}\bfseries,
    commentstyle=\color{gray}\itshape,
    stringstyle=\color{red},
    showstringspaces=false,
    breaklines=true,
    frame=lines,
    numbers=left,
    numberstyle=\tiny\color{gray},
    backgroundcolor=\color{lightgray!10},
    tabsize=2,
    captionpos=b
}

\lstdefinelanguage{bash} {
	keywords={},
	basicstyle=\ttfamily\small,
	keywordstyle=\color{blue}\bfseries,
	ndkeywords={iex},
	ndkeywordstyle=\color{purple}\bfseries,
	sensitive=true,
	commentstyle=\color{gray},
	stringstyle=\color{red},
	numbers=left,
	numberstyle=\tiny\color{gray},
	breaklines=true,
	frame=lines,
	backgroundcolor=\color{lightgray!10},
	tabsize=2,
	comment=[l]{\#},
	morecomment=[s]{/*}{*/},
	commentstyle=\color{gray}\ttfamily,
	stringstyle=\color{purple}\ttfamily,
	showstringspaces=false,
	captionpos=b
}

\begin{document}

\frame{\titlepage}

% Add table of contents slide
\begin{frame}[fragile]{Contents}
\vspace{15pt}
\begin{columns}[t]
\begin{column}{.4\textwidth}
\tableofcontents[sections={1-4}]
\end{column}
\begin{column}{.6\textwidth}
\tableofcontents[sections={5-7}]
\end{column}
\end{columns}
\end{frame}

\section{Instruksi Umum}
\begin{frame}[fragile]{Instruksi Umum}
\begin{enumerate}
\item Selesaikan soal berikut \textbf{TANPA} bantuan \textbf{ChatGPT} atau GenAI/Large-language Model AI lainnya (misal, Gemini, Grok, Deepseek, etc.).
\item Boleh menggunakan Search Engines dan Web Browser untuk mengakses dokumentasi Python dan contoh penggunaan fungsi-fungsi.
\item \textbf{Durasi test adalah 30 menit}.
\end{enumerate}
\end{frame}



\section{Soal 7}
%=============================
\begin{frame}[fragile]{Soal 7: Filter Transaksi dan Laporan Penjualan (1/3)}
\vspace{10pt}
\textbf{Tugas:}  
Buat program Python \texttt{store\_analysis.py} yang membaca dua file CSV (cabang A dan B), menggabungkan datanya, memfilter transaksi berdasarkan nilai tertentu, dan membuat laporan agregasi keseluruhan.

\textbf{Langkah:}
\begin{enumerate}
  \item Baca dua file: \texttt{store\_A.csv} dan \texttt{store\_B.csv} menggunakan \texttt{csv.DictReader}.
  \item Gabungkan seluruh data ke dalam satu list transaksi.
  \item Pilih hanya transaksi dengan \texttt{amount} lebih dari \texttt{10000}.
  \item Simpan hasil filter ke \texttt{sales\_over\_10000.csv}  
        (\texttt{branch,customer,amount}).
  \item Hitung total dan rata-rata penjualan per cabang dari \textbf{seluruh data (sebelum difilter)}.
  \item Simpan laporan ringkasan ke \texttt{report.txt}.
\end{enumerate}

\textbf{Tujuan:}  
Melatih pembacaan multi-file CSV, penggabungan data, pemfilteran numerik, dan pembuatan laporan agregasi lengkap.
\end{frame}

%=============================
\begin{frame}[fragile]{Soal 7: Data Masukan (2/3)}
\vspace{10pt}
\textbf{File Input 1: \texttt{store\_A.csv}}
\begin{lstlisting}[language=bash,basicstyle=\ttfamily\small]
branch,customer,amount
A,Andi,8000
A,Budi,12000
A,Bella,18000
\end{lstlisting}

\textbf{File Input 2: \texttt{store\_B.csv}}
\begin{lstlisting}[language=bash,basicstyle=\ttfamily\small]
branch,customer,amount
B,Bima,9000
B,Citra,20000
B,Dino,15000
\end{lstlisting}

\textbf{Deskripsi:}
  Masing-masing file mewakili data transaksi dari satu cabang.
  Kolom berisi nama cabang, pelanggan, dan jumlah transaksi.
\end{frame}

%=============================
\begin{frame}[fragile]{Soal 7: Hasil Keluaran (3/3)}
\vspace{20pt}
\textbf{Output 1: \texttt{sales\_over\_10000.csv}}  
(Hanya transaksi dengan penjualan di atas 10.000)
\begin{lstlisting}[language=bash,basicstyle=\ttfamily\small]
branch,customer,amount
A,Budi,12000
A,Bella,18000
B,Citra,20000
B,Dino,15000
\end{lstlisting}

\textbf{Output 2: \texttt{report.txt}}  
(Laporan dari seluruh data sebelum difilter)
\begin{lstlisting}[language=bash,basicstyle=\ttfamily\small]
STORE PERFORMANCE REPORT
=========================
Total branches: 2
Total sales: 82000
Average sales: 13666.67
Top branch: B (44000)
\end{lstlisting}
\end{frame}



\end{document}