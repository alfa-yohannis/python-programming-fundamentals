\documentclass[aspectratio=169, table]{beamer}
\usepackage[utf8]{inputenc}
\usepackage{listings} 
\usepackage[strings]{underscore}
\usepackage{caption}
\usepackage{float}

\usepackage{tikz}
\usetikzlibrary{shapes.geometric, arrows.meta, trees, positioning}


\renewcommand{\lstlistingname}{} 

\makeatletter
\def\input@path{{../../themes/Pradita}}
\makeatother

\usetheme{Pradita}

\subtitle{IF120203-Programming Fundamentals}

\title{Chapter-13:\\\LARGE{Debugging\\}
\vspace{10pt}}
\date[Serial]{\scriptsize {PRU/SPMI/FR-BM-18/0222}}
\author[Pradita]{\small{\textbf{Alfa Yohannis}}}


% Define Python language style for listings
\lstdefinestyle{PythonStyle}{
    language=Python,
    basicstyle=\ttfamily\footnotesize,
    keywordstyle=\color{blue}\bfseries,
    commentstyle=\color{gray}\itshape,
    stringstyle=\color{red},
    showstringspaces=false,
    breaklines=true,
    frame=lines,
    numbers=left,
    numberstyle=\tiny\color{gray},
    backgroundcolor=\color{lightgray!10},
    tabsize=2,
    captionpos=b
}

\lstdefinelanguage{bash} {
	keywords={},
	basicstyle=\ttfamily\small,
	keywordstyle=\color{blue}\bfseries,
	ndkeywords={iex},
	ndkeywordstyle=\color{purple}\bfseries,
	sensitive=true,
	commentstyle=\color{gray},
	stringstyle=\color{red},
	numbers=left,
	numberstyle=\tiny\color{gray},
	breaklines=true,
	frame=lines,
	backgroundcolor=\color{lightgray!10},
	tabsize=2,
	comment=[l]{\#},
	morecomment=[s]{/*}{*/},
	commentstyle=\color{gray}\ttfamily,
	stringstyle=\color{purple}\ttfamily,
	showstringspaces=false,
	captionpos=b
}

\begin{document}

\frame{\titlepage}

% Add table of contents slide
\begin{frame}[fragile]{Contents}
\vspace{15pt}
\begin{columns}[t]
\begin{column}{.4\textwidth}
\tableofcontents[sections={1-4}]
\end{column}
\begin{column}{.6\textwidth}
\tableofcontents[sections={5-7}]
\end{column}
\end{columns}
\end{frame}

\section{Pengenalan}
\begin{frame}{Pengenalan Debugging}
\vspace{20pt}

\begin{itemize}
    \item Debugging adalah proses menemukan, menganalisis, dan memperbaiki kesalahan (bug) dalam program.
    \item Python sensitif terhadap struktur kode, sehingga kesalahan kecil (indentasi, penamaan variabel, logika) dapat menyebabkan error.
    \item Debugging membantu menelusuri alur eksekusi program untuk memahami sumber masalah secara sistematis.
    \item Keterampilan ini meningkatkan efisiensi pengembangan karena masalah dapat ditemukan lebih cepat dan kode menjadi lebih rapi.
    \item Debugging dasar menjadi fondasi sebelum mempelajari teknik lanjutan seperti \texttt{pdb}, \texttt{breakpoint()}, dan fitur debugging pada IDE modern.
\end{itemize}

\end{frame}

\section{Jenis-Jenis Error dalam Python}

\begin{frame}[fragile]{Syntax Error}
\vspace{20pt}

\begin{columns}[T]
    \begin{column}{0.40\linewidth}
        \begin{itemize}
            \item Syntax Error terjadi ketika kode tidak mengikuti aturan sintaks Python.
            \item Muncul sebelum program dijalankan (tahap parsing).
            \item Penyebab umum: kurang titik dua, kurung tidak lengkap, indentasi salah, atau pengetikan perintah tidak valid.
            \item Pesan error menunjukkan baris yang perlu diperbaiki.
        \end{itemize}
    \end{column}

    \begin{column}{0.55\linewidth}
\begin{lstlisting}[style=PythonStyle, basicstyle=\ttfamily\scriptsize]
# Contoh Syntax Error
nilai = 75
if nilai >= 70
    print("Lulus")
\end{lstlisting}

\begin{lstlisting}[style=PythonStyle, basicstyle=\ttfamily\scriptsize]
# Perbaikan
nilai = 75
if nilai >= 70:
    print("Lulus")
\end{lstlisting}
    \end{column}
\end{columns}

\end{frame}

\begin{frame}[fragile]{Runtime Error}
\vspace{20pt}

\begin{columns}[T]
    \begin{column}{0.5\linewidth}
        Runtime Error terjadi saat program sudah berjalan tetapi menemukan operasi yang tidak valid.  
        Berbeda dengan Syntax Error, error ini hanya muncul ketika alur program benar-benar dieksekusi.

        \begin{itemize}
            \item Penyebab umum:
            \begin{itemize}
                \item pembagian dengan nol
                \item indeks list tidak valid
                \item file tidak ditemukan
                \item variabel belum didefinisikan
            \end{itemize}
            \item Penting untuk memvalidasi input dan menangani kondisi tak terduga.
        \end{itemize}
    \end{column}

    \begin{column}{0.45\linewidth}
\begin{lstlisting}[style=PythonStyle, basicstyle=\ttfamily\scriptsize]
# Contoh Runtime Error: pembagian dengan nol
angka = 10
pembagi = 0

hasil = angka / pembagi  # Error saat dijalankan
print("Hasil:", hasil)
\end{lstlisting}
    \end{column}
\end{columns}

\end{frame}

% ========================= FRAME 2 =========================

\begin{frame}[fragile]{Perbaikan Runtime Error}
\vspace{20pt}

\begin{columns}[T]
    \begin{column}{0.40\linewidth}
        Dua pendekatan umum mencegah Runtime Error:

        \begin{itemize}
            \item \textbf{Pengecekan kondisi} sebelum operasi dilakukan.
            \item \textbf{try-except} untuk menangani error tanpa menghentikan program.
        \end{itemize}

        Cara ini menjaga program tetap aman dan stabil.
    \end{column}

    \begin{column}{0.55\linewidth}
\begin{lstlisting}[style=PythonStyle, basicstyle=\ttfamily\scriptsize]
# Perbaikan dengan pengecekan
angka = 10
pembagi = 0

if pembagi != 0:
    print(angka / pembagi)
else:
    print("Error: pembagi tidak boleh nol.")

# Perbaikan dengan try-except
try:
    print(angka / pembagi)
except ZeroDivisionError:
    print("Terjadi kesalahan: tidak bisa membagi dengan nol.")
\end{lstlisting}
    \end{column}
\end{columns}

\end{frame}


\begin{frame}[fragile]{Logic Error}
\vspace{20pt}

\begin{columns}[T]
    \begin{column}{0.5\linewidth}
        Logic Error terjadi ketika program berjalan tanpa pesan error, tetapi hasilnya salah.  
        Python tidak memberikan peringatan karena sintaks dan eksekusi tidak bermasalah.

        \begin{itemize}
            \item Penyebab umum:
            \begin{itemize}
                \item operator salah
                \item kondisi percabangan keliru
                \item rumus perhitungan salah
                \item alur algoritma tidak tepat
            \end{itemize}
            \item Cara menemukan: memahami alur logika, memeriksa nilai variabel, print debugging.
        \end{itemize}
    \end{column}

    \begin{column}{0.45\linewidth}
\begin{lstlisting}[style=PythonStyle, basicstyle=\ttfamily\scriptsize]
# Logic Error: operator salah
nilai = 85

if nilai <= 70:  # Logika terbalik
    print("Lulus")
else:
    print("Tidak lulus")

# Perbaikan
nilai = 85
if nilai >= 70:
    print("Lulus")
else:
    print("Tidak lulus")
\end{lstlisting}
    \end{column}
\end{columns}

\end{frame}

% ========================= FRAME 2 =========================

\begin{frame}[fragile]{Logic Error dalam Perhitungan}
\vspace{20pt}

\begin{columns}[T]
    \begin{column}{0.40\linewidth}
        Logic Error juga sering muncul pada operasi matematis, terutama jika rumus
        atau jumlah pembagi tidak sesuai.  
        Contoh berikut memperlihatkan kesalahan menghitung rata-rata karena salah
        menggunakan pembagi.
    \end{column}

    \begin{column}{0.55\linewidth}
\begin{lstlisting}[style=PythonStyle, basicstyle=\ttfamily\scriptsize]
# Logic Error: rata-rata salah
a = 80; b = 90; c = 70
rata = (a + b + c) / 2   # Seharusnya 3
print("Rata-rata:", rata)

# Perbaikan
a = 80; b = 90; c = 70
rata = (a + b + c) / 3
print("Rata-rata:", rata)
\end{lstlisting}
    \end{column}
\end{columns}

\end{frame}

\section{Teknik Dasar Debugging}

\begin{frame}[fragile]{Print Debugging: Konsep dan Contoh Dasar}
\vspace{20pt}

\begin{columns}[T]
    \begin{column}{0.6\linewidth}
        Print debugging adalah teknik sederhana untuk memahami
        alur eksekusi program dengan menampilkan nilai variabel
        atau kondisi tertentu menggunakan \texttt{print()}.

        \begin{itemize}
            \item Mudah digunakan tanpa alat tambahan.
            \item Berguna untuk melacak perubahan variabel.
            \item Membantu memastikan blok kode dijalankan.
            \item Keterbatasan: jika terlalu banyak, kode menjadi berantakan.
        \end{itemize}

        Teknik ini sangat cocok untuk pemula sebelum menggunakan
        debugger yang lebih canggih.
    \end{column}

    \begin{column}{0.35\linewidth}
\begin{lstlisting}[style=PythonStyle, basicstyle=\ttfamily\scriptsize]
# Print debugging melacak perubahan nilai
total = 0

for i in range(1, 6):
    total += i
    print("Iterasi:", i,
          "Total sementara:", total)

print("Total akhir:", total)
\end{lstlisting}
    \end{column}
\end{columns}

\end{frame}

% ========================= FRAME 2 =========================

\begin{frame}[fragile]{Print Debug untuk Menemukan Logic Error}
\vspace{20pt}

\begin{columns}
    \begin{column}{0.35\linewidth}
        Print debugging juga membantu menemukan Logic Error,
        yaitu ketika logika program salah tetapi tidak
        menghasilkan pesan error.

        Contoh berikut menunjukkan kesalahan logika ketika
        menghitung jumlah bilangan genap. Debug print membantu
        menunjukkan data mana yang diproses secara keliru.
    \end{column}

    \begin{column}{0.6\linewidth}
\begin{lstlisting}[style=PythonStyle, basicstyle=\ttfamily\scriptsize]
# Logic Error ditemukan dengan print debugging
angka = [1,2,3,4,5,6]
jumlah_genap = 0

for x in angka:
    print("Memeriksa:", x)
    if x % 2 == 1:   # Logika salah
        jumlah_genap += 1
        print("Ditambahkan:", x)

print("Jumlah genap:", jumlah_genap)

# Perbaikan
jumlah_genap = 0
for x in angka:
    print("Memeriksa:", x)
    if x % 2 == 0:   # Logika benar
        jumlah_genap += 1
        print("Ditambahkan:", x)

print("Jumlah genap:", jumlah_genap)
\end{lstlisting}
    \end{column}
\end{columns}

\end{frame}


\begin{frame}[fragile]{Memahami Traceback}
\vspace{20pt}

\begin{columns}[T]
    \begin{column}{0.40\linewidth}
        Traceback adalah pesan yang ditampilkan Python ketika
        terjadi error saat program dijalankan. Pesan ini
        menunjukkan:

        \begin{itemize}
            \item file tempat error terjadi
            \item nomor baris
            \item baris kode yang bermasalah
            \item jenis error (misalnya \texttt{NameError}, \texttt{TypeError})
        \end{itemize}

        Memahami traceback membantu programmer menemukan sumber
        masalah tanpa menebak-nebak, terutama untuk error yang
        terjadi saat runtime.
    \end{column}

    \begin{column}{0.55\linewidth}
\begin{lstlisting}[style=PythonStyle, basicstyle=\ttfamily\scriptsize]
# NameError: variabel belum didefinisikan
x = 10
print(y)
\end{lstlisting}

\begin{lstlisting}[language=bash, basicstyle=\ttfamily\scriptsize]
Traceback (most recent call last):
  File "contoh.py", line 3, in <module>
    print(y)
NameError: name 'y' is not defined
\end{lstlisting}

\begin{lstlisting}[style=PythonStyle, basicstyle=\ttfamily\scriptsize]
# Perbaikan
x = 10
y = 5
print(y)
\end{lstlisting}
    \end{column}
\end{columns}

\end{frame}

% ========================= FRAME 2 =========================

\begin{frame}[fragile]{Traceback untuk TypeError}
\vspace{20pt}

\begin{columns}[T]
    \begin{column}{0.40\linewidth}
        Traceback juga membantu mendeteksi kesalahan argumen
        pada pemanggilan fungsi. Ketika fungsi dipanggil dengan
        jumlah argumen yang salah, Python menghasilkan
        \texttt{TypeError} beserta informasi detail lokasi error.
    \end{column}

    \begin{column}{0.55\linewidth}
\begin{lstlisting}[style=PythonStyle, basicstyle=\ttfamily\scriptsize]
# TypeError: argumen kurang
def tambah(a, b):
    return a + b

hasil = tambah(10)   # Harusnya 2 argumen
\end{lstlisting}

\begin{lstlisting}[language=bash, basicstyle=\ttfamily\scriptsize]
Traceback (most recent call last):
  File "contoh2.py", line 5, in <module>
    hasil = tambah(10)
TypeError: tambah() missing 1 required
positional argument: 'b'
\end{lstlisting}

\begin{lstlisting}[style=PythonStyle, basicstyle=\ttfamily\scriptsize]
# Perbaikan
def tambah(a, b):
    return a + b

hasil = tambah(10, 20)
print(hasil)
\end{lstlisting}
    \end{column}
\end{columns}

\end{frame}


% ========================= FRAME 1 =========================
\begin{frame}[fragile]{Membaca Pesan Error: ValueError}
\vspace{18pt}

\begin{columns}[T]
    \begin{column}{0.42\linewidth}
        Membaca pesan error sangat penting karena error Python
        selalu memberi tiga informasi utama:

        \begin{itemize}
            \item lokasi error (file dan nomor baris)
            \item potongan kode yang salah
            \item jenis dan deskripsi error
        \end{itemize}

        \texttt{ValueError} muncul ketika nilai yang diberikan
        ke suatu fungsi tidak valid. Pesan error membantu
        programmer memahami bahwa input harus divalidasi
        sebelum diproses.
    \end{column}

    \begin{column}{0.54\linewidth}
\begin{lstlisting}[style=PythonStyle, basicstyle=\ttfamily\scriptsize]
# ValueError: string tidak dapat dikonversi
angka = int("abc")
print(angka)
\end{lstlisting}

\begin{lstlisting}[language=bash, basicstyle=\ttfamily\scriptsize]
Traceback (most recent call last):
  File "contoh3.py", line 2, in <module>
    angka = int("abc")
ValueError: invalid literal for int()
with base 10: 'abc'
\end{lstlisting}

\begin{lstlisting}[style=PythonStyle, basicstyle=\ttfamily\scriptsize]
# Solusi: validasi input
s = "123"
angka = int(s)
print(angka)
\end{lstlisting}
    \end{column}
\end{columns}

\end{frame}

% ========================= FRAME 2 =========================
\begin{frame}[fragile]{Membaca Pesan Error: TypeError}
\vspace{18pt}

\begin{columns}[T]
    \begin{column}{0.42\linewidth}
        \texttt{TypeError} terjadi ketika operasi dilakukan
        pada tipe data yang tidak kompatibel, misalnya
        menjumlahkan \texttt{int} dengan \texttt{str}.

        Dengan membaca traceback, programmer dapat segera
        mengetahui:

        \begin{itemize}
            \item baris mana yang salah
            \item tipe data apa yang tidak cocok
            \item bagaimana memperbaikinya (konversi tipe)
        \end{itemize}
    \end{column}

    \begin{column}{0.54\linewidth}
\begin{lstlisting}[style=PythonStyle, basicstyle=\ttfamily\scriptsize]
# TypeError: operasi tidak sesuai tipe
a = 10
b = "20"
hasil = a + b
print(hasil)
\end{lstlisting}

\begin{lstlisting}[language=bash, basicstyle=\ttfamily\scriptsize]
Traceback (most recent call last):
  File "contoh4.py", line 5, in <module>
    hasil = a + b
TypeError: unsupported operand type(s)
for +: 'int' and 'str'
\end{lstlisting}

\begin{lstlisting}[style=PythonStyle, basicstyle=\ttfamily\scriptsize]
# Perbaikan dengan konversi tipe
a = 10
b = "20"
hasil = a + int(b)
print(hasil)
\end{lstlisting}
    \end{column}
\end{columns}

\end{frame}


% ========================= FRAME 1 =========================
\begin{frame}[fragile]{Mengenal Debugger Bawaan Python: \texttt{pdb}}
\vspace{16pt}

\begin{columns}[T]
    \begin{column}{0.44\linewidth}
\textbf{Modul \texttt{pdb}} adalah debugger bawaan Python yang dapat menghentikan eksekusi (breakpoint), memeriksa nilai variabel, menelusuri kode baris demi baris, masuk ke fungsi (step), melanjutkan eksekusi, dan menguji ekspresi dengan \texttt{p}. \textbf{Debugger} sangat membantu ketika print debugging kurang efektif, logika program kompleks, atau dibutuhkan analisis nilai secara interaktif. Breakpoint dapat ditambahkan dengan:
\textbf{\texttt{import pdb; pdb.set\_trace()}}


    \end{column}

    \begin{column}{0.52\linewidth}
\begin{lstlisting}[style=PythonStyle, basicstyle=\ttfamily\scriptsize]
# Contoh pdb untuk kesalahan logika
def hitung_diskon(harga, diskon):
    hasil = harga - harga * diskon  # SALAH
    return hasil

import pdb; pdb.set_trace()

total = hitung_diskon(100000, 20)
print("Total harga:", total)
\end{lstlisting}

\begin{lstlisting}[language=bash, basicstyle=\ttfamily\scriptsize]
> contoh_pdb.py(8)<module>()
-> total = hitung_diskon(100000, 20)
(Pdb)
\end{lstlisting}

\begin{lstlisting}[language=bash, basicstyle=\ttfamily\scriptsize]
(Pdb) n
(Pdb) p total
-1900000
\end{lstlisting}
    \end{column}
\end{columns}

\end{frame}

% ========================= FRAME 2 =========================
\begin{frame}[fragile]{Perbaikan Logika dan Cara Menjalankan \texttt{pdb}}
\vspace{16pt}

\begin{columns}[T]
    \begin{column}{0.46\linewidth}
        Setelah menggunakan \texttt{pdb}, kesalahan logika dapat
        ditemukan dengan cepat. Pada contoh sebelumnya, nilai
        diskon ditafsirkan sebagai angka murni, bukan persen.

        Berikut versi fungsi yang benar:

        Jalankan \texttt{pdb} tanpa mengubah kode:
        \begin{itemize}
            \item melalui terminal: \texttt{python -m pdb file.py}
            \item cocok untuk debugging cepat
        \end{itemize}

        Dengan memahami \texttt{pdb}, programmer dapat
        menganalisis alur eksekusi kompleks secara efisien.
    \end{column}

    \begin{column}{0.50\linewidth}
\begin{lstlisting}[style=PythonStyle, basicstyle=\ttfamily\scriptsize]
# Perbaikan fungsi diskon
def hitung_diskon(harga, diskon_persen):
    diskon = harga * (diskon_persen / 100)
    return harga - diskon

total = hitung_diskon(100000, 20)
print("Total harga:", total)
\end{lstlisting}

\begin{lstlisting}[language=bash, basicstyle=\ttfamily\scriptsize]
# Menjalankan pdb via terminal
python -m pdb program.py
\end{lstlisting}
    \end{column}
\end{columns}

\end{frame}

% ========================= FRAME 1 =========================
\begin{frame}[fragile]{Breakpoint (\texttt{breakpoint()})}
\vspace{10pt}

\begin{columns}[T]
\begin{column}{0.6\linewidth}
Fungsi \texttt{breakpoint()} adalah cara modern untuk
mengaktifkan debugger sejak Python 3.7. Dengan satu baris
kode, eksekusi dihentikan pada titik tertentu sehingga
programmer dapat:

\begin{itemize}
    \item memeriksa nilai variabel
    \item menelusuri alur kode baris demi baris
    \item menjalankan perintah interaktif
    \item melanjutkan eksekusi (\texttt{continue}) atau masuk fungsi (\texttt{step})
\end{itemize}

Secara default, \texttt{breakpoint()} membuka \texttt{pdb},
tetapi dapat diarahkan ke debugger lain (misal \texttt{ipdb})
melalui variabel lingkungan \texttt{PYTHONBREAKPOINT}.
\end{column}

\begin{column}{0.35\linewidth}
\begin{lstlisting}[style=PythonStyle, basicstyle=\ttfamily\scriptsize]
# Contoh penggunaan breakpoint()
angka = [10, 20, 30, 40]
total = 0

for x in angka:
    breakpoint()   # berhenti di sini
    total += x
    print("Total sementara:", total)

print("Total akhir:", total)
\end{lstlisting}
\end{column}
\end{columns}

\end{frame}

% ========================= FRAME 2 =========================
\begin{frame}[fragile]{Output Debugger dan Pengaturan Lanjut}
\vspace{10pt}

\begin{columns}[T]
\begin{column}{0.57\linewidth}
Ketika program mencapai \texttt{breakpoint()}, debugger aktif
dan menampilkan posisi eksekusi:

\begin{lstlisting}[language=bash, basicstyle=\ttfamily\scriptsize]
> contoh_breakpoint.py(6)<module>()
-> total += x
(Pdb)
\end{lstlisting}

Contoh perintah debugging:
\begin{lstlisting}[language=bash, basicstyle=\ttfamily\scriptsize]
(Pdb) p x
10
(Pdb) p total
0
(Pdb) n
\end{lstlisting}

Debugger memudahkan inspeksi nilai secara langsung sehingga
logic error lebih cepat ditemukan.
\end{column}

\begin{column}{0.38\linewidth}
\textbf{Menggunakan debugger lain}

\texttt{breakpoint()} dapat diarahkan ke \texttt{ipdb},
\texttt{pudb}, atau debugger lain melalui variabel lingkungan:

\begin{lstlisting}[language=bash, basicstyle=\ttfamily\scriptsize]
export PYTHONBREAKPOINT=ipdb.set_trace
python program.py
\end{lstlisting}

Dengan konfigurasi ini, \texttt{breakpoint()} akan membuka
\texttt{ipdb} alih-alih \texttt{pdb}, memberi fleksibilitas
lebih dalam debugging program besar atau antarmuka kaya.
\end{column}
\end{columns}

\end{frame}

% ========================= FRAME 1 =========================
\begin{frame}[fragile]{Debugging di VS Code: Dasar dan Breakpoint}
\vspace{10pt}

\begin{columns}[T]

% ---------------- LEFT COLUMN ----------------
\begin{column}{0.54\linewidth}
VS Code menyediakan (plugin) debugging Python yang memampukan:

\begin{itemize}
    \item memasang \textbf{breakpoint} dengan klik di kiri baris
    \item menjalankan debug via \textbf{Run and Debug} atau \textbf{F5}
    \item melihat variabel di panel \textit{Variables}
    \item memantau ekspresi melalui \textit{Watch}
    \item menelusuri kode dengan \textit{step over/into/out}
    \item memeriksa \textit{Call Stack}
\end{itemize}

Debugging menjadi lebih mudah secara visual.
\end{column}

% ---------------- RIGHT COLUMN ----------------
\begin{column}{0.41\linewidth}

Contoh program untuk diuji dengan debugger:

\begin{lstlisting}[style=PythonStyle, basicstyle=\ttfamily\scriptsize]
def hitung_total(harga, jumlah):
    return harga * jumlah

def main():
    total = hitung_total(25000, 3)
    print("Total:", total)

if __name__ == "__main__":
    main()
\end{lstlisting}

Tambahkan breakpoint dan tekan \textbf{F5}.
\end{column}

\end{columns}

\end{frame}



% ========================= FRAME 2 =========================
\begin{frame}[fragile]{Konfigurasi Debug VS Code (\texttt{launch.json})}
\vspace{10pt}

\begin{columns}[T]

% ---------------- LEFT COLUMN ----------------
\begin{column}{0.6\linewidth}
Untuk debugging terstruktur, VS Code dapat membuat
\texttt{launch.json} secara otomatis. Contoh:

\begin{lstlisting}[language=bash, basicstyle=\ttfamily\scriptsize]
{
 "version": "0.2.0",
 "configurations": [
  {
   "name": "Python Debug",
   "type": "python",
   "request": "launch",
   "program": "${file}",
   "console": "integratedTerminal"
  }
 ]
}
\end{lstlisting}

Setelah file dibuat, tekan \textbf{F5} untuk mulai debugging.
VS Code akan berhenti pada breakpoint pertama.
\end{column}

% ---------------- RIGHT COLUMN ----------------
\begin{column}{0.35\linewidth}

Panel Debug menampilkan informasi:

\begin{itemize}
    \item \textbf{Variables} — nilai variabel terkini
    \item \textbf{Watch} — ekspresi yang dipantau
    \item \textbf{Call Stack} — jejak pemanggilan fungsi
\end{itemize}

VS Code membuat debugging lebih visual,
intuitif, dan sangat membantu terutama bagi pemula.
\end{column}

\end{columns}

\end{frame}

\section{Best Practices Debugging}

% ========================= FRAME 1 =========================
\begin{frame}{Best Practices Debugging}
\vspace{10pt}

\begin{itemize}
    \item Jaga kode tetap rapi dan terstruktur agar mudah dianalisis.
    \item Reproduksi error secara konsisten sebelum memperbaiki.
    \item Baca traceback dan pesan error secara lengkap.
    \item Gunakan alat debugging sesuai kebutuhan:
    \begin{itemize}
        \item \texttt{print()} — untuk pengecekan cepat
        \item \texttt{pdb} / \texttt{breakpoint()} — debugging interaktif
        \item VS Code Debugger — visual, breakpoint, variable inspector
    \end{itemize}
    \item Gunakan \textit{unit testing} untuk mendeteksi error lebih awal.
    \item Validasi nilai penting dengan \textit{assertion}.
    \item Gunakan Git untuk melacak perubahan dan mempermudah rollback.
\end{itemize}

\end{frame}

\section{Rangkuman}

% ========================= FRAME 2 =========================
\begin{frame}{Rangkuman Debugging Python}
\vspace{10pt}

\begin{itemize}
    \item Debugging adalah proses menemukan dan memperbaiki kesalahan program.
    \item Tiga jenis error utama:
    \begin{itemize}
        \item \textbf{Syntax Error} — pelanggaran sintaks, muncul sebelum run.
        \item \textbf{Runtime Error} — terjadi saat eksekusi program.
        \item \textbf{Logic Error} — hasil salah tanpa pesan error.
    \end{itemize}
    \item Traceback membantu menemukan lokasi dan penyebab error.
    \item Teknik debugging umum:
    \begin{itemize}
        \item print debugging
        \item \texttt{pdb} / \texttt{breakpoint()}
        \item Debugger visual di IDE (VS Code)
    \end{itemize}
    \item Praktik terbaik: analisis sistematis, kode rapi, dan penggunaan alat yang tepat.
\end{itemize}

\end{frame}



\end{document}