\documentclass[aspectratio=169, table]{beamer}
\usepackage[utf8]{inputenc}
\usepackage{listings} 
\usepackage[strings]{underscore}
\usepackage{caption}
\usepackage{float}

\usepackage{tikz}
\usetikzlibrary{shapes.geometric, arrows.meta, trees, positioning}


\renewcommand{\lstlistingname}{} 

\makeatletter
\def\input@path{{../../themes/Pradita}}
\makeatother

\usetheme{Pradita}

\subtitle{IF120203-Programming Fundamentals}

\title{Chapter-11:\\\LARGE{GUI Programming 1\\}
\vspace{10pt}}
\date[Serial]{\scriptsize {PRU/SPMI/FR-BM-18/0222}}
\author[Pradita]{\small{\textbf{Alfa Yohannis}}}


% Define Python language style for listings
\lstdefinestyle{PythonStyle}{
    language=Python,
    basicstyle=\ttfamily\footnotesize,
    keywordstyle=\color{blue}\bfseries,
    commentstyle=\color{gray}\itshape,
    stringstyle=\color{red},
    showstringspaces=false,
    breaklines=true,
    frame=lines,
    numbers=left,
    numberstyle=\tiny\color{gray},
    backgroundcolor=\color{lightgray!10},
    tabsize=2,
    captionpos=b
}

\lstdefinelanguage{bash} {
	keywords={},
	basicstyle=\ttfamily\small,
	keywordstyle=\color{blue}\bfseries,
	ndkeywords={iex},
	ndkeywordstyle=\color{purple}\bfseries,
	sensitive=true,
	commentstyle=\color{gray},
	stringstyle=\color{red},
	numbers=left,
	numberstyle=\tiny\color{gray},
	breaklines=true,
	frame=lines,
	backgroundcolor=\color{lightgray!10},
	tabsize=2,
	comment=[l]{\#},
	morecomment=[s]{/*}{*/},
	commentstyle=\color{gray}\ttfamily,
	stringstyle=\color{purple}\ttfamily,
	showstringspaces=false,
	captionpos=b
}

\begin{document}

\frame{\titlepage}

% Add table of contents slide
\begin{frame}[fragile]{Contents}
\vspace{15pt}
\begin{columns}[t]
\begin{column}{.4\textwidth}
\tableofcontents[sections={1-4}]
\end{column}
\begin{column}{.6\textwidth}
\tableofcontents[sections={5-7}]
\end{column}
\end{columns}
\end{frame}

\section{Pengenalan Tkinter}
\begin{frame}[fragile]{Pengenalan Tkinter}
\vspace{20pt}
Tkinter adalah pustaka standar Python untuk membuat GUI.  
Konsep dasarnya dapat dijelaskan sebagai berikut:

\begin{itemize}
    \item Menyediakan jendela, tombol, label, dan elemen visual lainnya.
    \item Menggunakan model \textit{event-driven} untuk merespons tindakan pengguna.
    \item Sudah termasuk dalam instalasi Python, sehingga mudah digunakan pemula.
    \item Cocok untuk memahami bagaimana aplikasi modern bekerja.
\end{itemize}
\end{frame}

\section{Membuat Window Utama}
\begin{frame}[fragile]{Membuat Window Utama}
\vspace{20pt}
Pendekatan berorientasi objek membantu membuat aplikasi Tkinter lebih terstruktur.

\begin{itemize}
    \item Kelas utama biasanya mewarisi \texttt{tk.Tk}.
    \item Konfigurasi jendela ditempatkan dalam konstruktor.
    \item Mudah diperluas, dikelola, dan diorganisasikan.
\end{itemize}

\begin{lstlisting}[style=PythonStyle]
import tkinter as tk

class MainWindow(tk.Tk):
    def __init__(self):
        super().__init__()
        self.title("Aplikasi Tkinter Pertama")
        self.geometry("400x300")

if __name__ == "__main__":
    app = MainWindow()
    app.mainloop()
\end{lstlisting}
\end{frame}

%----------------------------------------


\section{Label}

\begin{frame}[fragile]{Label di Tkinter}
\vspace{20pt}
Label adalah widget dasar untuk menampilkan teks atau gambar pada GUI.

\begin{itemize}
    \item Dibuat dengan kelas \texttt{Label}.
    \item Mendukung teks, gambar, warna, dan pengaturan font.
    \item Harus ditempatkan menggunakan \texttt{pack()}, \texttt{grid()}, atau \texttt{place()}.
    \item Digunakan untuk judul, deskripsi, atau informasi statis.
\end{itemize}
\end{frame}

\begin{frame}[fragile]{Contoh Penggunaan Label}
\vspace{20pt}
\begin{lstlisting}[style=PythonStyle]
import tkinter as tk

root = tk.Tk()
root.title("Contoh Label")

label1 = tk.Label(root, text="Halo, selamat datang di Tkinter!")
label2 = tk.Label(root, text="Label dengan properti.", font=("Arial", 14), fg="blue", bg="lightgray")

label1.pack(pady=10)
label2.pack(pady=10)

root.mainloop()
\end{lstlisting}
\end{frame}


\section{Button}
\begin{frame}[fragile]{Button di Tkinter}
\vspace{20pt}
Button adalah widget interaktif yang digunakan untuk menjalankan aksi tertentu ketika ditekan.  

\begin{itemize}
    \item Dibuat menggunakan kelas \texttt{Button}.
    \item Menampilkan teks untuk menjelaskan fungsinya.
    \item Aksi dijalankan melalui \texttt{command}.
    \item Umum digunakan dalam pendekatan OOP agar tombol dapat berinteraksi dengan widget lain.
\end{itemize}
\end{frame}

\begin{frame}[fragile]{Contoh Penggunaan Button}
\vspace{20pt}
\begin{lstlisting}[style=PythonStyle, basicstyle=\ttfamily\scriptsize]
import tkinter as tk

class ButtonDemo(tk.Tk):
    def __init__(self):
        super().__init__()
        self.title("Contoh Button")
        self.geometry("300x200")

        self.label = tk.Label(self, text="Teks awal")
        self.label.pack(pady=10)

        self.button = tk.Button(self, text="Klik Saya", command=self.ubah_teks)
        self.button.pack(pady=10)

    def ubah_teks(self):
        self.label.config(text="Tombol ditekan!")

if __name__ == "__main__":
    app = ButtonDemo()
    app.mainloop()
\end{lstlisting}
\end{frame}

\section{EditText (Entry)}
\begin{frame}[fragile]{Entry di Tkinter}
\vspace{20pt}
Entry adalah widget untuk menerima input teks dari pengguna.

\begin{itemize}
    \item Digunakan untuk memasukkan data seperti nama atau angka.
    \item Nilai diambil menggunakan \texttt{get()}.
    \item Label diperbarui memakai \texttt{config()}.
    \item Umumnya dipadukan dengan Button untuk memicu aksi.
\end{itemize}
\end{frame}

\begin{frame}[fragile]{Contoh Entry dan Label}
\vspace{20pt}
\begin{lstlisting}[style=PythonStyle, basicstyle=\ttfamily\scriptsize]
import tkinter as tk

class EntryDemo(tk.Tk):
    def __init__(self):
        super().__init__()
        self.title("Contoh Entry dan Label")
        self.geometry("300x150")
        self.label = tk.Label(self, text="Hasil akan tampil di sini")
        self.label.pack(pady=5)
        self.entry = tk.Entry(self)
        self.entry.pack(pady=5)
        self.button = tk.Button(self, text="Tampilkan", command=self.copy_text)
        self.button.pack(pady=5)

    def copy_text(self):
        teks = self.entry.get()
        self.label.config(text=teks)

if __name__ == "__main__":
    app = EntryDemo()
    app.mainloop()
\end{lstlisting}
\end{frame}

\section{Layout pada Tkinter}
\begin{frame}[fragile]{Pengenalan Sistem Layout}
\vspace{20pt}
Tkinter menyediakan tiga metode utama untuk mengatur posisi widget:

\begin{itemize}
    \item \texttt{pack()} – menyusun widget secara vertikal atau horizontal.
    \item \texttt{grid()} – mengatur widget dalam baris dan kolom seperti tabel.
    \item \texttt{place()} – menempatkan widget berdasarkan koordinat absolut.
\end{itemize}

Memahami karakteristik ketiganya membantu pengembang memilih sistem layout yang sesuai dengan kebutuhan desain aplikasi.
\end{frame}

\begin{frame}[fragile]{Menggunakan \texttt{pack()}}
\vspace{30pt}
Metode \texttt{pack()} adalah layout paling sederhana di Tkinter.

\textbf{Parameter penting:}
\begin{itemize}
    \item \texttt{side} — menentukan sisi penempatan (top, left, right, bottom).
    \item \texttt{fill} — melebar secara horizontal (\texttt{x}) atau vertikal (\texttt{y}).
    \item \texttt{expand} — memanfaatkan ruang kosong ketika jendela diperbesar.
\end{itemize}

Cocok untuk tampilan berurutan dan struktur layout sederhana.
\end{frame}

\begin{frame}[fragile]{Kode \texttt{pack()}}
\vspace{20pt}
\begin{lstlisting}[style=PythonStyle, basicstyle=\ttfamily\scriptsize]
import tkinter as tk

class PackDemo(tk.Tk):
    def __init__(self):
        super().__init__()
        self.title("Contoh pack() Lengkap")
        self.geometry("300x250")

        tk.Label(self, text="Atas", bg="lightblue").pack(side="top", fill="x", pady=5)
        tk.Button(self, text="Kiri", bg="yellow").pack(side="left", fill="y", padx=10, pady=10)
        tk.Button(self, text="Kanan", bg="cyan").pack(side="right", expand=True, padx=10, pady=10)
        tk.Label(self, text="Bawah", bg="lightgreen").pack(side="bottom", pady=5)

if __name__ == "__main__":
    app = PackDemo()
    app.mainloop()
\end{lstlisting}
\end{frame}



\begin{frame}[fragile]{Menggunakan \texttt{grid()}}
\vspace{30pt}
Metode \texttt{grid()} menempatkan widget dalam baris dan kolom.  
\textbf{Parameter penting}:

\begin{itemize}
    \item \texttt{row}, \texttt{column} — posisi widget.
    \item \texttt{columnspan} — melebar ke beberapa kolom.
    \item \texttt{sticky} — menempel ke arah N, S, E, W.
\end{itemize}

Pendekatan berbasis kelas memudahkan pembuatan form terstruktur dan mudah dirawat.
\end{frame}

\begin{frame}[fragile]{Kode \texttt{grid()}}
\vspace{20pt}
\begin{lstlisting}[style=PythonStyle, basicstyle=\ttfamily\scriptsize]
import tkinter as tk

class GridDemo(tk.Tk):
    def __init__(self):
        super().__init__()
        self.title("Contoh grid()")
        self.geometry("300x150")
        tk.Label(self, text="Nama:").grid(row=0, column=0, padx=5, pady=5)
        self.entry_nama = tk.Entry(self)
        self.entry_nama.grid(row=0, column=1, padx=5, pady=5)

        tk.Label(self, text="Umur:").grid(row=1, column=0, padx=5, pady=5)
        self.entry_umur = tk.Entry(self)
        self.entry_umur.grid(row=1, column=1, padx=5, pady=5)

        tk.Button(self, text="Simpan").grid(row=2, column=0, columnspan=2, pady=10)

if __name__ == "__main__":
    app = GridDemo()
    app.mainloop()
\end{lstlisting}
\end{frame}


\begin{frame}[fragile]{Menggunakan \texttt{place()}}
\vspace{20pt}
Metode \texttt{place()} menempatkan widget menggunakan koordinat.

\begin{itemize}
    \item Menggunakan koordinat absolut: \texttt{x}, \texttt{y}.
    \item Mendukung posisi relatif: \texttt{relx}, \texttt{rely}.
    \item Memberikan kontrol penuh terhadap posisi widget.
    \item Kurang cocok untuk layout responsif atau jendela dinamis.
\end{itemize}
\end{frame}

\begin{frame}[fragile]{Kode \texttt{place()}}
\vspace{20pt}
\begin{lstlisting}[style=PythonStyle, basicstyle=\ttfamily\scriptsize]
import tkinter as tk

class PlaceDemo(tk.Tk):
    def __init__(self):
        super().__init__()
        self.title("Contoh place()")
        self.geometry("300x200")

        tk.Label(self, text="Label di x=20, y=30").place(x=20, y=30)
        tk.Button(self, text="Tombol").place(x=100, y=100)
        tk.Entry(self).place(x=150, y=150, width=120)

if __name__ == "__main__":
    app = PlaceDemo()
    app.mainloop()
\end{lstlisting}
\end{frame}


\begin{frame}[fragile]{Layout dalam Frame}
\vspace{20pt}
Frame adalah wadah (container) untuk mengelompokkan widget dalam area tertentu.

\begin{itemize}
    \item Membantu membuat layout lebih rapi dan terstruktur.
    \item Setiap frame dapat memakai metode layout berbeda (nested layout).
    \item Berguna untuk membagi antarmuka menjadi bagian seperti header, form, dan tombol.
    \item Pada pendekatan OOP, frame dibuat di konstruktor agar antarmuka modular.
\end{itemize}
\end{frame}

\begin{frame}[fragile]{Kode Layout dalam Frame}
\vspace{20pt}
\begin{lstlisting}[style=PythonStyle, basicstyle=\ttfamily\scriptsize]
import tkinter as tk

class FrameDemo(tk.Tk):
    def __init__(self):
        super().__init__()
        self.title("Layout dalam Frame")
        self.geometry("300x200")
        self.frame_atas = tk.Frame(self, bg="lightgray")
        self.frame_atas.pack(fill="x", pady=10)

        tk.Label(self.frame_atas, text="Ini bagian atas").pack(pady=5)
        self.frame_bawah = tk.Frame(self)
        self.frame_bawah.pack(pady=10)

        tk.Button(self.frame_bawah, text="Tombol 1").pack(side="left", padx=5)
        tk.Button(self.frame_bawah, text="Tombol 2").pack(side="left", padx=5)

if __name__ == "__main__":
    app = FrameDemo()
    app.mainloop()
\end{lstlisting}
\end{frame}


\begin{frame}[fragile]{Contoh Kombinasi Layout Sederhana}
\vspace{20pt}
Kombinasi layout sering digunakan untuk membuat tampilan lebih fleksibel.

\begin{itemize}
    \item \texttt{pack()} cocok untuk mengatur frame utama secara vertikal.
    \item \texttt{grid()} ideal untuk mengatur form dalam struktur baris–kolom.
    \item Kombinasi ini memberi keseimbangan antara fleksibilitas dan kemudahan.
    \item Pendekatan berbasis class membuat antarmuka modular dan mudah diperluas.
\end{itemize}
\end{frame}

\begin{frame}[fragile]{Kode Kombinasi Layout}
\vspace{20pt}
\begin{lstlisting}[style=PythonStyle, basicstyle=\ttfamily\scriptsize]
import tkinter as tk

class KombinasiLayoutDemo(tk.Tk):
	def __init__(self):
		super().__init__()
		self.title("Kombinasi Layout")
		self.geometry("300x250")
		self.frame_form = tk.Frame(self, padx=10, pady=10, relief="groove", borderwidth=2)
		self.frame_form.pack(pady=10)
		tk.Label(self.frame_form, text="Nama:").grid(row=0, column=0, padx=5, pady=5)
		tk.Entry(self.frame_form).grid(row=0, column=1, padx=5, pady=5)
		tk.Label(self.frame_form, text="Umur:").grid(row=1, column=0, padx=5, pady=5)
		tk.Entry(self.frame_form).grid(row=1, column=1, padx=5, pady=5)
		self.frame_tombol = tk.Frame(self)
		self.frame_tombol.pack(pady=10)
		tk.Button(self.frame_tombol, text="Simpan").pack(side="left", padx=5)
		tk.Button(self.frame_tombol, text="Batal").pack(side="left", padx=5)

if __name__ == "__main__":
	app = KombinasiLayoutDemo()
	app.mainloop()
\end{lstlisting}
\end{frame}

\section{Rangkuman}
\begin{frame}[fragile]{Rangkuman}
\vspace{20pt}

\begin{itemize}
    \item Tkinter adalah pustaka standar Python untuk membuat GUI dengan konsep \textit{event-driven}, memungkinkan program merespons aksi pengguna (klik, input, dll).
    \item Widget dasar seperti \texttt{Label}, \texttt{Button}, dan \texttt{Entry} digunakan untuk menampilkan informasi, menerima input, dan menjalankan perintah.
    \item Window utama dapat dibuat secara prosedural maupun dengan pendekatan OOP menggunakan class yang mewarisi \texttt{tk.Tk} sehingga struktur aplikasi lebih rapi dan modular.

    \item Tkinter menyediakan tiga metode layout:  
          \texttt{pack()} (sederhana), \texttt{grid()} (form terstruktur), dan \texttt{place()} (koordinat absolut).
    \item \texttt{Frame} membantu mengelompokkan widget sehingga antarmuka lebih terorganisasi dan mendukung \textit{nested layout}.
    \item Kombinasi layout seperti \texttt{pack()} untuk frame utama dan \texttt{grid()} untuk form memberikan fleksibilitas penyusunan tampilan.
\end{itemize}

\end{frame}


\end{document}