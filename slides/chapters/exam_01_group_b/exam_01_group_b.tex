\documentclass[aspectratio=169, table]{beamer}
\usepackage[utf8]{inputenc}
\usepackage{listings} 
\usepackage[strings]{underscore}
\usepackage{caption}
\usepackage{float}


\renewcommand{\lstlistingname}{} 

\makeatletter
\def\input@path{{../../themes/Pradita}}
\makeatother

\usetheme{Pradita}

\subtitle{IF120203-Programming Fundamentals}

\title{Chapter-07:\\\LARGE{Exam File I/O pada Python (B)\\}
\vspace{10pt}}
\date[Serial]{\scriptsize {PRU/SPMI/FR-BM-18/0222}}
\author[Pradita]{\small{\textbf{Alfa Yohannis}}}


% Define Python language style for listings
\lstdefinestyle{PythonStyle}{
    language=Python,
    basicstyle=\ttfamily\footnotesize,
    keywordstyle=\color{blue}\bfseries,
    commentstyle=\color{gray}\itshape,
    stringstyle=\color{red},
    showstringspaces=false,
    breaklines=true,
    frame=lines,
    numbers=left,
    numberstyle=\tiny\color{gray},
    backgroundcolor=\color{lightgray!10},
    tabsize=2,
    captionpos=b
}

\lstdefinelanguage{bash} {
	keywords={},
	basicstyle=\ttfamily\small,
	keywordstyle=\color{blue}\bfseries,
	ndkeywords={iex},
	ndkeywordstyle=\color{purple}\bfseries,
	sensitive=true,
	commentstyle=\color{gray},
	stringstyle=\color{red},
	numbers=left,
	numberstyle=\tiny\color{gray},
	breaklines=true,
	frame=lines,
	backgroundcolor=\color{lightgray!10},
	tabsize=2,
	comment=[l]{\#},
	morecomment=[s]{/*}{*/},
	commentstyle=\color{gray}\ttfamily,
	stringstyle=\color{purple}\ttfamily,
	showstringspaces=false,
	captionpos=b
}

\begin{document}

\frame{\titlepage}

% Add table of contents slide
\begin{frame}[fragile]{Contents}
\vspace{15pt}
\begin{columns}[t]
\begin{column}{.4\textwidth}
\tableofcontents[sections={1-4}]
\end{column}
\begin{column}{.6\textwidth}
\tableofcontents[sections={5-7}]
\end{column}
\end{columns}
\end{frame}

\section{Instruksi Umum}
\begin{frame}[fragile]{Instruksi Umum}
\begin{enumerate}
\item Selesaikan soal berikut \textbf{TANPA} bantuan \textbf{ChatGPT} atau GenAI/Large-language Model AI lainnya (misal, Gemini, Grok, Deepseek, etc.).
\item Boleh menggunakan Search Engines dan Web Browser untuk mengakses dokumentasi Python dan contoh penggunaan fungsi-fungsi.
\item \textbf{Durasi test adalah 30 menit}.
\item Kumpulkan semua jawaban (file *.py, *.txt, *.csv) ke \textcolor{blue}{\textbf{\url{https://drive.google.com/drive/folders/1S8uA1DaRdWFCoQKctazQm6wToWCOArSc?usp=drive_link}}} sesuai direktori NIM-Nama dan nomor soal.
\end{enumerate}
\end{frame}



\section{Soal 7}
%=============================
\begin{frame}[fragile]{Soal 7: Filter Skor Survei dan Kepuasan (1/3)}
\vspace{10pt}
\textbf{Tugas:}  
Buat program Python \texttt{survey\_summary.py} yang membaca dua file CSV hasil survei pelanggan (area Utara dan Selatan), menggabungkan datanya, memfilter skor tinggi, dan membuat laporan ringkasan keseluruhan.

\textbf{Langkah:}
\begin{enumerate}
  \item Baca dua file: \texttt{survey\_north.csv} dan \texttt{survey\_south.csv} menggunakan \texttt{csv.DictReader}.
  \item Gabungkan seluruh data responden ke dalam satu list.
  \item Pilih hanya responden dengan \texttt{score} lebih dari \texttt{80}.
  \item Simpan hasil filter ke \texttt{high\_scores.csv}.
  \item Dari \textbf{seluruh data (sebelum difilter)}, hitung total responden, rata-rata skor, dan wilayah dengan skor rata-rata tertinggi; simpan ke \texttt{report.txt}.
\end{enumerate}

\textbf{Tujuan:}  
Melatih pembacaan multi-file CSV, penggabungan data, filter numerik, dan pembuatan laporan statistik.
\end{frame}

%=============================
\begin{frame}[fragile]{Soal 7: Data Masukan (2/3)}
\vspace{10pt}
\textbf{File Input 1: \texttt{survey\_north.csv}}
\begin{lstlisting}[language=bash,basicstyle=\ttfamily\small]
region,respondent,score
North,Andi,75
North,Budi,82
North,Bella,95
\end{lstlisting}

\textbf{File Input 2: \texttt{survey\_south.csv}}
\begin{lstlisting}[language=bash,basicstyle=\ttfamily\small]
region,respondent,score
South,Bima,78
South,Citra,88
South,Dino,91
\end{lstlisting}

\textbf{Deskripsi:}  
Masing-masing file mewakili hasil survei dari wilayah berbeda;  
kolom mencakup wilayah, nama responden, dan skor kepuasan (0–100).
\end{frame}

%=============================
\begin{frame}[fragile]{Soal 7: Hasil Keluaran (3/3)}
\vspace{10pt}
\textbf{Output 1: \texttt{report.txt}}  
(Laporan dari seluruh data sebelum difilter)
\begin{lstlisting}[language=bash,basicstyle=\ttfamily\small]
CUSTOMER SATISFACTION REPORT
============================
Total regions: 2
Total respondents: 6
Average score: 84.83
Top region: North (84.0)
\end{lstlisting}

\textbf{Output 2: \texttt{high\_scores.csv}}  
(Hanya responden dengan skor di atas 80)
\begin{lstlisting}[language=bash,basicstyle=\ttfamily\small]
region,respondent,score
North,Budi,82
North,Bella,95
South,Citra,88
South,Dino,91
\end{lstlisting}
\end{frame}



\end{document}