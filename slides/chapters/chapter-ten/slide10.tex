\documentclass[aspectratio=169, table]{beamer}
\usepackage[utf8]{inputenc}
\usepackage{listings} 
\usepackage[strings]{underscore}
\usepackage{caption}
\usepackage{float}

\usepackage{tikz}
\usetikzlibrary{shapes.geometric, arrows.meta, trees, positioning}


\renewcommand{\lstlistingname}{} 

\makeatletter
\def\input@path{{../../themes/Pradita}}
\makeatother

\usetheme{Pradita}

\subtitle{IF120203-Programming Fundamentals}

\title{Chapter-10:\\\LARGE{Unit Test\\}
\vspace{10pt}}
\date[Serial]{\scriptsize {PRU/SPMI/FR-BM-18/0222}}
\author[Pradita]{\small{\textbf{Alfa Yohannis}}}


% Define Python language style for listings
\lstdefinestyle{PythonStyle}{
    language=Python,
    basicstyle=\ttfamily\footnotesize,
    keywordstyle=\color{blue}\bfseries,
    commentstyle=\color{gray}\itshape,
    stringstyle=\color{red},
    showstringspaces=false,
    breaklines=true,
    frame=lines,
    numbers=left,
    numberstyle=\tiny\color{gray},
    backgroundcolor=\color{lightgray!10},
    tabsize=2,
    captionpos=b
}

\lstdefinelanguage{bash} {
	keywords={},
	basicstyle=\ttfamily\small,
	keywordstyle=\color{blue}\bfseries,
	ndkeywords={iex},
	ndkeywordstyle=\color{purple}\bfseries,
	sensitive=true,
	commentstyle=\color{gray},
	stringstyle=\color{red},
	numbers=left,
	numberstyle=\tiny\color{gray},
	breaklines=true,
	frame=lines,
	backgroundcolor=\color{lightgray!10},
	tabsize=2,
	comment=[l]{\#},
	morecomment=[s]{/*}{*/},
	commentstyle=\color{gray}\ttfamily,
	stringstyle=\color{purple}\ttfamily,
	showstringspaces=false,
	captionpos=b
}

\begin{document}

\frame{\titlepage}

% Add table of contents slide
\begin{frame}[fragile]{Contents}
\vspace{15pt}
\begin{columns}[t]
\begin{column}{.4\textwidth}
\tableofcontents[sections={1-4}]
\end{column}
\begin{column}{.6\textwidth}
\tableofcontents[sections={5-7}]
\end{column}
\end{columns}
\end{frame}


%----------------------------------------
\section{Pendahuluan}

\begin{frame}[fragile]{Pendahuluan}
\vspace*{15pt}
\begin{itemize}
  \item \textbf{Unit testing} adalah proses menguji bagian kecil program seperti fungsi atau method.
  \item Tujuannya memastikan setiap unit bekerja sesuai yang diharapkan.
  \item Pengujian sejak awal membantu menemukan kesalahan logika lebih cepat.
  \item Dengan demikian, pengembang dapat memperbaiki bug sebelum kode digabungkan dengan komponen lain yang lebih kompleks.
\end{itemize}
\end{frame}

%----------------------------------------
\subsection{Mengapa Unit Test Penting}
\begin{frame}[fragile]{Mengapa Unit Test Penting}
\vspace*{10pt}
\begin{enumerate}
  \item \textbf{Deteksi dini kesalahan.} Bug ditemukan sebelum integrasi.
  \item \textbf{Refactoring aman.} Menjamin perilaku lama tetap konsisten.
  \item \textbf{Dokumentasi fungsional.} Menunjukkan cara fungsi bekerja.
  \item \textbf{Efisiensi jangka panjang.} Menghemat waktu debugging.
  \item \textbf{Dasar pengujian lanjutan.} Pondasi untuk integrasi dan end-to-end test.
\end{enumerate}
\end{frame}

%----------------------------------------
\subsection{Prinsip Dasar Pengujian Unit}

\begin{frame}[fragile]{Prinsip Dasar Pengujian Unit}
\vspace*{10pt}
\begin{itemize}
  \item Pengujian unit mengikuti prinsip agar test efektif dan mudah dirawat.
  \item \textbf{Isolasi:} setiap test menguji satu fungsi tanpa bergantung pada test lain.
  \item \textbf{Deterministik:} hasil harus sama setiap kali dijalankan.
  \item \textbf{Kemandirian:} test tidak bergantung pada database, jaringan, atau file eksternal kecuali diperlukan.
\end{itemize}
\end{frame}

%----------------------------------------
\section{Pengenalan Modul \texttt{unittest}}
\begin{frame}[fragile]{Pengenalan Modul \texttt{unittest}}
\vspace*{10pt}
\begin{itemize}
  \item Python menyediakan modul bawaan \texttt{unittest} untuk menulis dan menjalankan pengujian unit.
  \item Terinspirasi dari kerangka kerja \textit{xUnit} seperti \texttt{JUnit} dan \texttt{NUnit}.
  \item Fitur utama:
  \begin{itemize}
    \item Mendefinisikan \textit{test case}.
    \item Melakukan pengecekan hasil dengan \textit{assertion}.
    \item Mengelompokkan test menjadi \textit{test suite}.
    \item Menjalankan semua test secara otomatis.
  \end{itemize}
\end{itemize}
\end{frame}

%----------------------------------------
\subsection{Struktur Dasar \texttt{TestCase}}

\begin{frame}[fragile]{Struktur Dasar \texttt{TestCase}}
\vspace{20pt}
\begin{itemize}
  \item Setiap test ditulis sebagai kelas turunan dari \texttt{unittest.TestCase}.
  \item Metode yang diawali \texttt{test\_} akan dijalankan otomatis.
\end{itemize}

\begin{lstlisting}[style=PythonStyle, caption={Blok eksekusi test bawaan}]
if __name__ == "__main__":
    unittest.main()
\end{lstlisting}
\vspace{-10pt}
\begin{lstlisting}[style=PythonStyle, caption={Struktur dasar kelas unit test di Python}]
import unittest

class TestNamaFungsi(unittest.TestCase):
    def test_kasus_1(self):
        self.assertEqual(1 + 1, 2)

if __name__ == "__main__":
    unittest.main()
\end{lstlisting}
\end{frame}

%----------------------------------------
\begin{frame}[fragile]{Eksekusi dan Hasil Test}
\vspace*{10pt}
\begin{itemize}
  \item Saat dijalankan, \texttt{unittest} mengeksekusi semua metode \texttt{test\_}.
  \item Hasilnya ditandai dengan:
  \begin{itemize}
    \item \texttt{.} — test berhasil.
    \item \texttt{F} — test gagal.
    \item \texttt{E} — error selama eksekusi.
  \end{itemize}
\end{itemize}

\begin{lstlisting}[language=bash]
test_kasus_1 (test_fungsi.TestNamaFungsi) ... ok
------------------------------------------------------------------
Ran 1 test in 0.001s
OK
\end{lstlisting}
\end{frame}

%----------------------------------------
\subsection{Menjalankan Test}
\begin{frame}[fragile]{Menjalankan Test}
\vspace*{10pt}
\begin{itemize}
  \item Jalankan langsung:
  \begin{lstlisting}[language=bash]
  python test_fungsi.py
  \end{lstlisting}
  \item Gunakan perintah bawaan:
  \begin{lstlisting}[language=bash]
  python -m unittest discover
  \end{lstlisting}
  \item Jalankan test tertentu:
  \begin{lstlisting}[language=bash]
  python -m unittest test_fungsi.TestNamaFungsi.test_kasus_1
  \end{lstlisting}
  \item Tambah keluaran rinci:
  \begin{lstlisting}[language=bash]
  python -m unittest -v
  \end{lstlisting}
\end{itemize}
\end{frame}

%----------------------------------------
\section{Menulis Unit Test untuk Fungsi}

\begin{frame}[fragile]{Menulis Unit Test untuk Fungsi}
\vspace*{10pt}
\begin{itemize}
  \item Unit test untuk fungsi adalah bentuk pengujian paling sederhana.
  \item Fokus pada verifikasi input dan output yang deterministik.
  \item Langkah dasar:
  \begin{enumerate}
    \item Tulis fungsi yang akan diuji.
    \item Buat berkas test terpisah (\texttt{test\_fungsi.py}).
    \item Import fungsi dari modul utama.
    \item Buat kelas turunan \texttt{unittest.TestCase}.
    \item Gunakan metode \texttt{assert*} untuk membandingkan hasil.
  \end{enumerate}
\end{itemize}
\end{frame}

%----------------------------------------
\begin{frame}[fragile]{Contoh Fungsi dan Unit Test}
\vspace*{10pt}
\begin{lstlisting}[style=PythonStyle, caption={Fungsi yang akan diuji}]
# file: fungsi.py
def add(a, b):
    """Menjumlahkan dua bilangan."""
    return a + b
\end{lstlisting}

\begin{lstlisting}[style=PythonStyle, caption={Pengujian fungsi add()}]
# file: test_fungsi.py
import unittest
from fungsi import add

class TestAdd(unittest.TestCase):
    def test_positif(self):
        self.assertEqual(add(2, 3), 5)
\end{lstlisting}
\end{frame}

%----------------------------------------
\begin{frame}[fragile]{Menjalankan dan Memeriksa Hasil}
\vspace*{10pt}
\begin{lstlisting}[language=bash]
python -m unittest test_fungsi.py
\end{lstlisting}

\begin{lstlisting}[language=bash]
...
------------------------------------------------------------------
Ran 3 tests in 0.001s

OK
\end{lstlisting}

\begin{itemize}
  \item Simbol hasil:
  \begin{itemize}
    \item \texttt{.} test sukses.
    \item \texttt{F} test gagal.
    \item \texttt{E} terjadi error.
  \end{itemize}
  \item Pengujian dapat dijalankan berulang tanpa dependensi eksternal.
\end{itemize}
\end{frame}

%----------------------------------------
\section{Menulis Unit Test untuk Method di Kelas}

\begin{frame}[fragile]{Menguji Method di Kelas}
\vspace*{10pt}
\begin{itemize}
  \item Pengujian method mirip dengan fungsi, tetapi melibatkan \textit{state} objek.
  \item Langkah umum:
  \begin{enumerate}
    \item Buat instance objek pada awal test.
    \item Panggil method yang diuji dengan argumen relevan.
    \item Gunakan \texttt{assert*} untuk memeriksa hasil.
    \item Gunakan \texttt{assertRaises} bila perlu menguji error.
  \end{enumerate}
  \item Untuk \texttt{@staticmethod} dan \texttt{@classmethod}, panggil dengan \texttt{NamaKelas.method()}.
\end{itemize}
\end{frame}

%----------------------------------------
\begin{frame}[fragile]{Menggunakan \texttt{setUp} dan \texttt{tearDown}}
\vspace*{10pt}
\begin{itemize}
  \item \texttt{setUp()} dijalankan sebelum setiap test case untuk menyiapkan objek atau data.
  \item \texttt{tearDown()} dijalankan sesudah test untuk membersihkan sumber daya.
  \item Kedua metode menjamin setiap test berjalan dalam kondisi segar dan terisolasi.
\end{itemize}

\begin{lstlisting}[style=PythonStyle, caption={Contoh penggunaan setUp/tearDown}]
import unittest

class TestUser(unittest.TestCase):
    def setUp(self):
        self.user = User("Alfa")

    def tearDown(self):
        del self.user
\end{lstlisting}
\end{frame}


%----------------------------------------
\begin{frame}[fragile]{Kode Kelas Contoh untuk Diuji (1/5)}
\vspace*{20pt}
\begin{lstlisting}[style=PythonStyle]
# file: calculator.py

class Calculator:
    def __init__(self, memory: float = 0.0):
        # state internal: menyimpan nilai terakhir
        self.memory = float(memory)

    # -------- Instance methods --------
    def add(self, a: float, b: float) -> float:
        """Menjumlahkan a dan b, memperbarui memory, 
        dan mengembalikan hasil."""
        result = float(a) + float(b)
        self.memory = result
        return result
\end{lstlisting}
\end{frame}

%----------------------------------------
\begin{frame}[fragile]{Kode Kelas Contoh untuk Diuji (2/5)}
\vspace*{20pt}
\begin{lstlisting}[style=PythonStyle]
    def divide(self, a: float, b: float) -> float:
        """Membagi a dengan b. 
        Memunculkan ZeroDivisionError jika b == 0."""
        if b == 0:
            raise ZeroDivisionError("pembagian dengan nol")
        result = float(a) / float(b)
        self.memory = result
        return result

    # -------- Static method --------
    @staticmethod
    def is_even(n: int) -> bool:
        """Mengembalikan True jika n genap, False jika ganjil."""
        return (n % 2) == 0
\end{lstlisting}
\end{frame}

%----------------------------------------
\begin{frame}[fragile]{Kode Kelas Contoh untuk Diuji (3/5)}
\vspace*{20pt}
\begin{lstlisting}[style=PythonStyle]
    # -------- Class method --------
    @classmethod
    def from_string(cls, s: str) -> "Calculator":
        """
        Alternate constructor: membuat Calculator 
        dari string angka. Jika string tidak valid, 
        ValueError.
        """
        try:
            val = float(s.strip())
        except Exception as e:
            raise ValueError(f"nilai tidak valid: {s}") from e
        return cls(memory=val)
\end{lstlisting}
\end{frame}

%----------------------------------------
\begin{frame}[fragile]{Pengujian Method Kelas (1/5)}
\vspace*{20pt}
\begin{lstlisting}[style=PythonStyle]
# file: test_calculator.py
import unittest
from calculator import Calculator

class TestCalculatorMethods(unittest.TestCase):
    def setUp(self):
        self.calc = Calculator()

    def tearDown(self):
        self.calc = None

    def test_add_mengembalikan_hasil_dan_update_memory(self):
        hasil = self.calc.add(2, 3)
        self.assertEqual(hasil, 5.0)
\end{lstlisting}
\end{frame}

%----------------------------------------
\begin{frame}[fragile]{Pengujian Method Kelas (2/5)}
\vspace*{20pt}
\begin{lstlisting}[style=PythonStyle]
        self.assertEqual(self.calc.memory, 5.0)

    def test_divide_zero_raises(self):
        with self.assertRaises(ZeroDivisionError):
            self.calc.divide(1, 0)

    def test_add_dengan_variansi_input(self):
        kasus = [
            (0, 0, 0.0),
            (1, -1, 0.0),
            (2.5, 0.5, 3.0),
            (-3, -7, -10.0),
        ]
        for a, b, expected in kasus:
            with self.subTest(a=a, b=b):
\end{lstlisting}
\end{frame}

%----------------------------------------
\begin{frame}[fragile]{Pengujian Method Kelas (3/5)}
\vspace*{20pt}
\begin{lstlisting}[style=PythonStyle]
                self.assertEqual(self.calc.add(a, b), expected)

    def test_is_even(self):
        # Bisa dipanggil via kelas atau instance
        self.assertTrue(Calculator.is_even(2))
        self.assertFalse(self.calc.is_even(3))

    def test_from_string_valid(self):
        c = Calculator.from_string("  42.5 ")
        self.assertIsInstance(c, Calculator)
        self.assertEqual(c.memory, 42.5)

    def test_from_string_invalid_raises(self):
        with self.assertRaises(ValueError):
\end{lstlisting}
\end{frame}

%----------------------------------------
\begin{frame}[fragile]{Pengujian Method Kelas (4/5)}
\vspace*{20pt}
\begin{lstlisting}[style=PythonStyle]
            Calculator.from_string("bukan-angka")

if __name__ == "__main__":
    unittest.main()
\end{lstlisting}
\end{frame}

%----------------------------------------
\begin{frame}[fragile]{Menjalankan Pengujian Kelas}
\vspace*{20pt}
\begin{lstlisting}[language=bash]
python -m unittest test_calculator.py
\end{lstlisting}
\begin{lstlisting}[language=bash, caption={Keluaran yang diharapkan (ilustrasi)}]
test_add_dengan_variansi_input ... ok
test_add_mengembalikan_hasil_dan_update_memory ... ok
test_divide_normal ... ok
test_divide_zero_raises ... ok
test_from_string_invalid_raises ... ok
test_from_string_valid ... ok
test_is_even ... ok
-----------------------------------------------------
Ran 7 tests in 0.00Xs
OK
\end{lstlisting}
\end{frame}

%----------------------------------------
\begin{frame}[fragile]{Penekanan Utama dari Contoh}
\vspace*{20pt}
\begin{itemize}
  \item \textbf{Instance method}: diuji untuk hasil benar dan skenario error menggunakan \texttt{assertRaises}.
  \item \textbf{Static method}: diuji seperti fungsi murni tanpa bergantung pada state objek.
  \item \textbf{Class method}: diuji sebagai \textit{alternate constructor} untuk memvalidasi input.
  \item \textbf{\texttt{setUp}/\texttt{tearDown}}: memastikan setiap test berjalan di lingkungan bersih.
  \item \textbf{\texttt{subTest}}: menguji banyak variasi input tanpa membuat test terpisah.
\end{itemize}
\end{frame}


%----------------------------------------
\section{Assertion Dasar dalam \texttt{unittest}}

\begin{frame}[fragile]{Konsep Dasar Assertion}
\vspace*{20pt}
\begin{itemize}
  \item Pengujian dilakukan dengan memeriksa apakah hasil aktual sesuai dengan ekspektasi.
  \item Pemeriksaan dilakukan menggunakan metode \textit{assertion} dari \texttt{unittest.TestCase}.
  \item Jika kondisi tidak terpenuhi, test akan berstatus \textit{failed}.
  \item Assertion berfungsi sebagai “kontrak” yang harus selalu valid bagi fungsi yang diuji.
\end{itemize}

\begin{lstlisting}[style=PythonStyle]
self.assertEqual(a, b)
self.assertTrue(expr)
self.assertFalse(expr)
\end{lstlisting}
\end{frame}

%----------------------------------------
\begin{frame}[fragile]{Contoh Penggunaan Assertion Dasar}
\vspace*{20pt}
\begin{lstlisting}[style=PythonStyle]
import unittest

def is_even(n): return (n % 2) == 0
def square(x): return x * x

class TestAssertionDasar(unittest.TestCase):
    def test_assert_equal(self):
        self.assertEqual(square(3), 9)
        self.assertEqual(square(-4), 16)

    def test_assert_true_false(self):
        self.assertTrue(is_even(2))
        self.assertFalse(is_even(3))

if __name__ == "__main__":
    unittest.main()
\end{lstlisting}
\end{frame}

%----------------------------------------
\begin{frame}[fragile]{Contoh Hasil dan Variasi Assertion}
\vspace*{20pt}
\begin{lstlisting}[language=bash]
...
--------------------------------------------------------------
Ran 2 tests in 0.000s

OK
\end{lstlisting}

\begin{itemize}
  \item \texttt{assertNotEqual(a, b)}
  \item \texttt{assertIsNone(x)}, \texttt{assertIsNotNone(x)}
  \item \texttt{assertIn(member, container)}, \texttt{assertNotIn(member, container)}
  \item \texttt{assertAlmostEqual(a, b, places=n)}
\end{itemize}
\end{frame}

%----------------------------------------
\begin{frame}[fragile]{Menggunakan \texttt{assertRaises}}
\vspace*{20pt}
\begin{lstlisting}[style=PythonStyle]
import unittest

def divide(a, b):
    if b == 0:
        raise ZeroDivisionError("pembagian dengan nol")
    return a / b

class TestErrorHandling(unittest.TestCase):
    def test_divide_normal(self):
        self.assertEqual(divide(10, 2), 5)

    def test_divide_zero(self):
        with self.assertRaises(ZeroDivisionError):
            divide(4, 0)
\end{lstlisting}
\end{frame}

%----------------------------------------
\begin{frame}[fragile]{assertRaises (lanjutan)}
\vspace*{20pt}
\begin{lstlisting}[style=PythonStyle]
    def test_divide_zero_pakai_lambda(self):
        self.assertRaises(ZeroDivisionError, divide, 1, 0)

if __name__ == "__main__":
    unittest.main()
\end{lstlisting}

\begin{lstlisting}[language=bash]
...
----------------------------------------------------------------------
Ran 3 tests in 0.001s

OK
\end{lstlisting}

\texttt{assertRaises} memastikan error dilempar dengan benar 
pada kondisi yang salah — menguji skenario negatif secara eksplisit.
\end{frame}

%----------------------------------------
\begin{frame}[fragile]{\texttt{subTest} untuk Menguji Beberapa Kasus}
\vspace*{20pt}
\begin{lstlisting}[style=PythonStyle]
import unittest

def multiply(a, b):
    return a * b

class TestSubTest(unittest.TestCase):
    def test_multiply_variatif(self):
        kasus = [ (2, 3, 6),
            (0, 10, 0),
            (-2, 4, -8),
            (1.5, 2, 3.0),
        ]
        for a, b, expected in kasus:
            with self.subTest(a=a, b=b):
                self.assertEqual(multiply(a, b), expected)

if __name__ == "__main__":
    unittest.main()
\end{lstlisting}
\end{frame}

%----------------------------------------
\begin{frame}[fragile]{Hasil Eksekusi dan Manfaat \texttt{subTest}}
\vspace*{20pt}
\begin{lstlisting}[language=bash]
...
--------------------------------------------------------------
Ran 1 test in 0.000s

OK
\end{lstlisting}

\begin{lstlisting}[language=bash]
FAIL: test_multiply_variatif (__main__.TestSubTest) (a=2, b=3)
AssertionError: 5 != 6
\end{lstlisting}

\begin{itemize}
  \item Semua kasus tetap dijalankan meskipun satu gagal.
  \item Laporan menunjukkan kombinasi input yang menyebabkan error.
  \item Menghemat waktu, meningkatkan keterbacaan, 
        dan menghindari duplikasi kode.
\end{itemize}
\end{frame}

%----------------------------------------
\subsection{Menjalankan via CLI: \texttt{python -m unittest}}

\begin{frame}[fragile]{Menjalankan Unit Test via CLI}
\vspace*{20pt}
\begin{lstlisting}[language=bash, caption={Menjalankan file test tertentu}]
python -m unittest test_fungsi.py
\end{lstlisting}

\begin{itemize}
  \item Menjalankan semua test dalam satu berkas.
  \item Hanya metode yang diawali \texttt{test\_} yang akan dijalankan.
  \item Untuk menjalankan semua test di proyek:
\end{itemize}

\begin{lstlisting}[language=bash, caption={Menjalankan semua test (discovery)}]
python -m unittest discover
\end{lstlisting}

\begin{itemize}
  \item Mencari file \texttt{test\_*.py} atau \texttt{*\_test.py} 
        di seluruh direktori proyek.
\end{itemize}
\end{frame}

%----------------------------------------
\begin{frame}[fragile]{Menjalankan Test Spesifik dan Verbose}
\vspace*{20pt}
\begin{lstlisting}[language=bash, caption={Menjalankan test tertentu}]
python -m unittest test_fungsi.TestAddFunction.test_penjumlahan_positif
\end{lstlisting}

\begin{lstlisting}[language=bash, caption={Mode verbose}]
python -m unittest -v
\end{lstlisting}

\begin{lstlisting}[language=bash, caption={Beberapa modul sekaligus}]
python -m unittest test_fungsi test_calculator
\end{lstlisting}

\begin{lstlisting}[language=bash, caption={Contoh hasil verbose}]
test_penjumlahan_positif ... ok
test_penjumlahan_negatif ... ok
test_penjumlahan_nol ... ok
------------------------------------------------------------------
Ran 3 tests in 0.001s
OK
\end{lstlisting}
\end{frame}

%----------------------------------------
\begin{frame}[fragile]{Interpretasi Hasil dan Mode Ringkas}
\vspace*{20pt}
\begin{itemize}
  \item \texttt{ok} $\rightarrow$ Test berhasil.
  \item \texttt{FAIL} $\rightarrow$ Hasil tidak sesuai ekspektasi.
  \item \texttt{ERROR} $\rightarrow$ Error tak tertangani.
  \item texttt{.} $\rightarrow$ Test berhasil (passed), \texttt{F} $\rightarrow$ Test gagal (failed), \texttt{E} $\rightarrow$ Test error (exception) \\
\end{itemize}

\begin{lstlisting}[language=bash]
..F
==================================================================
FAIL: test_penjumlahan_nol (test_fungsi.TestAddFunction)
------------------------------------------------------------------
AssertionError: 0 != 1
------------------------------------------------------------------
Ran 3 tests in 0.000s
FAILED (failures=1)
\end{lstlisting}
\end{frame}

%----------------------------------------
\subsection{Menafsirkan Output dan Traceback}

\begin{frame}[fragile]{Menafsirkan Output dan Traceback}
\vspace*{20pt}
\begin{itemize}
  \item Jika test gagal, \texttt{unittest} menampilkan laporan \textit{traceback}.
  \item Struktur laporan:
  \begin{itemize}
    \item \textbf{Nama test gagal:} contoh  
      \texttt{FAIL: test\_penjumlahan\_nol (TestAddFunction)}.
    \item \textbf{Lokasi kesalahan:} menunjukkan file dan nomor baris.
    \item \textbf{Assertion gagal:} menampilkan nilai aktual dan ekspektasi.
    \item \textbf{Ringkasan hasil:} misalnya  
      \texttt{FAILED (failures=1)}.
  \end{itemize}
\end{itemize}

\begin{lstlisting}[language=bash]
FAILED (failures=1)
\end{lstlisting}
\end{frame}

%----------------------------------------
\begin{frame}[fragile]{Contoh Traceback pada ERROR}
\vspace*{20pt}
\begin{lstlisting}[language=bash]
E
==================================================================
ERROR: test_divide_zero (test_calculator.TestCalculatorMethods)
------------------------------------------------------------------
Traceback (most recent call last):
  File "test_calculator.py", line 17, in test_divide_zero
    self.calc.divide(1, 0)
  File "calculator.py", line 12, in divide
    raise ZeroDivisionError("pembagian dengan nol")
ZeroDivisionError: pembagian dengan nol
------------------------------------------------------------------
Ran 1 test in 0.000s
FAILED (errors=1)
\end{lstlisting}
\end{frame}

%----------------------------------------
\begin{frame}[fragile]{Perbedaan dan Tips Membaca Traceback}
\vspace*{20pt}
\begin{itemize}
  \item \textbf{FAIL}: assertion gagal (nilai tidak sesuai ekspektasi).
  \item \textbf{ERROR}: exception tak tertangani di luar assertion.
\end{itemize}

\textbf{Tips membaca traceback:}
\begin{enumerate}
  \item Fokus pada baris paling bawah untuk pesan error utama.
  \item Gunakan nomor baris untuk menuju sumber kesalahan.
  \item Tambahkan pesan kustom untuk memperjelas test:
\end{enumerate}

\begin{lstlisting}[style=PythonStyle]
self.assertEqual(add(2, 3), 5, 
    "Fungsi add tidak menjumlahkan dengan benar")
\end{lstlisting}
\end{frame}


%----------------------------------------
\section{Refactoring dan Pemeliharaan Test}

\begin{frame}[fragile]{Konsep Refactoring dan Pemeliharaan Test}
\vspace*{20pt}
\begin{itemize}
  \item Setelah semua unit test berjalan baik, langkah selanjutnya adalah menjaga agar kode tetap mudah dibaca dan dirawat.
  \item \textbf{Refactoring}: memperbaiki struktur internal kode tanpa mengubah perilaku eksternal.
  \item Unit test berfungsi sebagai \textbf{jaring pengaman} untuk mendeteksi efek samping perubahan.
  \item Jika seluruh test tetap \texttt{OK}, maka refactoring berhasil tanpa menimbulkan regresi.
\end{itemize}
\end{frame}

%----------------------------------------
\begin{frame}[fragile]{Menambah Fungsi Baru}
\vspace*{20pt}
\begin{itemize}
  \item Setiap fitur baru perlu diikuti dengan test baru agar cakupan pengujian meningkat.
  \item Contoh: menambahkan fungsi \texttt{subtract(a, b)} di file \texttt{fungsi.py}.
\end{itemize}

\begin{lstlisting}[style=PythonStyle, caption={Fungsi baru: subtract()}]
# file: fungsi.py
def add(a, b):
    return a + b

def subtract(a, b):
    """Mengembalikan hasil pengurangan a - b."""
    return a - b
\end{lstlisting}
\end{frame}

%----------------------------------------
\begin{frame}[fragile]{Menambahkan Test untuk Fungsi Baru}
\vspace*{20pt}
\begin{lstlisting}[style=PythonStyle, caption={Test untuk fungsi add() dan subtract()}]
# file: test_fungsi.py
import unittest
from fungsi import add, subtract

class TestMathFunctions(unittest.TestCase):
    def test_add(self):
        self.assertEqual(add(2, 3), 5)

    def test_subtract(self):
        self.assertEqual(subtract(5, 3), 2)
        self.assertEqual(subtract(0, 7), -7)
        self.assertEqual(subtract(-2, -3), 1)

if __name__ == "__main__":
    unittest.main()
\end{lstlisting}
\end{frame}

%----------------------------------------
\begin{frame}[fragile]{Memastikan Tidak Ada Regresi}
\vspace*{20pt}
\begin{lstlisting}[language=bash, caption={Menjalankan ulang semua test}]
python -m unittest -v
\end{lstlisting}

\begin{lstlisting}[language=bash, caption={Hasil keluaran}]
test_add (test_fungsi.TestMathFunctions) ... ok
test_subtract (test_fungsi.TestMathFunctions) ... ok
------------------------------------------------------------------
Ran 2 tests in 0.000s
OK
\end{lstlisting}

\begin{itemize}
  \item Setiap perubahan pada kode diuji ulang secara otomatis.
  \item Jika fungsi \texttt{subtract} diubah secara salah, test langsung gagal.
  \item Suite test menjadi dokumentasi perilaku sistem yang selalu terkini.
\end{itemize}
\end{frame}

%----------------------------------------
\begin{frame}[fragile]{Menjaga Keterbacaan dan Konsistensi}
\vspace*{20pt}
\begin{itemize}
  \item Jumlah test bisa mencapai ratusan atau ribuan seiring pertumbuhan proyek.
  \item Beberapa praktik terbaik:
  \begin{itemize}
    \item Gunakan nama test yang deskriptif:
\begin{lstlisting}[style=PythonStyle]
def test_divide_munculkan_error_jika_bagi_nol(self):
    ...
\end{lstlisting}
    \item Pisahkan test per modul atau fitur:
\begin{lstlisting}[language=bash]
project/
├── app/
│   ├── fungsi.py
│   └── kalkulator.py
└── tests/
    ├── test_fungsi.py
    └── test_kalkulator.py
\end{lstlisting}
  \end{itemize}
\end{itemize}
\end{frame}

%----------------------------------------
\begin{frame}[fragile]{Praktik Terbaik Lainnya}
\vspace*{20pt}
\begin{itemize}
  \item Hindari duplikasi kode test — gunakan \texttt{setUp()} atau helper.
  \item Gunakan data deterministik (hindari random tanpa seed).
  \item Komentar hanya jika perlu, jelaskan “mengapa” bukan “bagaimana”.
  \item Jalankan semua test sebelum commit/push agar bug tidak masuk ke main branch.
\end{itemize}

\begin{lstlisting}[style=PythonStyle, caption={Refactoring aman pada fungsi multiply()}]
def multiply(a, b):
    result = 0
    for _ in range(b):
        result += a
    return result

# Refactoring lebih ringkas
def multiply(a, b):
    return a * b
\end{lstlisting}
\end{frame}

%----------------------------------------
\begin{frame}[fragile]{Menjalankan Test dan Ringkasan}
\vspace*{20pt}
\begin{lstlisting}[language=bash, caption={Menjalankan ulang semua test}]
python -m unittest
\end{lstlisting}

\begin{itemize}
  \item Jika seluruh test tetap \texttt{OK}, refactoring aman diterapkan.
  \item Tambahkan test baru untuk fitur baru.
  \item Gunakan nama test yang jelas dan struktur folder rapi.
  \item Jalankan semua test setiap kali ada perubahan.
  \item Test berfungsi sebagai jaminan bahwa perubahan tidak mengubah perilaku aplikasi.
\end{itemize}
\end{frame}

%----------------------------------------
\section{Rangkuman}

\begin{frame}[fragile]{Rangkuman}
\vspace*{20pt}
\begin{itemize}
  \item \textbf{Konsep:} Unit test memastikan setiap fungsi/method bekerja sesuai ekspektasi.
  \item \textbf{Prinsip:} isolasi, deterministik, kemandirian, repeatable, FAST.
  \item \textbf{unittest:} struktur \texttt{TestCase}, \texttt{setUp}/\texttt{tearDown}, \texttt{test\_*.}
  \item \textbf{Assertion:} \texttt{assertEqual/True/False}, \texttt{assertRaises}, \texttt{subTest}.
  \item \textbf{CLI:} \texttt{python -m unittest}, \texttt{-v}, \texttt{discover}, jalankan test spesifik.
  \item \textbf{Output:} simbol \texttt{.}/\texttt{F}/\texttt{E}, interpretasi hasil dan traceback.
  \item \textbf{Praktik:} pisahkan test per modul, nama deskriptif, data deterministik, hindari duplikasi.
  \item \textbf{Refactoring:} jalankan semua test untuk mencegah regresi dan menjaga kualitas.
\end{itemize}
\end{frame}

\end{document}