\documentclass[aspectratio=169, table]{beamer}
\usepackage[utf8]{inputenc}
\usepackage{listings} 
\usepackage[strings]{underscore}
\usepackage{caption}
\usepackage{float}


\renewcommand{\lstlistingname}{} 

\makeatletter
\def\input@path{{../../themes/Pradita}}
\makeatother

\usetheme{Pradita}

\subtitle{IF120203-Programming Fundamentals}

\title{Chapter-06:\\\LARGE{Struktur Data\\}
\vspace{10pt}}
\date[Serial]{\scriptsize {PRU/SPMI/FR-BM-18/0222}}
\author[Pradita]{\small{\textbf{Alfa Yohannis}}}


% Define Python language style for listings
\lstdefinestyle{PythonStyle}{
language=Python,
basicstyle=\ttfamily\footnotesize,
keywordstyle=\color{blue},
commentstyle=\color{gray},
stringstyle=\color{red},
breaklines=true,
showstringspaces=false,
tabsize=2,
captionpos=b,
numbers=left,
numberstyle=\tiny\color{gray},
comment=[l]{//},
morecomment=[s]{/*}{*/},
commentstyle=\color{gray}\ttfamily,
string=[s]{'}{'},
morestring=[s]{"}{"},
}

\begin{document}

\frame{\titlepage}

% Add table of contents slide
\begin{frame}[fragile]{Contents}
\vspace{15pt}
\begin{columns}[t]
\begin{column}{.4\textwidth}
\tableofcontents[sections={1-7}]
\end{column}
\begin{column}{.6\textwidth}
\tableofcontents[sections={8-14}]
\end{column}
\end{columns}
\end{frame}


\section{Pendahuluan}


\begin{frame}[fragile]{Pendahuluan}
\vspace{20pt}

\begin{itemize}
  \item Struktur data membantu menyimpan dan mengelola informasi secara efisien. 
  \item Python memiliki beberapa struktur data bawaan seperti 
        \texttt{list}, \texttt{tuple}, \texttt{dictionary}, dan \texttt{set}. 
  \item Pemahaman terhadap struktur data penting sebagai dasar pengolahan data 
        dan pengembangan algoritma.
	\begin{lstlisting}[style=PythonStyle]
data = [10, 20, 30]
print(data[0])
\end{lstlisting}
\end{itemize}


\end{frame}


\section{List}

\begin{frame}[fragile]{List: Konsep dan Contoh}
\vspace{20pt}

\begin{columns}[T]
  % ==== KIRI ====
  \begin{column}{0.48\textwidth}
    \begin{itemize}
      \item List adalah struktur data berurutan (\textit{sequential}) dan bersifat \textbf{mutable}.
      \item Dapat menyimpan kumpulan data sejenis maupun campuran.
      \item Elemen dapat \textbf{ditambah}, \textbf{dibaca}, \textbf{diubah}, atau \textbf{dihapus}.
      \item Cocok untuk data yang bersifat dinamis, seperti daftar nama atau nilai.
    \end{itemize}
  \end{column}

  % ==== KANAN ====
  \begin{column}{0.48\textwidth}
    \begin{lstlisting}[style=PythonStyle]
buah = ["apel", "jeruk", "mangga"]

print("Daftar buah:", buah)
print("Buah pertama:", buah[0])

buah[1] = "anggur"
buah.append("pisang")
buah.remove("mangga")

print("Setelah diubah:", buah)
    \end{lstlisting}
  \end{column}
\end{columns}

\end{frame}

\section{Contoh List}

\begin{frame}[fragile]{Menghitung Total Penjualan Harian}
\vspace{20pt}

Program ini menghitung total penjualan dari daftar transaksi harian menggunakan satu tingkat perulangan.

\begin{columns}[T]
  % ==== KIRI ====
  \begin{column}{0.48\textwidth}
    \begin{itemize}
      \item Data penjualan disimpan dalam \texttt{list}.
      \item Setiap elemen mewakili nilai transaksi harian.
      \item Perulangan menjumlahkan semua nilai untuk mendapatkan total penjualan.
    \end{itemize}
  \end{column}

  % ==== KANAN ====
  \begin{column}{0.5\textwidth}
    \begin{lstlisting}[style=PythonStyle, numbers=left, firstnumber=1]
penjualan = [120000, 85000, 95000, 110000, 130000]

total = 0
for p in penjualan:
    total += p

print("Total penjualan minggu ini:",
      f"Rp{total:,}")
    \end{lstlisting}
  \end{column}
\end{columns}

\end{frame}


\section{Tuple}

\begin{frame}[fragile]{Tuple: Konsep dan Contoh}
\vspace{20pt}

\begin{columns}[T]
  % ==== KIRI ====
  \begin{column}{0.48\textwidth}
    \begin{itemize}
      \item \textbf{Tuple} mirip dengan \texttt{list}, tetapi bersifat \textit{immutable}.
      \item Elemen di dalamnya tidak dapat diubah setelah dibuat.
      \item Cocok untuk data yang bersifat tetap, seperti koordinat, ukuran, atau warna RGB.
      \item Jika perlu mengubah nilai, buat tuple baru berdasarkan tuple lama.
    \end{itemize}
  \end{column}

  % ==== KANAN ====
  \begin{column}{0.48\textwidth}
    \begin{lstlisting}[style=PythonStyle]
koordinat = (10, 20)
print("Koordinat awal:", koordinat)
print("Nilai x:", koordinat[0])

koordinat_baru = (koordinat[0], 25)
print("Koordinat baru:", koordinat_baru)
    \end{lstlisting}
  \end{column}
\end{columns}

\end{frame}


\begin{frame}[fragile]{Koordinat Lokasi dengan Tuple}
\vspace{20pt}

Program ini menyimpan dan menampilkan koordinat lokasi yang bersifat tetap menggunakan \texttt{tuple}.

\begin{columns}[T]
  % ==== KIRI (40%) ====
  \begin{column}{0.4\textwidth}
    \begin{itemize}
      \item \texttt{Tuple} digunakan untuk menyimpan data yang tidak berubah, seperti koordinat.
      \item Nilai di dalam tuple bersifat tetap selama program berjalan.
      \item Cocok untuk posisi objek, ukuran gambar, atau konfigurasi awal.
    \end{itemize}
  \end{column}

  % ==== KANAN (60%) ====
  \begin{column}{0.6\textwidth}
    \begin{lstlisting}[style=PythonStyle, numbers=left, firstnumber=1]
lokasi = (6.256, 106.618)  # koordinat Serpong

print("Koordinat Lokasi:")
print("Lintang :", lokasi[0])
print("Bujur  :", lokasi[1])

# tuple tidak dapat diubah
# lokasi[0] = 6.300  # akan menyebabkan error
    \end{lstlisting}
  \end{column}
\end{columns}

\end{frame}


\section{Dictionary (Map)}

\begin{frame}[fragile]{Dictionary: Konsep dan Contoh}
\vspace{20pt}

\begin{columns}[T]
  % ==== KIRI ====
  \begin{column}{0.48\textwidth}
    \begin{itemize}
      \item \textbf{Dictionary} menyimpan pasangan \texttt{key:value}.
      \item Setiap \texttt{key} bersifat unik dan digunakan untuk mengakses nilainya.
      \item Cocok untuk data berlabel, seperti identitas pengguna atau data mahasiswa.
      \item Data dapat \textbf{ditambah}, \textbf{dibaca}, \textbf{diperbarui}, dan \textbf{dihapus}.
      \item Efisien untuk penyimpanan data terstruktur dalam aplikasi nyata.
    \end{itemize}
  \end{column}

  % ==== KANAN ====
  \begin{column}{0.48\textwidth}
    \begin{lstlisting}[style=PythonStyle]
mahasiswa = {
    "nama": "Andi",
    "umur": 20,
    "jurusan": "Informatika"
}

print("Data awal:", mahasiswa)
print("Nama:", mahasiswa["nama"])

mahasiswa["umur"] = 21
mahasiswa["kota"] = "Tangerang"

del mahasiswa["jurusan"]

print("Setelah diubah:", mahasiswa)
    \end{lstlisting}
  \end{column}
\end{columns}

\end{frame}


\begin{frame}[fragile]{Menampilkan Data Pelanggan}
\vspace{20pt}

Program ini menampilkan data pelanggan dalam \texttt{dictionary}.

\begin{columns}[T]
  % ==== KIRI (40%) ====
  \begin{column}{0.4\textwidth}
    \begin{itemize}
      \item Data pelanggan disimpan dalam bentuk pasangan \texttt{key:value}.
      \item Perulangan digunakan untuk menampilkan semua atribut pelanggan.
      \item Cocok untuk laporan data pengguna atau entri basis data sederhana.
    \end{itemize}
  \end{column}

  % ==== KANAN (60%) ====
  \begin{column}{0.6\textwidth}
    \begin{lstlisting}[style=PythonStyle, numbers=left, firstnumber=1]
pelanggan = {
    "nama": "Sinta Dewi",
    "usia": 28,
    "kota": "Tangerang"
}

print("Data Pelanggan:")

for k, v in pelanggan.items():
    print(f"{k.capitalize():<6}: {v}")

# ubah data
pelanggan["usia"] = 29
print("\nUsia diperbarui:",
      pelanggan["usia"])
    \end{lstlisting}
  \end{column}
\end{columns}

\end{frame}


\section{Set}

\begin{frame}[fragile]{Set: Konsep dan Contoh}
\vspace{20pt}

\begin{columns}[T]
  % ==== KIRI ====
  \begin{column}{0.48\textwidth}
    \begin{itemize}
      \item \textbf{Set} berisi kumpulan elemen unik tanpa urutan tertentu.
      \item Python secara otomatis menghapus nilai duplikat.
      \item Digunakan untuk memfilter data unik atau operasi himpunan.
      \item Mendukung operasi \textbf{gabungan} (\texttt{|}) dan \textbf{irisan} (\texttt{\&}).
      \item Efisien untuk memastikan keunikan dan pemeriksaan keanggotaan data.
    \end{itemize}
  \end{column}

  % ==== KANAN ====
  \begin{column}{0.48\textwidth}
    \begin{lstlisting}[style=PythonStyle]
angka = {1, 2, 3, 3, 2}
print("Data awal:", angka)

angka.add(4)
angka.add(5)
angka.discard(2)

print("Setelah diubah:", angka)

angka_lain = {3, 4, 6}
print("Gabungan:", angka | angka_lain)
print("Irisan:", angka & angka_lain)
    \end{lstlisting}
  \end{column}
\end{columns}

\end{frame}


\begin{frame}[fragile]{Data Unik dengan Set}
\vspace{20pt}

Program ini menggunakan \texttt{set} untuk menyimpan data unik dan menampilkannya menggunakan perulangan.

\begin{columns}[T]
  % ==== KIRI (40%) ====
  \begin{column}{0.5\textwidth}
    \begin{itemize}
      \item \texttt{Set} menyimpan elemen unik tanpa urutan tertentu.
      \item Cocok untuk menghapus duplikasi data seperti daftar nama atau produk.
      \item Perulangan digunakan untuk menampilkan seluruh elemen unik.
      \item Efisien untuk pencarian atau pemrosesan data yang tidak boleh ganda.
    \end{itemize}
  \end{column}

  % ==== KANAN (60%) ====
  \begin{column}{0.5\textwidth}
    \begin{lstlisting}[style=PythonStyle, numbers=left, firstnumber=1]
produk = {"kopi", "teh", "gula", "kopi", "susu"}

print("Daftar produk unik:")
for p in produk:
    print("-", p)

# tambahkan elemen baru
produk.add("keju")
print("\nSetelah ditambah:")
for p in produk:
    print("-", p)
    \end{lstlisting}
  \end{column}
\end{columns}

\end{frame}


\section{Range}

\begin{frame}[fragile]{Range: Konsep dan Contoh}
\vspace{20pt}

\begin{columns}[T]
  % ==== KIRI ====
  \begin{column}{0.48\textwidth}
    \begin{itemize}
      \item \textbf{Range} digunakan untuk menghasilkan urutan angka secara efisien.
      \item Umumnya dipakai dalam perulangan untuk mengontrol jumlah iterasi.
      \item Tidak menyimpan seluruh nilai di memori, tetapi menghasilkan angka saat dibutuhkan.
      \item Dapat dikonversi menjadi list dengan \texttt{list(range())}.
      \item Hemat memori dan ideal untuk operasi berulang.
    \end{itemize}
  \end{column}

  % ==== KANAN ====
  \begin{column}{0.48\textwidth}
    \begin{lstlisting}[style=PythonStyle]
for i in range(3):
    print("Iterasi ke-", i)

angka = list(range(1, 6))
print("Daftar angka:", angka)
    \end{lstlisting}
  \end{column}
\end{columns}

\end{frame}

\section{Struktur Data Bertingkat (Nested Structure)}

\begin{frame}[fragile]{Struktur Data Bertingkat: Konsep dan Contoh}
\vspace{20pt}

\begin{columns}[T]
  % ==== KIRI ====
  \begin{column}{0.48\textwidth}
    \begin{itemize}
      \item Struktur data bertingkat berisi struktur data lain di dalamnya.
      \item Contohnya: \texttt{list} di dalam \texttt{list}, atau \texttt{dictionary} di dalam \texttt{list}.
      \item Berguna untuk menyimpan data kompleks seperti tabel, daftar objek, atau data JSON.
      \item Memungkinkan representasi hubungan antar data secara alami dan hierarkis.
    \end{itemize}
  \end{column}

  % ==== KANAN ====
  \begin{column}{0.48\textwidth}
    \begin{lstlisting}[style=PythonStyle]
matriks = [
    [1, 2, 3],
    [4, 5, 6]
]
print("Elemen baris 1 kolom 2:", matriks[0][1])

mahasiswa = [
    {"nama": "Andi", "umur": 20},
    {"nama": "Budi", "umur": 21}
]
print("Nama mahasiswa kedua:",
      mahasiswa[1]["nama"])
    \end{lstlisting}
  \end{column}
\end{columns}

\end{frame}

\section{Perkalian Matriks Menggunakan List}

\begin{frame}[fragile]{Perkalian Matriks Menggunakan List}
\vspace{20pt}
Program ini melakukan perkalian matriks antara A dan B, kemudian menyimpan hasilnya di C.
\begin{columns}[T]
  % ==== KIRI ====
  \begin{column}{0.2\textwidth}
    \begin{lstlisting}[style=PythonStyle, numbers=left, firstnumber=1, basicstyle=\ttfamily\scriptsize]
A = [
    [1, 2, 3],
    [4, 5, 6]
]

B = [
    [7, 8],
    [9, 10],
    [11, 12]
]

C = [
    [0, 0],
    [0, 0]
]

    \end{lstlisting}
  \end{column}

  % ==== KANAN ====
  \begin{column}{0.68\textwidth}
    \begin{lstlisting}[style=PythonStyle, numbers=left, firstnumber=16, basicstyle=\ttfamily\scriptsize]
for i in range(0, len(A)):
    for j in range(0, len(B[0])):
        total = 0
        for k in range(0, len(B)):
            total = total + A[i][k] * B[k][j]
            print(f"{A[i][k]}*{B[k][j]}", end=" ")
            if k < len(B) - 1:
                print("+", end=" ")
        print(f"= {total}")
        C[i][j] = total

print("Hasil akhir matriks C:")
for row in C:
    print(row)
    \end{lstlisting}
  \end{column}
\end{columns}

\end{frame}

\section{Mengolah Data Bersarang dengan Perulangan dan Kondisi}

\begin{frame}[fragile]{Mengolah Data Bersarang dengan Perulangan dan Kondisi}
\vspace{20pt}

Program ini menghitung rata-rata nilai mahasiswa dan 
    menentukan kategori hasil belajar berdasarkan skor akhir.

\begin{columns}[T]
  % ==== KIRI ====
  \begin{column}{0.48\textwidth}
    \begin{lstlisting}[style=PythonStyle, numbers=left, firstnumber=1, basicstyle=\ttfamily\scriptsize]
mahasiswa = [
	{"nama": "Andi", "nilai": [80, 85, 90]},
	{"nama": "Budi", "nilai": [60, 70, 65]},
	{"nama": "Citra", "nilai": [90, 95, 100]}
]

for m in mahasiswa:
    total = 0
    for n in m["nilai"]:
        total += n
    \end{lstlisting}
  \end{column}

  % ==== KANAN ====
  \begin{column}{0.48\textwidth}
    \begin{lstlisting}[style=PythonStyle, numbers=left, firstnumber=11, basicstyle=\ttfamily\scriptsize]
    rata = total / len(m["nilai"])
    
    if rata >= 85:
        kategori = "Sangat Baik"
    elif rata >= 70:
        kategori = "Cukup"
    else:
        kategori = "Perlu Perbaikan"
    
    print(f"{m['nama']} - "
          f"Rata-rata: {rata:.1f} "
          f"({kategori})")
    \end{lstlisting}
  \end{column}
\end{columns}

\end{frame}

\section{Kesimpulan}

\begin{frame}{Kesimpulan}
\vspace{20pt}

\begin{itemize}
  \item Struktur data membantu menyimpan, mengelola, dan memproses informasi secara efisien. 
  \item Python menyediakan berbagai struktur data bawaan seperti \texttt{list}, \texttt{tuple}, \texttt{dictionary}, \texttt{set}, dan \texttt{range}.
  \item Setiap struktur memiliki karakteristik dan kegunaan berbeda — mutable, immutable, berurutan, atau tidak berurutan.
  \item Struktur data bertingkat memungkinkan representasi data kompleks seperti tabel atau objek bersarang.
  \item Pemahaman konsep ini menjadi dasar penting dalam pengembangan algoritma, analisis data, dan aplikasi nyata.
\end{itemize}

\end{frame}




\end{document}