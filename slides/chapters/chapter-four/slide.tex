\documentclass[aspectratio=169, table]{beamer}
\usepackage[utf8]{inputenc}
\usepackage{listings} 
\usepackage[strings]{underscore}
\usepackage{caption}
\usepackage{float}


\renewcommand{\lstlistingname}{} 

\makeatletter
\def\input@path{{../../themes/Pradita}}
\makeatother

\usetheme{Pradita}

\subtitle{IF120203-Programming Fundamentals}

\title{Chatper-04:\\\LARGE{Fungsi dan Modul\\}
\vspace{10pt}}
\date[Serial]{\scriptsize {PRU/SPMI/FR-BM-18/0222}}
\author[Pradita]{\small{\textbf{Alfa Yohannis}}}


% Define Python language style for listings
\lstdefinestyle{PythonStyle}{
language=Python,
basicstyle=\ttfamily\footnotesize,
keywordstyle=\color{blue},
commentstyle=\color{gray},
stringstyle=\color{red},
breaklines=true,
showstringspaces=false,
tabsize=2,
captionpos=b,
numbers=left,
numberstyle=\tiny\color{gray},
comment=[l]{//},
morecomment=[s]{/*}{*/},
commentstyle=\color{gray}\ttfamily,
string=[s]{'}{'},
morestring=[s]{"}{"},
}

\begin{document}

\frame{\titlepage}

% Add table of contents slide
\begin{frame}[fragile]{Contents}
\vspace{15pt}
\begin{columns}[t]
\begin{column}{.5\textwidth}
\tableofcontents[sections={1}]
\end{column}
\begin{column}{.5\textwidth}
\tableofcontents[sections={2-3}]
\end{column}
\end{columns}
\end{frame}

\section{Fungsi}

\begin{frame}{Fungsi di Python}
Fungsi adalah blok kode terorganisir yang memiliki nama tertentu dan dapat dipanggil berulang kali untuk melakukan tugas spesifik, mengurangi redundansi kode, dan mempermudah pengelolaan program. Fungsi membantu kita menulis kode yang lebih rapi, modular, dan mudah dipelihara.
\end{frame}

\begin{frame}{Jenis-Jenis Fungsi di Python}
    \begin{itemize}
        \item Fungsi tanpa parameter dan return
        \item Fungsi dengan parameter
        \item Fungsi dengan return nilai tunggal
        \item Fungsi dengan return lebih dari satu nilai
        \item Fungsi dengan parameter default
        \item Fungsi dengan argumen keyword dan positional
        \item Fungsi dengan jumlah argumen variabel (\texttt{*args})
    \end{itemize}
\end{frame}

\subsection{Fungsi Tanpa Parameter dan Return}
\begin{frame}[fragile]{Fungsi Tanpa Parameter dan Return}
Fungsi sederhana yang tidak menerima parameter dan tidak mengembalikan nilai. Fungsi ini hanya menjalankan perintah tertentu.

\begin{lstlisting}[style=PythonStyle, caption={Kode Python: basic_function.py}]
def greet():
    print("Halo, selamat datang!")

# Memanggil fungsi
greet()
\end{lstlisting}
\end{frame}

\subsection{Fungsi dengan Parameter}

\begin{frame}[fragile]{Fungsi dengan Parameter}
Fungsi bisa memiliki parameter. Dengan adanya parameter, suatu nilai bisa di-sisipkan ke dalam fungsi secara dinamis saat pemanggilannya.
Parameter sendiri merupakan istilah untuk variabel yang menempel pada fungsi, yang mengharuskan kita untuk menyisipkan nilai pada parameter tersebut saat pemanggilan fungsi.

\begin{lstlisting}[style=PythonStyle, caption={Kode Python: parameter_function.py}]
def greet_with_name(nama):
    print(f"Halo, {nama}!")

# Memanggil fungsi dengan argumen
greet_with_name("Jessie")
\end{lstlisting}
\end{frame}

\subsection{Fungsi dengan Nilai Kembalian (Return Value)}

\begin{frame}[fragile]{Fungsi dengan Nilai Kembalian (1)}
Fungsi dapat mengembalikan hasil yang dapat disimpan atau digunakan dalam perhitungan lain.

\subsubsection{Return Nilai Tunggal}
\begin{itemize}
    \item Fungsi dengan satu nilai kembalian
        \begin{lstlisting}[style=PythonStyle, caption={Kode Python: function_with_single_return.py}]
        def add(a, b):
            return a + b

        summation_result = add(5, 3)
        print(summation_result)  # Output: 8
        \end{lstlisting}
\end{itemize}
\end{frame}

\begin{frame}[fragile]{Fungsi dengan Nilai Kembalian (2)}
\subsubsection{Return Lebih dari Satu Nilai}
\begin{itemize}
    \item Fungsi dengan lebih dari satu nilai kembalian
        \begin{lstlisting}[style=PythonStyle, caption={Kode Python: function_with_multiple_return.py}]
        def operate(a, b):
            return a + b, a * b

        sum_result, product_result = operate(4, 5)
        print(sum_result)     # 9
        print(product_result) # 20
        \end{lstlisting}
\end{itemize}
\end{frame}

\subsection{Fungsi dengan Parameter Default}
\begin{frame}[fragile]{Fungsi dengan Parameter Default}
Fungsi dapat memiliki nilai default untuk parameter jika argumen tidak diberikan saat pemanggilan.

\begin{lstlisting}[style=PythonStyle, caption={Kode Python: function_with_default_parameter.py}]
def greet_you(name="Friend"):
    print(f"Hello, {name}!")

greet_you()        # Output: Hello, Friend!
greet_you("Andrew")  # Output: Hello, Andrew!
\end{lstlisting}
\end{frame}

\subsection{Fungsi Argumen Keyword dan Positional}
\begin{frame}[fragile]{Keyword and Positional Arguments (1)}
Fungsi di Python bisa dipanggil menggunakan positional arguments atau keyword arguments. Positional argument adalah istilah untuk urutan parameter/argument fungsi. Pengisian argument saat pemanggilan fungsi harus urut sesuai dengan deklarasi parameternya. Keyword argument atau named argument adalah metode pengisian argument pemanggilan fungsi disertai nama parameter yang ditulis secara jelas (eksplisit).
\end{frame}

\begin{frame}[fragile]{Keyword and Positional Arguments (2)}
\begin{lstlisting}[style=PythonStyle, caption={Kode Python: function_with_keyword_and_positional.py}]
def student_info(name, age, major):
    print(f"{name}, {age} years old, majoring in {major}")

# Positional arguments
student_info("Delta", 20, "Computer Science")

# Keyword arguments
student_info(major="Information Systems", name="Echo", age=21)
\end{lstlisting}
\end{frame}

\subsection{Fungsi dengan Jumlah Argumen Variabel}

\begin{frame}[fragile]{Fungsi dengan Jumlah Argumen Variabel}
Jika jumlah argumen tidak pasti, kita bisa menggunakan \texttt{*args}.
\begin{lstlisting}[style=PythonStyle, caption={Kode Python: function_with_variable_arguments.py}]
def sum_numbers(*numbers):
    total = sum(numbers)
    print(f"Total: {total}")

sum_numbers(1, 2, 3, 4)  # Output: Total: 10
sum_numbers(5, 6, 7)     # Output: Total: 18
\end{lstlisting}
\end{frame}

\subsection{Fungsi Rekursif}
\begin{frame}[fragile]{Fungsi Rekursif}
Fungsi yang memanggil dirinya sendiri. Biasanya digunakan untuk masalah yang dapat dipecah menjadi sub-masalah.

\begin{lstlisting}[style=PythonStyle, caption={Kode Python: recursive_function.py}]
def factorial(n):
    if n == 0 or n == 1:
        return 1
    else:
        return n * factorial(n-1)

print(factorial(5))  # Output: 120
\end{lstlisting}
\end{frame}

\section{Modul}

\begin{frame}[fragile]{Modul}
Modul adalah file Python (\texttt{.py}) yang berisi kode seperti fungsi, variabel, atau kelas, yang bisa digunakan kembali di program lain. Modul membantu memecah program menjadi bagian-bagian yang lebih kecil dan terstruktur.
\end{frame}

\subsection{Membuat Modul}

\begin{frame}[fragile]{Membuat Modul (1)}
Modul sendiri dibuat dengan membuat file Python baru. Misalnya kita buat file \texttt{math_operations.py}:

\begin{lstlisting}[style=PythonStyle, caption={Kode Python: math_operations.py}]
def add(a, b):
    """Mengembalikan hasil penjumlahan a + b"""
    return a + b

def subtract(a, b):
    """Mengembalikan hasil pengurangan a - b"""
    return a - b
\end{lstlisting}
\end{frame}

\begin{frame}[fragile]{Membuat Modul (2)}
\begin{lstlisting}[style=PythonStyle, caption={Kode Python: math_operations.py}]
def multiply(a, b):
    """Mengembalikan hasil perkalian a * b"""
    return a * b

def divide(a, b):
    """Mengembalikan hasil pembagian a / b"""
    if b == 0:
        return "Error: Division by zero!"
    return a / b

def power(a, b):
    """Mengembalikan hasil a pangkat b"""
    return a ** b
\end{lstlisting}
\end{frame}

\begin{frame}[fragile]{Membuat Modul (3)}
Lalu kita bisa menggunakan modul ini di file program lain:

\begin{lstlisting}[style=PythonStyle, caption={Kode Python: calculator.py}]
import math_operations

print(math_operations.add(5, 3))        # Output: 8
print(math_operations.subtract(10, 4))   # Output: 6
print(math_operations.multiply(2, 7))    # Output: 14
print(math_operations.divide(10, 2))     # Output: 5.0
print(math_operations.power(2, 3))       # Output: 8
\end{lstlisting}
    
\end{frame}

\subsection{Mengimpor Modul dengan Alias}
\begin{frame}[fragile]{Mengimpor Modul dengan Alias}
Kita bisa memberi nama alias saat mengimpor modul agar lebih ringkas:

\begin{lstlisting}[style=PythonStyle, caption={Kode Python: calculator.py}]
import math_operations as mo

print(mo.add(5, 3))        # Output: 8
print(mo.subtract(10, 4))  # Output: 6
print(mo.multiply(2, 7))   # Output: 14
print(mo.divide(10, 2))    # Output: 5.0
print(mo.power(3, 4))      # Output: 81
\end{lstlisting}
\end{frame}

\subsection{Menaruh Modul dalam Folder}
\begin{frame}[fragile]{Menaruh Modul dalam Folder (1)}
Selain membuat modul di satu file, kita juga bisa menaruh modul di dalam folder supaya lebih rapi. Misalnya:

\begin{verbatim}
project/
│
├── main.py
└── utils/
└── string_utils.py
\end{verbatim}

Isi \texttt{string_utils.py} misalnya:
\end{frame}

\begin{frame}[fragile]{Menaruh Modul dalam Folder (2)}
\begin{lstlisting}[style=PythonStyle, caption={Kode Python: utils/string_utils.py}]
def to_upper(text):
return text.upper()

def to_lower(text):
return text.lower()
\end{lstlisting}

Di \texttt{main.py}, kita bisa mengimpor modul ini dari folder \texttt{utils}:

\begin{lstlisting}[style=PythonStyle, caption={Kode Python: main.py}]
from utils import string_utils

print(string_utils.to_upper("Python")) # Output: PYTHON
print(string_utils.to_lower("Python")) # Output: python
\end{lstlisting}
\end{frame}

\subsubsection{Best Practice: Paket dengan __init__.py}
\begin{frame}[fragile]{Best Practice: Paket dengan __init__.py}
Untuk project yang lebih besar atau modul yang akan digunakan di banyak file, sebaiknya folder modul dijadikan \textbf{package} dengan menambahkan file __init__.py:

\begin{verbatim}
project/
│
├── main.py
└── utils/
├── __init__.py
└── string_utils.py
\end{verbatim}

Dengan __init__.py, Python mengenali folder sebagai package.

Cara import tetap sama:

\begin{lstlisting}[style=PythonStyle]
from utils import string_utils
\end{lstlisting}
\end{frame}

\subsection{Modul Bawaan Python}
\begin{frame}[fragile]{Modul Bawaan Python (1)}
Python memiliki banyak modul bawaan yang bisa langsung digunakan tanpa instalasi. Beberapa modul bawaan yang sering dipakai antara lain:

\begin{itemize}
  \item \texttt{math}  — untuk operasi matematika, seperti akar, pangkat, atau konstanta \(\pi\).
  \item \texttt{random} — untuk menghasilkan angka acak.
  \item \texttt{datetime} — untuk mengelola tanggal dan waktu.
  \item \texttt{os} — untuk berinteraksi dengan sistem operasi, misal folder, file, path.
\end{itemize}
\end{frame}

\begin{frame}[fragile]{Modul Bawaan Python (2)}
Contoh penggunaan modul bawaan:

\begin{lstlisting}[style=PythonStyle, caption={Kode Python: math_module.py}]
import math

print(math.sqrt(16)) # Output: 4.0
print(math.pi) # Output: 3.141592653589793
\end{lstlisting}

\begin{lstlisting}[style=PythonStyle, caption={Kode Python: random_module.py}]
import random

print(random.randint(1, 10)) # Output: angka acak antara 1 sampai 10
\end{lstlisting}
\end{frame}

\begin{frame}[fragile]{Modul Bawaan Python (3)}
\begin{lstlisting}[style=PythonStyle, caption={Kode Python: datetime_module.py}]
from datetime import date

today = date.today()
print(today) # Output: tanggal hari ini, misal 2025-09-20
\end{lstlisting}
\end{frame}

\subsection{Mengimpor Fungsi atau Variabel Tertentu}
\begin{frame}[fragile]{Mengimpor Fungsi atau Variabel Tertentu}
Jika hanya membutuhkan beberapa fungsi/variabel dari modul, bisa langsung diimpor:

\begin{lstlisting}[style=PythonStyle]
from math import sqrt, pi

print(sqrt(36)) # Output: 6.0
print(pi) # Output: 3.141592653589793
\end{lstlisting}
\end{frame}

\section{Penutup}
\begin{frame}{Penutup}
\begin{itemize}
    \item \textbf{Fungsi:} Blok kode bernama, bisa dipanggil berulang kali.
    \item \textbf{Parameter and Return:} Input fleksibel, bisa mengembalikan nilai.
    \item \textbf{Fungsi Rekursif:} Memanggil dirinya sendiri.
    \item \textbf{Modul:} File .py berisi fungsi/variabel/kelas, untuk kode modular.
    \item \textbf{Import Modul:} Bisa dengan \texttt{import}, alias (\texttt{as}), atau \texttt{from ... import ...}.
    \item \textbf{Package:} Folder modul + \_\_init\_\_.py untuk proyek besar.
    \item \textbf{Modul Bawaan:} Contoh: \texttt{math}, \texttt{random}, \texttt{datetime}, \texttt{os}.
\end{itemize}
\end{frame}

\end{document}