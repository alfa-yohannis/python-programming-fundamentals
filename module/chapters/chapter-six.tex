\chapter{Struktur Data Bawaan dalam Python}

\section{Pendahuluan}
Struktur data adalah cara menyimpan dan mengatur data agar dapat digunakan secara efisien dalam program.
Python menyediakan berbagai struktur data bawaan seperti \texttt{list}, \texttt{tuple}, \texttt{dictionary}, dan \texttt{set}.
Pemahaman terhadap struktur data sangat penting karena menjadi dasar dalam pengolahan informasi dan pengembangan algoritma.

\begin{lstlisting}[style=PythonStyle]
data = [10, 20, 30]
print(data[0])  # Mengakses elemen pertama
\end{lstlisting}
Kode di atas menggunakan list untuk menyimpan tiga nilai dan menampilkan elemen pertama.

% =====================================================
\section{List}
List adalah struktur data berurutan (sequential) yang bersifat mutable, artinya elemen di dalamnya dapat diubah setelah dibuat. 
List digunakan untuk menyimpan kumpulan data yang sejenis atau campuran, seperti daftar nama, nilai, atau hasil input pengguna. 
Karena bersifat dinamis, kita dapat menambah, membaca, memperbarui, dan menghapus elemen kapan pun diperlukan.

\begin{lstlisting}[style=PythonStyle]
buah = ["apel", "jeruk", "mangga"]

print("Daftar buah:", buah)
print("Buah pertama:", buah[0])

buah[1] = "anggur"
buah.append("pisang")
buah.remove("mangga")

print("Setelah diubah:", buah)
\end{lstlisting}

Kode di atas memperlihatkan bagaimana list dapat dimodifikasi dengan mudah. 
Program dimulai dengan tiga elemen awal, lalu menampilkan isi list dan elemen pertama. 
Nilai pada posisi tertentu dapat diubah langsung dengan indeks, elemen baru dapat ditambahkan dengan \texttt{append()}, 
dan elemen tertentu dapat dihapus dengan \texttt{remove()}. 
List menjadi struktur dasar yang sangat penting karena fleksibel dan mudah digunakan dalam berbagai situasi.

% =====================================================
\section{Tuple}
Tuple mirip dengan list, tetapi bersifat \textit{immutable}, artinya elemen di dalamnya tidak dapat diubah setelah dibuat. 
Struktur ini cocok digunakan untuk menyimpan data yang bersifat tetap seperti koordinat, ukuran gambar, warna RGB, atau data konfigurasi. 
Jika ingin melakukan perubahan, kita harus membuat tuple baru berdasarkan tuple lama.

\begin{lstlisting}[style=PythonStyle]
koordinat = (10, 20)
print("Koordinat awal:", koordinat)
print("Nilai x:", koordinat[0])

# membuat tuple baru berdasarkan tuple lama
koordinat_baru = (koordinat[0], 25)
print("Koordinat baru:", koordinat_baru)
\end{lstlisting}

Kode di atas menunjukkan bahwa tuple dapat dibaca seperti list, 
tetapi tidak bisa dimodifikasi secara langsung. 
Untuk "memperbarui" data, kita membuat tuple baru dari data lama. 
Karena bersifat tetap, tuple aman digunakan untuk data yang tidak boleh berubah selama program berjalan.

% =====================================================
\section{Dictionary (Map)}
Dictionary adalah struktur data yang menyimpan pasangan \texttt{key:value}. 
Setiap \texttt{key} bersifat unik dan digunakan untuk mengakses nilainya secara langsung. 
Struktur ini sangat berguna untuk merepresentasikan data berlabel, seperti identitas pengguna, data mahasiswa, atau konfigurasi sistem. 
Kita dapat menambah, membaca, memperbarui, maupun menghapus pasangan data dengan mudah.

\begin{lstlisting}[style=PythonStyle]
mahasiswa = {"nama": "Andi", "umur": 20, "jurusan": "Informatika"}

print("Data awal:", mahasiswa)
print("Nama mahasiswa:", mahasiswa["nama"])

mahasiswa["umur"] = 21
mahasiswa["kota"] = "Tangerang"

del mahasiswa["jurusan"]

print("Setelah diubah:", mahasiswa)
\end{lstlisting}

Kode di atas menunjukkan bagaimana dictionary memungkinkan pengelolaan data berbasis label. 
Kita dapat membaca nilai tertentu dengan menyebutkan kuncinya, 
menambahkan data baru dengan menetapkan pasangan \texttt{key:value}, 
memperbarui nilai yang sudah ada, atau menghapus entri dengan \texttt{del}. 
Struktur ini sangat efisien untuk data berpasangan dan sering digunakan dalam aplikasi nyata seperti penyimpanan data pengguna atau pengaturan sistem.

% =====================================================
\section{Set}
Set adalah struktur data yang berisi kumpulan elemen unik dan tidak memiliki urutan tertentu. 
Python secara otomatis menghapus elemen duplikat, sehingga setiap nilai di dalam set hanya muncul satu kali. 
Struktur ini sering digunakan untuk memfilter data unik, melakukan operasi himpunan seperti gabungan dan irisan, 
atau memeriksa keanggotaan suatu elemen dalam kumpulan data.

\begin{lstlisting}[style=PythonStyle]
angka = {1, 2, 3, 3, 2}
print("Data awal:", angka)

angka.add(4)
angka.add(5)
angka.discard(2)

print("Setelah diubah:", angka)

angka_lain = {3, 4, 6}
print("Gabungan:", angka | angka_lain)
print("Irisan:", angka & angka_lain)
\end{lstlisting}

Kode di atas memperlihatkan bahwa set secara otomatis menghapus nilai duplikat 
dan memungkinkan penambahan atau penghapusan elemen dengan mudah. 
Operasi seperti gabungan (\texttt{|}) dan irisan (\texttt{\&}) dapat dilakukan untuk menggabungkan atau mencari elemen yang sama di antara dua set. 
Struktur ini sangat efisien untuk memastikan keunikan data dan operasi pencarian cepat.


% =====================================================
\section{Range}
Range adalah struktur bawaan Python yang digunakan untuk menghasilkan urutan angka secara efisien. 
Objek ini sering digunakan dalam perulangan untuk mengontrol jumlah iterasi tanpa perlu membuat daftar angka secara manual. 
Tidak seperti list, \texttt{range} tidak menyimpan seluruh nilai di memori, melainkan menghasilkan angka satu per satu saat dibutuhkan, sehingga lebih hemat sumber daya.

\begin{lstlisting}[style=PythonStyle]
for i in range(3):
    print("Iterasi ke-", i)

angka = list(range(1, 6))
print("Daftar angka:", angka)
\end{lstlisting}

Contoh di atas menunjukkan penggunaan \texttt{range()} dalam dua bentuk. 
Pertama, untuk mengatur jumlah pengulangan di dalam perulangan \texttt{for}. 
Kedua, untuk membuat daftar angka berurutan dengan fungsi \texttt{list()}. 
Dengan \texttt{range()}, kita dapat membuat urutan angka dengan batas awal, akhir, dan langkah tertentu tanpa membebani memori.


% =====================================================
\section{Struktur Data Bertingkat (Nested Structure)}
Struktur data bertingkat adalah struktur yang di dalamnya terdapat struktur data lain. 
Contohnya adalah list di dalam list, atau dictionary di dalam list. 
Struktur seperti ini berguna untuk menyimpan data yang lebih kompleks, seperti tabel, daftar objek, atau data dalam format JSON. 
Dengan menggunakan struktur bertingkat, kita dapat merepresentasikan hubungan antar data secara lebih alami.

\begin{lstlisting}[style=PythonStyle]
matriks = [
    [1, 2, 3],
    [4, 5, 6]
]

print("Elemen pada baris 1 kolom 2:", matriks[0][1])

mahasiswa = [
    {"nama": "Andi", "umur": 20},
    {"nama": "Budi", "umur": 21}
]

print("Nama mahasiswa kedua:", mahasiswa[1]["nama"])
\end{lstlisting}

Contoh di atas memperlihatkan dua bentuk struktur bertingkat. 
Pertama, list di dalam list yang digunakan untuk menyimpan data berbentuk dua dimensi seperti matriks. 
Kedua, list yang berisi dictionary yang digunakan untuk menyimpan daftar data berstruktur, misalnya data mahasiswa. 
Struktur seperti ini umum digunakan untuk pengolahan data yang bersifat hierarkis, penyimpanan data JSON, maupun hasil query dari basis data.

% =====================================================
\section{Perkalian Matriks Menggunakan List}
Matriks dapat direpresentasikan sebagai list dua dimensi, di mana setiap elemen di dalam list utama berisi list lain yang mewakili baris. 
Perkalian matriks dilakukan dengan menjumlahkan hasil kali antara elemen-elemen baris dari matriks pertama dan kolom dari matriks kedua. 
Konsep ini banyak digunakan dalam berbagai bidang seperti pemrosesan citra, machine learning, dan grafik komputer.

\begin{lstlisting}[style=PythonStyle]
A = [
    [1, 2, 3],
    [4, 5, 6]
]

B = [
    [7, 8],
    [9, 10],
    [11, 12]
]

C = [
    [0, 0],
    [0, 0]
]

for i in range(0, len(A)):
    for j in range(0, len(B[0])):
        total = 0
        for k in range(0, len(B)):
            total = total + A[i][k] * B[k][j]
            print(f"{A[i][k]}*{B[k][j]}", end=" ")
            if k < len(B) - 1:
                print("+", end=" ")
        print(f"= {total}")
        C[i][j] = total

print("Hasil akhir matriks C:")
for row in C:
    print(row)
\end{lstlisting}

Contoh di atas menunjukkan bagaimana dua matriks dapat dikalikan menggunakan list bersarang. 
Matriks \texttt{A} berukuran dua baris dan tiga kolom dikalikan dengan matriks \texttt{B} berukuran tiga baris dan dua kolom. 
Hasilnya adalah matriks baru \texttt{C} berukuran dua baris dan dua kolom. 
Proses perhitungan dilakukan dengan menggunakan \textit{nested loop} di dalam \textit{list comprehension}, 
di mana setiap elemen hasil merupakan penjumlahan hasil kali antara baris pada matriks pertama dan kolom pada matriks kedua. 
Pendekatan ini memperlihatkan bagaimana struktur data sederhana seperti list dapat digunakan untuk menyelesaikan perhitungan matematis secara efisien tanpa memerlukan pustaka tambahan.

\section{Mengolah Data Bersarang dengan Perulangan dan Kondisi}
Struktur data bersarang sering digunakan dalam situasi di mana setiap elemen di dalam kumpulan data memiliki beberapa atribut. 
Dengan memanfaatkan perulangan bertingkat (\textit{nested loop}) dan kondisi, kita dapat menelusuri dan memproses data tersebut dengan lebih terarah. 
Pendekatan ini banyak digunakan dalam pengolahan data, analisis hasil survei, atau pembuatan laporan berdasarkan kriteria tertentu.

\begin{lstlisting}[style=PythonStyle]
mahasiswa = [
    {"nama": "Andi", "nilai": [80, 85, 90]},
    {"nama": "Budi", "nilai": [60, 70, 65]},
    {"nama": "Citra", "nilai": [90, 95, 100]}
]

for m in mahasiswa:
    total = 0
    for n in m["nilai"]:
        total += n
    rata = total / len(m["nilai"])
    
    if rata >= 85:
        kategori = "Sangat Baik"
    elif rata >= 70:
        kategori = "Cukup"
    else:
        kategori = "Perlu Perbaikan"
    
    print(f"{m['nama']} - Rata-rata: {rata:.1f} ({kategori})")
\end{lstlisting}

Contoh di atas memperlihatkan bagaimana list yang berisi dictionary dan list lain di dalamnya dapat diolah menggunakan dua tingkat perulangan. 
Perulangan pertama menelusuri setiap mahasiswa, sedangkan perulangan kedua menjumlahkan nilai-nilai mereka untuk menghitung rata-rata. 
Kondisi \texttt{if} kemudian digunakan untuk menentukan kategori hasil belajar berdasarkan nilai rata-rata tersebut. 
Contoh ini menggabungkan konsep struktur data bersarang, perulangan bertingkat, dan logika percabangan dalam satu program sederhana yang menggambarkan situasi dunia nyata.

% =====================================================
\section{Latihan}
Latihan berikut dirancang agar mahasiswa dapat menerapkan berbagai konsep struktur data Python 
secara terpadu — meliputi list, tuple, dictionary, serta penggunaan perulangan, kondisi, dan struktur bersarang. Semua solusi harus dibuat moduler ke dalam modul dan fungsi-fungsi yang kemudian dipanggil di fungsi utama. 

\begin{enumerate}
\item \textbf{Mencari Nilai Tertinggi, Terendah, dan Nilai Spesifik} \\
Sebuah kelas memiliki daftar nilai ujian yang disimpan dalam bentuk \texttt{list}. 
Mahasiswa diminta untuk menentukan nilai tertinggi, nilai terendah, 
dan memeriksa apakah nilai tertentu ada di dalam daftar tersebut.  

Gunakan struktur data berikut sebagai dasar:

\begin{lstlisting}[style=PythonStyle]
nilai_ujian = [78, 85, 90, 67, 88, 92, 74, 90, 81]
\end{lstlisting}

Tentukan:
\begin{itemize}
  \item \textbf{Nilai tertinggi}: nilai maksimum dari daftar.
  \item \textbf{Nilai terendah}: nilai minimum dari daftar.
  \item \textbf{Nilai spesifik}: periksa apakah nilai tertentu (misalnya 90) ada di dalam daftar.
\end{itemize}


Latihan ini membantu memahami penggunaan fungsi bawaan seperti 
\texttt{max()}, \texttt{min()}, dan operator \texttt{in} untuk pencarian elemen pada list.

\item \textbf{Mengurutkan Data Menggunakan List} \\
Sebuah daftar nilai ujian siswa disimpan dalam bentuk \texttt{list}. 
Mahasiswa diminta untuk menampilkan data yang telah diurutkan dalam dua cara:  
(1) dari nilai tertinggi ke terendah, dan  
(2) dari nilai terendah ke tertinggi.

Gunakan struktur data berikut sebagai dasar:

\begin{lstlisting}[style=PythonStyle]
nilai = [85, 90, 78, 92, 88, 75, 95]
\end{lstlisting}

Tentukan dua hasil pengurutan berikut:
\begin{itemize}
  \item \textbf{Ascending}: dari nilai terkecil ke terbesar.
  \item \textbf{Descending}: dari nilai terbesar ke terkecil.
\end{itemize}

Hasil akhir yang diharapkan memiliki bentuk struktur data seperti berikut:

\begin{lstlisting}[style=PythonStyle]
urut_ascending = [75, 78, 85, 88, 90, 92, 95]
urut_descending = [95, 92, 90, 88, 85, 78, 75]
\end{lstlisting}

Latihan ini membantu memahami penggunaan fungsi bawaan \texttt{sorted()} 
dan metode \texttt{.sort()} untuk mengurutkan data numerik secara menaik dan menurun.


  \item \textbf{Koordinat Pusat Massa Kubus} \\
  Sebuah kubus memiliki delapan titik sudut dengan koordinat dalam ruang tiga dimensi. 
  Setiap titik direpresentasikan sebagai tuple \texttt{(x, y, z)} dan seluruh titik disimpan dalam sebuah list.  
  Buat program untuk menghitung koordinat pusat massa (titik rata-rata) dari kubus tersebut.  
  Gunakan perulangan untuk menjumlahkan seluruh nilai koordinat dan tampilkan hasil akhirnya dalam bentuk tuple baru.



  \item \textbf{Seleksi dan Gabung Data Berdasarkan Kriteria} \\
Terdapat dua struktur data yang saling melengkapi. 
Struktur pertama menyimpan daftar produk dan harganya, 
sementara struktur kedua menyimpan daftar produk dan jumlah stok yang tersedia. 
Kedua struktur disusun dalam bentuk list berisi dictionary seperti berikut:

\begin{lstlisting}[style=PythonStyle]
produk_harga = [
    {"nama": "Laptop", "harga": 9500000},
    {"nama": "Mouse", "harga": 150000},
    {"nama": "Keyboard", "harga": 350000},
    {"nama": "Monitor", "harga": 2200000}
]

produk_stok = [
    {"nama": "Laptop", "stok": 3},
    {"nama": "Mouse", "stok": 25},
    {"nama": "Keyboard", "stok": 10},
    {"nama": "Monitor", "stok": 4}
]
\end{lstlisting}

Dari dua struktur tersebut, mahasiswa diminta menyeleksi produk 
dengan harga di bawah batas tertentu dan stok di atas jumlah tertentu, 
kemudian menggabungkannya ke dalam satu struktur data baru yang berisi 
\texttt{nama}, \texttt{harga}, dan \texttt{stok}. 

Hasil akhir yang diharapkan berbentuk seperti berikut:

\begin{lstlisting}[style=PythonStyle]
produk_terpilih = [
    {"nama": "Mouse", "harga": 150000, "stok": 25},
    {"nama": "Keyboard", "harga": 350000, "stok": 10}
]
\end{lstlisting}

Struktur ini merepresentasikan data hasil seleksi yang memenuhi kriteria dan sudah digabungkan dari dua sumber data berbeda.


\item \textbf{Operasi Himpunan Menggunakan Set} \\
Dua kelompok data mewakili pelanggan dari dua cabang toko yang berbeda. 
Setiap kelompok disimpan dalam bentuk \texttt{set} karena setiap nama pelanggan bersifat unik.  
Mahasiswa diminta melakukan berbagai operasi himpunan untuk menganalisis kesamaan dan perbedaan antar cabang.

\begin{lstlisting}[style=PythonStyle]
cabang_a = {"Andi", "Budi", "Citra", "Dewi", "Eka"}
cabang_b = {"Budi", "Dewi", "Farah", "Gilang", "Hadi"}
\end{lstlisting}

Gunakan operasi berikut untuk menemukan hasilnya:
\begin{itemize}
  \item \textbf{Irisan (AND)}: pelanggan yang berbelanja di kedua cabang.
  \item \textbf{Gabungan (OR)}: seluruh pelanggan dari kedua cabang tanpa duplikasi.
  \item \textbf{Selisih (NOT IN)}: pelanggan yang hanya berbelanja di satu cabang tertentu.
  \item \textbf{Selisih Simetris (XOR)}: pelanggan yang hanya berbelanja di salah satu cabang saja.
\end{itemize}

Hasil akhir diharapkan memiliki bentuk struktur data seperti berikut:

\begin{lstlisting}[style=PythonStyle]
pelanggan_and = {"Budi", "Dewi"}
pelanggan_or = {"Andi", "Budi", "Citra", "Dewi", "Eka", "Farah", "Gilang", "Hadi"}
pelanggan_not_in_a = {"Farah", "Gilang", "Hadi"}
pelanggan_xor = {"Andi", "Citra", "Eka", "Farah", "Gilang", "Hadi"}
\end{lstlisting}

Struktur ini membantu memahami hubungan antar himpunan data dan dapat digunakan untuk 
analisis pelanggan, pencocokan data, maupun perbandingan hasil survei.

\end{enumerate}

% =====================================================
\section{Kesimpulan}
Python memiliki berbagai struktur data bawaan yang fleksibel dan mudah digunakan.
Pemahaman tentang list, tuple, dictionary, dan set menjadi dasar penting sebelum mempelajari algoritma dan struktur data lanjutan.
Dengan menguasai struktur data ini, programmer dapat menulis kode yang lebih efisien, terstruktur, dan mudah dipelihara.
