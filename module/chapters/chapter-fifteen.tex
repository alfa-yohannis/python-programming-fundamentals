\chapter{Tugas Akhir}

Buatlah Game multiplayer (minimal 2 player) dengan menggunakan bahasa pemrograman Python yang mencakup semua topik yang diajarkan pada semester ini. Berikut adalah persentasi komponen penilaian dari tugas tersebut.

\begin{table}[h]
\centering
\begin{tabular}{l p{.4\textwidth} r}
\hline
\textbf{No} & \textbf{Komponen Penilaian} & \textbf{Bobot (\%)} \\
\hline
1  & variabel dan konstanta                                & 2  \\
2  & tipe data dan konversi tipe data                      & 2  \\
3  & operator                                               & 2  \\
4  & pengkondisian                                          & 5  \\
5  & modul                                                  & 4  \\
6  & fungsi                                                 & 4  \\
7  & perulangan                                             & 5  \\
8  & struktur data (salah satu: list/tuple/dictionary /set)        & 5  \\
9  & file input (read)                                      & 4  \\
10 & file output (save)                                     & 4  \\
11 & kelas, objek, atribut, metode                          & 8  \\
12 & exception                                              & 4  \\
13 & unit test                                              & 4  \\
14 & tkinter                                                & 7  \\
15 & tingkat kesulitan                                      & 20 \\
16 & kelengkapan, keutuhan, atau penyelesaian software (berfungsi dengan baik)                                       & 20 \\
\hline
   & \textbf{Total}                                         & \textbf{100} \\
\hline
\end{tabular}
\caption{Komponen Penilaian dan Bobot}
\end{table}


Kumpulkan \textbf{source code} disertai dengan \textbf{video} (\textbf{durasi maksimum 5 menit}) yang menjelaskan kode game yang Anda buat. Di video tersebut:
\begin{itemize}
\item Deskripsikan secara umum program yang Anda buat.
\item Demo-kan game Anda.
\item Jelaskan cara kerjanya.
\item Semua komponen penilaian di atas harus Anda sebutkan di bagian mana (di dalam program) mereka digunakan serta fungsinya di program tersebut.
\end{itemize}

Tempat pengumpulan akan diberitahukan kemudian.
