\chapter{Perulangan (Looping)}

\section{Perulangan di Python}
Dalam pemrograman, seringkali kita perlu menjalankan blok kode yang sama berulang kali. 
Misalnya, mencetak angka dari 1 sampai 100, membaca setiap baris dari sebuah file, atau 
memproses setiap elemen dalam sebuah daftar data. 
Proses pengulangan eksekusi blok kode ini dikenal sebagai \textbf{perulangan} atau \textit{looping} (iterasi).

Python menyediakan dua mekanisme utama untuk melakukan perulangan:
\begin{enumerate}
    \item \textbf{Perulangan \texttt{for}}: Digunakan untuk melakukan iterasi pada sebuah urutan (seperti \texttt{list}, \texttt{tuple}, \texttt{string}) atau objek \textit{iterable} lainnya. Perulangan ini sering disebut sebagai \textit{definite iteration} karena jumlah pengulangannya sudah ditentukan oleh panjang urutan.
    \item \textbf{Perulangan \texttt{while}}: Digunakan untuk mengulang blok kode selama sebuah kondisi bernilai \texttt{True}. Perulangan ini disebut \textit{indefinite iteration} karena jumlah pengulangannya tidak pasti dan bergantung pada kapan kondisi menjadi \texttt{False}.
\end{enumerate}

Menguasai perulangan adalah langkah fundamental untuk menulis program yang efisien dan otomatis.

\subsection{For Loop}

Perulangan \texttt{for} di Python bekerja dengan cara mengambil setiap elemen dari sebuah urutan secara bergantian.

\subsection{Sintaks Dasar}
Sintaks umum dari perulangan \texttt{for} adalah sebagai berikut:
\begin{lstlisting}[style=PythonStyle, caption={Sintaks Dasar Perulangan for}]
for nama_variabel in urutan:
    # Blok kode yang akan diulang
    # ...
\end{lstlisting}

\subsection{Iterasi Menggunakan \texttt{range()}}
Fungsi \texttt{range()} sangat umum digunakan bersama \texttt{for} untuk menghasilkan urutan angka.
\begin{itemize}
    \item \texttt{range(stop)}: Membuat urutan dari 0 hingga \texttt{stop-1}.
    \item \texttt{range(start, stop)}: Membuat urutan dari \texttt{start} hingga \texttt{stop-1}.
    \item \texttt{range(start, stop, step)}: Membuat urutan dari \texttt{start} hingga \texttt{stop-1} dengan lompatan sebesar \texttt{step}.
\end{itemize}

\begin{lstlisting}[style=PythonStyle, caption={Kode Python: for_with_range.py}]
# Mencetak angka dari 0 sampai 4
print("Contoh 1: range(5)")
for i in range(5):
    print(f"Perulangan ke-{i}")

# Mencetak angka dari 2 sampai 5
print("\nContoh 2: range(2, 6)")
for j in range(2, 6):
    print(f"Angka: {j}")
\end{lstlisting}

\subsection{Iterasi pada List dan String}
Anda bisa melakukan iterasi secara langsung pada elemen-elemen dari sebuah \texttt{list} atau karakter-karakter dari sebuah \texttt{string}.

\begin{lstlisting}[style=PythonStyle, caption={Kode Python: list_and_string_iteration.py}]
# Iterasi pada sebuah list
daftar_buah = ["apel", "mangga", "jeruk"]
for buah in daftar_buah:
    print(f"Saya suka {buah}")

# Iterasi pada sebuah string
nama = "PYTHON"
for huruf in nama:
    print(huruf, end=' ')
\end{lstlisting}

\section{While Loop}
Perulangan \texttt{while} akan terus mengeksekusi blok kode di dalamnya selama kondisi yang diberikan bernilai \texttt{True}.

\subsection{Sintaks Dasar}
\begin{lstlisting}[style=PythonStyle, caption={Sintaks Dasar Perulangan while}]
while kondisi:
    # Blok kode yang akan diulang
    # ...
    # Penting: Harus ada perubahan yang membuat kondisi akhirnya False
\end{lstlisting}

\textbf{Perhatian:} Pastikan di dalam blok \texttt{while} ada sebuah mekanisme (misalnya, inkrementasi variabel) yang pada akhirnya akan mengubah nilai kondisi menjadi \texttt{False}. Jika tidak, program akan masuk ke dalam \textit{infinite loop} atau perulangan tak terbatas.

\subsection{Contoh Penggunaan}
\begin{lstlisting}[style=PythonStyle, caption={Kode Python: while_loop.py}]
# Menghitung dari 1 sampai 5
angka = 1
while angka <= 5:
    print(f"Hitungan: {angka}")
    angka = angka + 1 # atau angka += 1

print("Selesai")
\end{lstlisting}

\begin{lstlisting}[style=PythonStyle, caption={Kode Python: cowok_selalu_salah.py}]
def cewek_nanya():
    print('Cewek: Kamu salah ga?')

def respon_cowok():
    return input('Cowok: ')

def cek_jawaban_cowok(jawaban):
    if jawaban.startswith('iy'):
        return True
    else:
        return False
    
def main():
    while True:
        cewek_nanya()
        jawaban_cowo = respon_cowok()

        if cek_jawaban_cowok(jawaban_cowo):
            break

main()
\end{lstlisting}

\section{Kontrol Alur Perulangan}
Python menyediakan dua statement untuk mengontrol alur eksekusi di dalam perulangan: \texttt{break} dan \texttt{continue}.

\subsection{\texttt{break} Statement}
Statement \texttt{break} digunakan untuk menghentikan paksa (keluar dari) perulangan saat itu juga, bahkan jika kondisi perulangan masih terpenuhi.

\begin{lstlisting}[style=PythonStyle, caption={Kode Python: break_keyword.py}]
# Mencari angka 5 dalam rentang 1-10
for i in range(1, 11):
    print(i, end=' ')
    if i == 5:
        print("\nAngka 5 ditemukan, perulangan dihentikan!")
        break 
\end{lstlisting}

\subsection{\texttt{continue} Statement}
Statement \texttt{continue} digunakan untuk melewati sisa blok kode pada iterasi saat ini dan langsung melanjutkan ke iterasi berikutnya.

\begin{lstlisting}[style=PythonStyle, caption={Kode Python: continue_keyword.py}]
# Mencetak angka ganjil dari 1 sampai 10
for i in range(1, 11):
    if i % 2 == 0: # Jika angka genap
        continue   # Lewati iterasi ini dan lanjut ke angka berikutnya
    print(f"Angka ganjil: {i}")
\end{lstlisting}

\section{Perulangan Bersarang (\textit{Nested Loops})}
\textit{Nested loop} adalah sebuah perulangan yang berada di dalam perulangan lainnya. Perulangan di dalam (\textit{inner loop}) akan menyelesaikan seluruh iterasinya untuk setiap satu iterasi dari perulangan di luar (\textit{outer loop}).

Konsep ini sering digunakan untuk memproses data dalam format dua dimensi, seperti matriks atau tabel.

\begin{lstlisting}[style=PythonStyle, caption={Kode Python: nested_loop.py}]
# Outer loop untuk baris
for i in range(1, 4):  # Baris 1 sampai 3
    # Inner loop untuk kolom
    for j in range(1, 4): # Kolom 1 sampai 3
        print(f"{i}x{j} = {i*j}", end='\t')
    print() # Pindah ke baris baru setelah inner loop selesai
\end{lstlisting}

\section{Latihan Soal}
Kerjakan soal-soal di bawah ini untuk menguji pemahaman Anda.

\begin{enumerate}
    \item \textbf{Faktorial}: Buatlah sebuah program yang meminta pengguna memasukkan sebuah bilangan bulat positif, lalu hitung dan tampilkan nilai faktorial dari bilangan tersebut menggunakan perulangan \texttt{for}. ($n! = n \times (n-1) \times \dots \times 1$).
    
    \item \textbf{Tebak Angka}: Buatlah sebuah permainan tebak angka sederhana. Program akan memilih sebuah angka acak antara 1 dan 50. Pengguna diminta menebak angka tersebut. Gunakan perulangan \texttt{while} untuk terus meminta input dari pengguna hingga tebakannya benar. Berikan petunjuk "Terlalu besar" atau "Terlalu kecil" di setiap tebakan yang salah.
    
    \item \textbf{Pola Bintang - Segitiga Siku-siku}: Gunakan \textit{nested loop} untuk menampilkan pola segitiga siku-siku seperti di bawah ini (untuk tinggi 5 baris):
    \begin{verbatim}
*
**
***
****
*****
    \end{verbatim}

    \item \textbf{Pola Bintang - Segitiga Siku-siku Terbalik}: Gunakan \textit{nested loop} untuk menampilkan pola segitiga siku-siku terbalik (dengan puncak di bawah) seperti di bawah ini (untuk tinggi 5 baris):
    \begin{verbatim}
*****
****
***
**
*
    \end{verbatim}

    \item \textbf{Pola Bintang - Piramida}: Gunakan \textit{nested loop} untuk menampilkan pola piramida seperti di bawah ini (untuk tinggi 5 baris):
    \begin{verbatim}
    *
   ***
  *****
 *******
*********
    \end{verbatim}
    
    \item \textbf{Bilangan Prima}: Buatlah program untuk memeriksa apakah sebuah bilangan yang diinput oleh pengguna adalah bilangan prima atau bukan. Gunakan perulangan dan pernyataan \texttt{break} untuk efisiensi. (Bilangan prima adalah bilangan yang hanya habis dibagi 1 dan dirinya sendiri).

\end{enumerate}
